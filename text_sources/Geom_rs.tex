\documentclass[12pt,a4paper]{article}
\makeatletter
\renewcommand{\@evenhead}{}
\renewcommand{\@oddhead}{}
\renewcommand{\@evenfoot}{\hfill}
\renewcommand{\@oddfoot}{\hfill}
\makeatother
\usepackage{cmap}
\usepackage{euscript}
\usepackage[T1]{fontenc}
\usepackage[utf8]{inputenc}
\usepackage[russian]{babel}
\usepackage{amssymb, amsmath}
\usepackage[pdftex]{graphicx}
\abovedisplayskip=.4\abovedisplayskip
\belowdisplayskip=.4\belowdisplayskip
\usepackage[left=1.5cm,right=1cm,top=1cm,bottom=20mm]{geometry}
\usepackage{tikz}
%\usepackage{fancyhdr}
\usetikzlibrary{calc,intersections}

\pagestyle{plain}

\newcounter{tasknum}
\setcounter{tasknum}{0}
\def\thetasknum{{\textbf{\arabic{tasknum}}}}
\newcommand{\task}{\refstepcounter{tasknum}\vspace{5pt}\noindent \textbf{№} \thetasknum\textbf{.}~}
\newcommand{\taskk}{\refstepcounter{tasknum}\vspace{5pt}\noindent \textbf{№} \thetasknum\textbf{*.}~}
\newcommand{\taskkk}{\refstepcounter{tasknum}\vspace{5pt}\noindent \textbf{№} \thetasknum\textbf{**.}~}

\newcounter{exnum}
\setcounter{exnum}{0}
\def\theexnum{{\textbf{\arabic{exnum}}}}
\newcommand{\ex}{\refstepcounter{exnum}\vspace{5pt}\noindent \textbf{Пример} \theexnum\textbf{.}~}

\newcommand*{\sol}{\vspace{5pt}\par\noindent\textbf{\qquad Решение.}~}
\newcommand*{\comment}{\vspace{5pt}\par\noindent\textbf{\qquad Комментарий.}~}




\begin{document}


%\title{}
%\author{Бебчук Д.Е.}
%\date{}
%\maketitle
%\thispagestyle{empty}

\hrule
\tableofcontents
\bigskip
\hrule

\newpage
\part{Планиметрия}


\setcounter{tasknum}{0}
\setcounter{exnum}{0}
\section{Геометрические фигуры и их свойства}

Геометрия -- интересный предмет, но и сложный: чтобы решить задачу, нужно много знать, да еще и обладать опытом решения чего-то похожего, чтобы найти знакомые элементы, провести между ними связи и т.д. А ведь иногда нужно выполнить дополнительное построение... Иногда требуется не вычислить что-то, а доказать новое утверждение.\\
Как и вся математика, геометрия строится подобно зданию: в ее фундаменте лежат аксиомы и определения основных понятий, на них строится первый этаж -- простейшие факты, которые мы считаем очевидными (например, <<сумма длин частей, на которые поделен отрезок, равна длине отрезка>> и т.п.), затем второй этаж, содержащий менее очевидные факты и т.д. Каждый следующий уровень базируется на предыдущем, и все они стоят на фундаменте аксиом и определений.

В параграфах 1-3 этого раздела мы повторим и систематизируем некоторые понятия, которые понадобятся при углубленном изучении геометрии.\\

Вот одна из цепочек понятий, на которых мы будем основываться:
\begin{center}
прямая $\rightarrow$ отрезок $\rightarrow$ ломаная $\rightarrow$ многоугольник
\end{center}
Понятие прямой не определяется, а остальные определяются через предыдущие:\\
\textbf{Отрезок} -- часть прямой, заключенная между двумя точками. \textsl{Это нестрогое определение: надо бы еще уточнить, что значит <<заключенная между>>. В большинстве случаев нас будут интересовать только \textbf{невырожденные} отрезки, т.е. такие, у которых начало и конец не совпадают.}\\
\textbf{Ломаная} -- это последовательность отрезков $A_1A_2$, $A_2A_3$,... -- начало каждого следующего совпадает с концом предыдущего. При этом точки $A_1,A_2,A_3,\ldots$ называются \textbf{вершинами} ломаной, а отрезки $A_1A_2$, $A_2A_3$,... -- \textbf{звеньями} ломаной.\\
\textbf{Многоугольник} -- это конечная замкнутая ломаная без самопересечений, т.е. такая, которая состоит из конечного числа отрезков (их не менее трёх), и каждая вершина которой принадлежит ровно двум отрезкам (и, как следствие, является их концами). \textsl{Уточним, что никакие три подряд идущие вершины ломаной не должны лежать на одной прямой, иначе два (или более) отрезка составляют одну сторону многоугольника.}\\

Отсюда треугольник -- это многоугольник с тремя сторонами, четырехугольник -- с четырьмя и т.д. Здесь же уместно сказать о выпуклости многоугольника -- у этого понятия есть несколько равносильных определений, вот одно из них: многоугольник называется \textbf{выпуклым}, если для любых двух внутренних точек $A,B$ отрезок $AB$ также целиком лежит внутри этого многоугольника.

\subsection{Многоугольники. Замечательные отрезки и особые точки треугольника}

Давайте отметим основные <<вехи>> знакомства с многоугольниками и их свойствами:
\begin{itemize}
\item классификация треугольников по углам и сторонам (остроугольные, прямоугольные, тупоугольные; равнобедренные, равносторонние и прочие);
\item признаки равенства треугольников;
\item признаки подобия треугольников;
\item пропорциональные отрезки в прямоугольном треугольнике, основы тригонометрии;
\item замечательные отрезки в треугольнике (медианы, высоты, биссектрисы, срединные перпендикуляры) и их свойства;
\item площадь треугольника;
\item классификация четырехугольников, их свойства и признаки;
\item свойства вписанного и описанного четырехугольника;
\item площадь четырехугольника;
\item прочие многоугольники.
\end{itemize}

\ex Найдите сумму внутренних углов выпуклого $n$-угольника.
\sol Выберем произвольную точку $P$ внутри выпуклого многоугольника $A_1A_2\ldots A_n$. Отрезки $PA_1,PA_2,\ldots,PA_n$ целиком лежат внутри многоугольника (ввиду его выпуклости) и делят его на $n$ треугольников: $\triangle PA_1A_2, \triangle PA_2A_3, \ldots,\triangle PA_nA_1$. Углы этих треугольников образуют все углы исходного многоугольника (без наложений) и угол $360^\circ$ с вершиной в точке $P$, а их сумма равна $n\cdot180^\circ$ -- значит, сумма углов многоугольника равна $180^\circ n-360^\circ=180^\circ(n-2)$.\\

\ex Доказать, что биссектриса внешнего угла при вершине равнобедренного треугольника параллельна его основанию.\\
\begin{center}
\includegraphics[scale=0.6]{1.jpeg}\\
\end{center}
\sol Пусть треугольник $ABC$ -- равнобедренный: $AB=BC$. На луче $CB$ отметим точку $D$ так, что точка $B$ расположена на отрезке $CD$ -- угол $\angle ABD$ является внешним углом при вершине $A$ треугольника $ABC$. Проведем его биссектрису $BE$ и введем обозначение: $\angle A=\angle C=\alpha$, откуда $\angle ABC=180^\circ-2\alpha$ и $\angle ABD=2\alpha$, а значит, $\angle ABE=\alpha$. Прямые $AC$ и $BE$ параллельны, поскольку образуют равные накрест лежащие углы $\alpha$ с секущей $AB$. Что и требовалось доказать.

Пожалуй, большинство геометрический задач так или иначе связаны с исследованием треугольника. Казалось бы, треугольник -- <<простейший>> из многоугольников, потому что имеет минимальное число сторон. Даже если это и так, то мы скоро увидим, что всё гораздо интереснее. Вот список того, что необходимо знать о треугольнике:
\begin{enumerate}
\item \textbf{неравенство треугольника}: сумма длин двух сторон треугольника всегда больше длины третьей стороны;
\item три признака равенства треугольников;
\item три признака подобия треугольников;
\item определения и свойства медиан, биссектрис, высот и срединных перпендикуляров;
\item геометрические определения синуса, косинуса, тангенса и котангенса острого угла в прямоугольном треугольнике;
\item пропорциональные отрезки в прямоугольном треугольнике;
\item пять формул площади треугольника: $S=\frac12ah_a$, $S=\frac12ab\sin\phi$ (здесь $\phi$ -- угол между сторонами длины $a$ и $b$), $S=pr$ ($p$ -- полупериметр, $r$ -- радиус вписанной окружности), $S=\frac{abc}{4R}$ ($R$ -- радиус описанной окружности), $S=\sqrt{p(p-a)(p-b)(p-c)}$ (\textbf{формула Герона}).
\end{enumerate}

Определения будут даны в этом и последующем разделах, а свойства и теоремы будут либо доказаны, либо предложены к доказательству читателем.

Основное свойство медиан состоит в том, что они проходят через общую точку (центроид треугольника) и делятся этой точкой в отношении $2:1$, считая от соответствующих вершин треугольника. Важно и то, что медианы делят треугольник на два \textbf{равновеликих} (т.е. равных по площади) треугольника.\\

Биссектрисы тоже пересекаются в одной точке (это центр вписанной окружности, или \textbf{инцентр} треугольника).\\

\textbf{Свойство биссектрисы.} Докажите, что биссектриса делит сторону треугольника в том же отношении, в котором <<состоят>> две образующие ее угол стороны.\\
\begin{center}
\includegraphics[scale=0.6]{2.jpeg}\\
\end{center}
\textsl{Доказательство.} Итак, в треугольнике $ABC$ проведена биссектриса $CP$, причем $P\in AB$. Докажем, что $\frac{BP}{PA}=\frac{BC}{CA}$.\\
Для этого на луче $CP$ отметим точку $D$ так, что $AD\parallel CB$. Треугольники $BPC$ и $APD$ подобны по двум углам ($\angle BPC=\angle APD$, $\angle BCP=\angle ADP$), откуда $\frac{BP}{PA}=\frac{BC}{AD}$. Заметим, что $\triangle ACD$ -- равнобедренный (ввиду $\angle ACD=\angle ADC$), поэтому $AD=CA$, откуда получаем $\frac{BP}{PA}=\frac{BC}{CA}$, что и требовалось доказать.\\

Срединные перпендикуляры треугольника пересекаются в одной точке, и эта точка является центром окружности, описанной около упомянутого треугольника.\\
Высоты треугольника также пересекаются в одной точке, а треугольник, вершинами которого являются основания высот, называют \textbf{ортотреугольником}).

\textbf{Свойство высот.} Докажите, что высоты треугольника пересекаются в одной точке (эта точка называется \textbf{ортоцентром} треугольника

\textsl{Доказательство.} Пусть дан $\triangle ABC$, высоты $BE$ и $CF$ которого пересекаются в точке $H$ (эти высоты не могут быть параллельны, поскольку иначе параллельны и две стороны исходного треугольника). Прямая $AH$ пересекает сторону $BC$ в точке $D$ -- докажем, что $AD\perp BC$, т.е. $AD$ тоже является высотой $\triangle ABC$.\\
Заметим, что $\angle AFH=\angle AEH=90^\circ$, значит, четырехугольник $AEHF$ -- вписанный, отсюда $\angle AFE=\angle AHE$ -- вписанные углы, опирающиеся на одну дугу.\\ Кроме того, четырехугольник $CEFB$ -- тоже вписанный, поскольку прямоугольные треугольники $CEB$ и $CFB$ имеют общую гипотенузу, совпадающую с диаметром описанной около них окружности. Отсюда $\angle BCE=180^\circ-\angle BFE=180^\circ-(180^\circ-\angle AFE)=\angle AFE$.\\
\begin{center}
\includegraphics[scale=0.6]{10.jpeg}\\
\end{center}
Итак, $\angle AHE=\angle AFE=\angle BCE$, значит, четырехугольник $CEHD$ -- тоже вписанный, а поскольку $\angle CEH=90^\circ$, то и $\angle CDH=180^\circ-\angle CEH=90^\circ$, что и требовалось доказать.
\textsl{\comment Попутно мы доказали, что $\triangle ABC$ и $\triangle AEF$ подобны (по двум углам).}\\

\textbf{Теорема Фалеса (для непараллельных прямых).} Пусть две стороны угла пересечены параллельными прямыми $AA_1,BB_1,CC_1,DD_1$ (точки $A,B,C,D$ в указанном порядке лежат на одной стороне угла, а точки $A_1,B_1,C_1,D_1$ -- на другой). Тогда если $AB=CD$, то $A_1B_1=C_1D_1$.

\textsl{Доказательство.} Проведем через точки $A,C$ прямые $AB_2$ и $CD_2$, параллельные $A_1B_1$ ($B_2$ лежит на $BB_1$, $D_2$ лежит на $DD_1$). Тогда $AA_1B_1B_2$, $CC_1D_1D_2$ -- параллелограммы, откуда $AB_2=A_1B_1$, $CD_2=C_1D_1$.\\
Треугольники $\triangle ABB_2$ и $\triangle CDD_2$ равны по стороне прилежащим к ней углам: $\angle BAB_2=\angle DCD_2$ как соответственные ($AB_2\parallel CD_2$), $AB=CD$ по условию, $\angle ABB_2=\angle CDD_2$ как соответственные. Отсюда $AB_2=CD_2$ и, следовательно, $A_1B_1=C_1D_1$, что и требовалось доказать.\\
\begin{center}
\includegraphics[scale=0.6]{26.jpeg}\\
\end{center}

\subsection*{Задачи}

\task Верно ли, что если $\triangle ABC=\triangle A_1B_1C_1$, то $\angle A=\angle A_1$?

\task Докажите (без тригонометрии), что в прямоугольном треугольнике с углом $30^\circ$ один из катетов вдвое меньше гипотенузы.

\task Найдите угол между биссектрисами внутренних углов параллелограмма, имеющих общую сторону.

\task В равнобедренном треугольнике $ABC$ провели биссектрису $AL$ и оказалось, что треугольники $ABL$ и $ACL$ тоже равнобедренные, причем один из них подобен $\triangle ABC$. Найдите углы треугольника $ABC$.

\task В трапеции $ABCD$ с основаниями $AD$ и $BC$ проведены биссектрисы всех внутренних углов. Найдите угол между биссектрисами углов $B$ и $C$, если угол между биссектрисами углов $A$ и $D$ равен $144^\circ$.

\task На высоте $AH$ треугольника $ABC$ выбрана точка $M$. Докажите, что $AB^2-AC^2=MB^2-MC^2$.

\task Докажите \textbf{теорему Фалеса для параллельных прямых}:\\
Пусть две параллельные прямые $a\parallel b$ пересечены параллельными прямыми $AA_1,BB_1,CC_1,DD_1$ (точки $A,B,C,D$ в указанном порядке лежат на прямой $a$, а точки $A_1,B_1,C_1,D_1$ -- на прямой $b$). Тогда если $AB=CD$, то $A_1B_1=C_1D_1$.

\task Докажите, что биссектрисы треугольника пересекаются в одной точке.

\task Докажите, что срединные перпендикуляры к сторонам треугольника пересекаются в одной точке.

\task Докажите, что отношение длин отрезков, на которые чевиана делит сторону треугольника, равно отношению площадей двух треугольников, на которые эта же чевиана разбивает исходный треугольник.

\task Докажите, что если среди трех точек -- центроида, инцентра и ортоцентра треугольника -- две совпадают, то и третья совпадает с ними, а сам треугольник является правильным.

\task В треугольнике $ABC$ проведены высоты $AA_1$ и $BB_1$. Докажите, что треугольники $ABC$ $A_1B_1C$ подобны и найдите их коэффициент подобия, если $\angle ACB=30^\circ$.

\task Докажите, что биссектрисы двух внешних углов и биссектриса несмежного с ними внутреннего угла треугольника пересекаются в одной точке.

\task Окружность высекает на всех сторонах выпуклого многоугольника равные отрезки. Докажите, что в этот многоугольник можно вписать окружность.\\
\textsl{Окружность называется \textbf{вписанной} в многоугольник, если она касается всех сторон этого многоугольника.}

\taskk Стороны четырехугольника лежат на прямых $a,b,c,d$. Известны углы между некоторыми из них: $\angle(a,b)=70^\circ,\angle(b,c)=25^\circ,\angle(c,d)=75^\circ$. Найдите углы упомянутого четырехугольника.

\taskk Придумайте и докажите нетривиальный признак равенства треугольников.

\taskk Докажите, что если точка $P$ пересечения срединных перпендикуляров к сторонам треугольника расположена вне этого треугольника, то сам треугольник является тупоугольным, а если точка $P$ расположена на стороне треугольника, то треугольник -- прямоугольный, и $P$ делит его гипотенузу пополам.

\taskk Докажите, что если ортоцентр треугольника расположен вне этого треугольника, то сам треугольник является тупоугольным, а если ортоцентр расположен на стороне треугольника, то треугольник -- прямоугольный, и ортоцентр совпадает с вершиной его прямого угла.

\taskk Докажите, что высоты треугольника совпадают с биссектрисами его ортотреугольника.

\taskk Биссектриса внешнего угла при вершине $A$ треугольника $ABC$ пересекает продолжение стороны $BC$ в точке $D$. Докажите, что $\frac{DB}{DC}=\frac{AB}{AC}$.

\taskk Используя только теорему Фалеса, докажите теорему Менелая.\\
\textbf{Теорема Менелая}. На продолжении стороны $AB$ треугольника $ABC$ отмечена точка $C_1$ так, что точка $B$ расположена между точками $A$ и $C_1$. Прямая, проходящая через точку $C_1$, пересекает стороны $AC$ и $BC$ в точках $B_1$ и $A_1$ соответственно. Докажите, что $\frac{AB_1}{B_1C}\cdot\frac{CA_1}{A_1B}\cdot\frac{BC_1}{C_1A}=1$.

\taskk Придумайте и докажите нетривиальный признак равенства трапеций.

\taskkk В выпуклом $n$-угольнике провели все диагонали, и оказалось, что никакие три из них не имеют общей точки, отличной от вершины этого многоугольника. Найдите количество диагоналей и количество точек их пересечения.









\subsection{Окружности. Окружности треугольника}
\setcounter{tasknum}{0}
\setcounter{exnum}{0}

\textbf{Окружность} (не путайте её с \textbf{кругом} -- множеством точек внутри окружности) -- это множество всех точек плоскости, равноудаленных от заданной точки, называемой \textbf{центром окружности}. Дадим еще несколько сопутствующих определений:\\
\textbf{Дуга окружности} -- это часть окружности, заключенная между двумя точками окружности. \textsl{Это <<определение>> столь же нестрого, как и <<определение>> отрезка, приведенное в начале главы. Дело еще и в том, что две точки на окружности делят ее на две дуги, дополняющие друг друга до всей окружности.\\}
\textbf{Радиус окружности} -- это отрезок, соединяющий точку окружности с ее центром. \textsl{Говоря <<радиус>>, иногда имеют в виду не отрезок, а его длину.\\}
\textbf{Хорда окружности} -- это отрезок, соединяющий две точки окружности.\\
\textbf{Диаметр окружности} -- это хорда окружности, проходящая через ее центр. \textsl{Докажите, что длина диаметра вдвое больше длины радиуса.\\}
\textbf{Касательная} к окружности -- это прямая, имеющая с окружностью ровно одну общую точку. \textbf{Секущая} к окружности -- это прямая, имеющая две общие точки с окружностью.\\
\textbf{Центральный угол} окружности -- это угол, вершина которого расположена в центре окружности.\\ 
\textbf{Вписанный угол} окружности -- это угол, вершина которого лежит на окружности, а стороны пересекают эту окружность.\\


\textbf{Теорема о касательной к окружности.} Прямая $a$ касается в точке $A$ окружности $\omega$ с центром $O$. Докажите, что $OA\perp a$.

\textsl{Доказательство.} Предположим, что $OA$ не перпендикулярен прямой $a$. Тогда найдется такая (единственная) точка $B\in a$, что $OB\perp a$. Поскольку кратчайшее среди всех расстояний от всевозможных точек прямой $a$ до точки $O$ -- это длина перпендикуляра (то есть $OB$), то $OB<OA$ -- значит, точка $B$ расположена внутри окружности $\omega$ и прямая $a$ является не касательной (которая по определению имеет только одну общую точку с окружностью), а секущей, поскольку на прямой $a$ найдется точка $C$, такая, что $BA=BC$, откуда $OA=OC$. Итак, прямая $a$ имеет две общие точки с окружностью и является секущей, что противоречит условию. Значит, наше предположение было неверно, и $OA\perp a$, что и требовалось доказать.

По определению, \textbf{градусная мера дуги} окружности -- это в точности градусная мера центрального угла, опирающегося на эту дугу (т.е. стороны этого угла проходят через границы дуги).\\

\textbf{Теорема о вписанном угле.} Градусная мера дуги вдвое больше градусной меры вписанного угла, опирающегося на эту дугу.

\textsl{Доказательство.} Рассмотрим случай, когда одна из сторон вписанного угла является диаметром окружности. Пусть $AB$ -- этот диаметр окружности с центром $O$, а $C$ -- точка на этой окружности. Докажем, что $\angle ABC$ равен половине градусной меры дуги $AC$. Сначала проведем радиус $CO$ и заметим, что треугольник $BOC$ -- равнобедренный, откуда $\angle ABC=\angle BCO$. Из суммы углов треугольника $BCO$ получим $\angle BOC=180^\circ-2\angle ABC$, далее $\angle AOC=180^\circ-\angle BOC=2\angle ABC$. Учитывая, что $\angle AOC$ равен градусной мере дуги $AC$, сразу получим требуемое: вписанный угол $\angle ABC$ равен половине градусной меры дуги $AC$, на которую он опирается, что и требовалось доказать.


Можно говорить о многих окружностях, связанных с треугольником -- не только о вписанной, описанной и вневписанных, но и, например, об окружности Эйлера, -- каждая из них однозначно определена для заданного треугольника. 
Определение вписанной окружности дано в параграфе 1.1 <<Многоугольники>>.

\textbf{Описанная} около многоугольника (в частности, около треугольника) окружность -- это окружность, на которой лежат все вершины этого многоугольника.

\textbf{Вневписанная окружность} -- это окружность, касающаяся стороны треугольника и продолжений двух других его сторон.\\

\ex Докажите, что прямая, проходящая через центры вневписанных окружностей треугольника $ABC$, касающихся сторон $AB$ и $AC$, перпендикулярна прямой, проходящей через центр вписанной окружности и вершину $A$.\\
\begin{center}
\includegraphics[scale=0.6]{27.jpeg}\\
\end{center}

\textsl{Доказательство.} Пусть $O_1,O_2$ -- центры вневписанных окружностей треугольника $ABC$, касающихся сторон $AB$ и $AC$. Точки $O_1$ и $O_2$ лежат на биссектрисе внешнего угла $A$, которая перпендикулярна биссектрисе угла $A$, проходящей через центр вписанной окружности, что и требовалось доказать.\\

\ex Точки $O_1,O_2,O_3$ -- центры вневписанных окружностей треугольника. Доказать, что $\triangle O_1O_2O_3$ -- остроугольный.

\textsl{Доказательство.} Пусть $O_1$ -- центр вневписанной окружности $\triangle ABC$, касающейся стороны $BC$. Точка $O_1$ является точкой пересечения биссектрис внешних углов при вершинах $B$ и $C$, поэтому $ \angle O_1CB = \frac{180^{\circ}-\angle C}{2}$ и $ \angle O_1BC = \frac{180^{\circ}-\angle B}{2}$. Следовательно, $ \angle BO_1C =  \frac{180^{\circ}-\angle A}{2} < 90^{\circ}$. Аналогично доказываем, что остальные углы $\triangle O_1O_2O_3$ -- тоже острые.\\

\ex \textsl{Всероссийская олимпиада по математике, 2010 г.}\\Периметр треугольника $ABC$ равен $4$. На лучах $AB$ и $AC$ отмечены точки $X$ и $Y$ так, что $AX = AY = 1$.  Отрезки $BC$ и $XY$ пересекаются в точке $M$. Докажите, что периметр одного из треугольников $ABM$ и $ACM$ равен $2$.

\textsl{Доказательство.}  Поскольку отрезки $BC$ и $XY$ пересекаются, можно считать, что $AB > AX$ и $AC < AY$. Пусть вневписанная окружность $\omega$ исходного треугольника касается стороны $BC$ в точке $R$, а продолжений сторон $AB$ и $AC$ -- в точках $P$ и $Q$ соответственно.\\
\begin{center}
\includegraphics[scale=0.7]{28.jpeg}\\
\end{center}
Отрезки $AP$ и $AQ$ равны полупериметру треугольника $ABC$, то есть $2$. Отсюда следует, что $X$ и $Y$ -- середины этих отрезков. Значит, прямая $XY$ -- радикальная ось окружности $\omega$ и точки $A$. Поэтому $AM = MR$, и периметр треугольника $ACM$ равен $AC + CM + MR = AC + CR = AC + CQ = AQ = 2$, что и требовалось доказать.\\
\textsl{\textbf{Радикальная ось окружности $\omega$ и точки $A$ вне ее} -- это геометрическое место таких точек $K$, что длина касательной к $\omega$ из точки $K$ равна длине отрезка $KA$. Тот факт, что радикальная ось является прямой, будет доказан в параграфе 1.14.}

\subsection*{Задачи}

\task Докажите, что касательные к окружности, проведенные из одной точки, равны.

\task Докажите, что в выпуклый четырехугольник, суммы противоположных сторон которого равны, можно вписать окружность.

\task Докажите, что окружность однозначно задается любыми тремя своими точками.

\task Завершите доказательство теоремы о вписанном угле: докажите ее для вписанных углов, ни одна из сторон которых не проходит через центр окружности.

\task $AB$ и $CD$ -- хорды окружности, пересекающиеся в точке $P$. Докажите, что $\angle APC$ равен полусумме градусных мер дуг $AC$ и $BD$.

\task Из одной точки проведены две секущие к одной окружности. Докажите, что угол между этими секущими равен полуразности градусных мер дуг, ограниченных упомянутыми секущими.

\task \textbf{Теорема о касательной и хорде}. Докажите, что угол между касательной и хордой, имеющими общую точку, равен половине градусной меры дуги, которую стягивает эта хорда.

\task \textbf{Теорема о касательной и секущей}. Из точки $A$ к одной окружности проведены касательная $AB$ (точка $B$ лежит на окружности) и секущая, пересекающая окружность в точках $C$ и $D$. Докажите, что $AB^2=AC\cdot AD$.

\task \textbf{Теорема об отрезках пересекающихся хорд}. Хорды $AB$ и $CD$ пересекаются в точке $P$. Докажите, что $PA\cdot PB=PC\cdot PD$.

\task Докажите, что каждому треугольнику соответствуют ровно три вневписанные окружности.

\task Пусть $O$ -- центр окружности, вписанной в треугольник $ABC$, а точка $P$ -- центр его вневписанной окружности, касающейся стороны $AB$. Докажите, что точки $A,O,B,P$ лежат на одной окружности.

\task Пусть $r$ -- радиус окружности, касающейся гипотенузы и продолжений катетов прямоугольного треугольника со сторонами $a, b, c$. Докажите, что $r =  \frac{a+b+c}{2}$.

\taskk Две окружности касаются в точке $T$ внешним образом, а их общая внешняя касательная касается их в точках $A$ и $B$. Докажите, что треугольник $ABT$ -- прямоугольный.

\taskk Докажите, что из любой точки внутри окружности диаметр этой окружности виден под тупым углом, а из точки вне окружности -- под острым углом.

\taskk Докажите, что если сумма двух противоположных внутренних углов четырехугольника равна $180^\circ$, то этот четырехугольник -- вписанный.

\taskk Докажите, что точка, симметричная ортоцентру треугольника относительно его стороны, лежит на окружности, описанной около этого треугольника.

\taskk Докажите, что расстояние от вершины треугольника до его ортоцентра в два раза больше расстояния от центра описанной окружности до противолежащей стороны.









\subsection{Подобие треугольников}
\setcounter{tasknum}{0}
\setcounter{exnum}{0}

%%%%%%%%%%%%%%%%%%%%%%%%%%%%%%%%%%%%%%%%%%%%%%%%%%%%%%%%%%%%%%

\subsection*{Задачи}

\task Каждая из двух сторон треугольника разделена на семь равных частей; соответствующие точки деления соединены отрезками.
Найдите эти отрезки, если третья сторона треугольника равна 28.

\task Пусть $AA_1$ и $BB_1$ -- высоты треугольника $ABC$. Докажите, что треугольники $A_1B_1C$ и $ABC$ подобны. Чему равен их коэффициент подобия?

\task Kаждый из двух подобных треугольников разрезали на два треугольника так, что одна из получившихся частей одного треугольника подобна одной из частей другого треугольника. Bерно ли, что оставшиеся части также подобны?

\task В треугольник с основанием $a$ и высотой $h$ вписан квадрат так, что две его вершины лежат на основании треугольника, а две другие -- на боковых сторонах.
Найдите сторону квадрата.

\task В треугольник, основание которого равно 48, а высота -- 16, вписан прямоугольник с отношением сторон $5 : 9$,  причём большая сторона лежит на основании треугольника. Найдите стороны прямоугольника.

\taskk Одна из диагоналей вписанного в окружность четырёхугольника является диаметром. Докажите, что проекции противоположных сторон на другую диагональ равны.

\taskk Диагонали трапеции взаимно перпендикулярны. Докажите, что произведение длин оснований трапеции равно сумме произведений длин отрезков одной диагонали и длин отрезков другой диагонали, на которые они делятся точкой пересечения.

\taskk Три прямые, параллельные сторонам данного треугольника, отсекают от него три треугольника, причём остается равносторонний шестиугольник.
Найдите длину стороны шестиугольника, если длины сторон треугольника равны $a$, $b$ и $c$.

\taskk Биссектриса угла $A$ треугольника $ABC$ пересекает сторону $BC$ в точке $D$. Окружность радиуса 35, центр которой лежит на прямой $BC$, проходит через точки $A$ и $D$. Известно, что $AB^2-AC^2=216$, а площадь треугольника $ABC$ равна 90. Найдите радиус описанной окружности треугольника $ABC$.

\taskk Две окружности касаются внешним образом в точке $K$. Прямая, проходящая через точку $K$, пересекает первую окружность в точке $L$, а вторую -- в точке $M$. Касательная к первой окружности, проходящая через точку $L$, пересекает вторую окружность в точках $A$ и $B$ (точка $B$ лежит между $A$ и $L$). Известно, что $BM = 3$ и  $KM = 1$. Найдите длину отрезка $KL$ и расстояние от точки $L$ до центра окружности, касающейся отрезка $KB$ и продолжений отрезков $AB$ и $AK$ за точки $B$ и $K$ соответственно.

\taskkk \textsl{Неравенство Птолемея.} Дан четырёхугольник ABCD. Докажите, что в четырехугольнике $ABCD$ выполнено $AC\cdot BD\le AB\cdot CD+BC\cdot AD$.

\taskkk \textsl{Теорема Морли (Морлея).} В треугольнике $ABC$ проведены триссектрисы (лучи, делящие углы на три равные части). Ближайшие к стороне $BC$ триссектрисы углов $B$ и $C$ пересекаются в точке $A_1$; аналогично определяются точки $B_1$ и $C_1$. Докажите, что треугольник $A_1B_1C_1$ -- равносторонний.

\taskkk \textsl{Московская олимпиада по геометрии, 2008 г.} Дан треугольник $ABC$ и точки $P$ и $Q$. Известно, что треугольники, образованные проекциями $P$ и $Q$ на стороны $ABC$, подобны (соответствуют друг другу вершины, лежащие на одних и тех же сторонах исходного треугольника). Докажите, что прямая $PQ$ проходит через центр описанной окружности треугольника $ABC$.



\subsection{Ортоцентр}
\setcounter{tasknum}{0}
\setcounter{exnum}{0}

%%%%%%%%%%%%%%%%%%%%%%%%%%%%%%%%%%%%%%%%%%%%%%%%%%%%%%%%%%%%%

\subsection*{Задачи}

\task Докажите, что точки, симметричные точке пересечения высот (ортоцентру) треугольника $ABC$ относительно прямых, содержащих его стороны, лежат на описанной окружности этого треугольника.

\task Высоты треугольника $ABC$ пересекаются в точке $H$. Докажите, что радиусы окружностей, описанных около треугольников $ABC, AHB, BHC$ и $AHC$, равны между собой.

\task Докажите, что если ортоцентр делит высоты треугольника в одном и том же отношении, то этот треугольник -- правильный.

\taskk \textsl{Олимпиада по геометрии им. И.Ф.Шарыгина, 2011 г.} Точка $H$ -- ортоцентр треугольника $ABC$. Касательные, проведённые к описанным окружностям треугольников $CHB$ и $AHB$ в точке $H$, пересекают прямую $AC$ в точках $A_1$ и $C_1$ соответственно. Докажите, что $A_1H = C_1H$.

\taskk \textsl{Московская математическая олимпиада, 2012 г.} В треугольнике $ABC$ высоты или их продолжения пересекаются в точке $H$, а $R$ -- радиус его описанной окружности. Докажите, что если $\angle A\le\angle B\le\angle C$, то $AH+BH\ge 2R$.

\taskk \textsl{Олимпиада по геометрии им. И.Ф.Шарыгина, 2015 г.} Пусть $H$ -- ортоцентр остроугольного треугольника $ABC$, $O$ -- центр его описанной окружности. Серединный перпендикуляр к отрезку $BH$ пересекает стороны $BA, BC$ в точках $A_0, C_0$ соответственно. Докажите, что периметр треугольника $A_0OC_0$ равен $AC$.

\taskk \textsl{Всероссийская олимпиада по математике, 2006 г.} Через точку пересечения высот остроугольного треугольника $ABC$ проходят три окружности, каждая из которых касается одной из сторон треугольника в основании высоты. Докажите, что вторые точки пересечения окружностей являются вершинами треугольника, подобного исходному.

\taskk \textsl{Олимпиада по геометрии им. И.Ф.Шарыгина, 2021 г.} В равнобедренном треугольнике $ABC$ ($AB=BC$) проведен луч $l$ из вершины $B$. На луче $l$ внутри треугольника взяты точки $P$ и $Q$ так, что $\angle BAP=\angle QCA$. Докажите, что $\angle PAQ=\angle PCQ$.

\taskk Пусть $H$ -- точка пересечения высот треугольника $ABC$. Докажите, что расстояние между серединами отрезков $BC$ и $AH$ равно радиусу описанной окружности треугольника $ABC$.

\taskkk \textsl{Московская математическая олимпиада, 2007 г.} Стороны треугольника $ABC$ видны из точки $T$ под углами $120^\circ$. Докажите, что прямые, симметричные прямым $AT, BT$ и $CT$ относительно прямых $BC, CA$ и $AB$ соответственно, пересекаются в одной точке.\\
\textsl{Эта точка называется \textbf{точкой Торричелли} треугольника.}

\taskkk Найдите углы остроугольного треугольника $ABC$, если известно, что его биссектриса $AD$ равна стороне $AC$ и перпендикулярна отрезку $OH$, где O -- центр описанной окружности, $H$ -- ортоцентр треугольника $ABC$.




\subsection{Лемма о трезубце}
\setcounter{tasknum}{0}
\setcounter{exnum}{0}

Далее приведем еще один факт, который может пригодиться и на олимпиаде, и даже при сдаче школьных экзаменов -- \textbf{лемма Мансиона} (иногда одну из ее расширенных вариаций называют <<леммой о трезубце>>). Итак,

\textbf{Лемма о трезубце.} Пусть $O$ -- центр окружности, вписанной в треугольник $ABC$, точка $P$ -- центр его вневписанной окружности, касающейся стороны $BC$, а точка $L$ -- точка пересечения отрезка $OP$ с дугой $BC$ описанной около $\triangle ABC$ окружности (имеется в виду та дуга $BC$, на которой НЕ лежит вершина $A$). Тогда $LO=LB=LC=LP$.\\
\begin{center}
\includegraphics[scale=0.6]{4.jpeg}\\
\end{center}
\textsl{Доказательство.} Луч $AO$ (он же -- луч $AP$) -- биссектриса угла $A$ треугольника $ABC$, поэтому точка $L$ лежит на середине дуги $BC$ описанной около $\triangle ABC$ окружности -- из этого следует $BL=CL$. \\
Далее заметим, что $\angle BOL=\frac12\angle A+\frac12\angle ABC$, поскольку $\angle BOL$ -- внешний угол треугольника $AOB$. Кроме того, $\angle LBO=\angle LBC+\angle CBO=\frac12\angle A+\frac12\angle ABC$, поскольку $\angle LBC$ и $\angle LAC$ равны как вписанные, опирающиеся на одну дугу $LC$. Значит, $\triangle BLO$ -- равнобедренный, т.е. $BL=LO$. Таким образом, $BL=OL=CL$. Осталось доказать, что эти отрезки равны $PL$.\\
Продлим отрезок $AB$ за точку $B$ и выберем на этом продолжении произвольную точку $K$. Для доказательства $BL=PL$ достаточно показать, что $\angle LBP=\angle LPB$. Действительно, $\angle LBP=\frac12\angle KBC-\frac12\angle A$ и $\angle KBP=\frac12\angle A+\angle BPA$, т.к. $\angle KBP$ -- внешний в $\triangle BPA$, т.е. $\frac12\angle KBC-\frac12\angle A=\angle BPA$. Лемма доказана.

\ex Вокруг прямоугольного треугольника $ABC$ с прямым углом $C$ описана окружность, на меньших дугах $AC$ и $BC$ взяты их середины -- $K$ и $P$ соответственно. Отрезок $KP$ пересекает катет $AC$ в точке $N$. Центр вписанной окружности треугольника $ABC$ -- $I$. Найти угол $NIC$.
\begin{center}
\includegraphics[scale=0.7]{30.jpeg}\\
\end{center}
\sol По лемме о трезубце точки $K$ и $P$ являются центрами описанных окружностей треугольников $IAC, IBC$ соответственно. Значит, $KP$ является серединным перпендикуляром к отрезку $CI$. Следовательно, $N$ -- точка касания $AC$ с вписанной окружностью и $\angle NIC = 45^\circ$.

\ex Две окружности радиуса 1 пересекаются в точках $X, Y$, расстояние между которыми также равно 1. Из точки $C$ одной окружности проведены касательные $CA, CB$ к другой. Прямая $CB$ вторично пересекает первую окружность в точке $A'$. Найти расстояние $AA'$.

\sol Пусть $B'$ -- точка пересечения первой окружности $\Omega$ с прямой $CA$, $O$ -- её центр, $O'$ -- центр другой окружности $\Omega'$. Прямая $CO'$ -- биссектриса угла $ACB$, поэтому пересекает $\Omega$ в середине $K$ дуги $A'B'$. Степень точки $O'$ относительно $\Omega$ равна $O'O^2 - OС^2 = 2$,  значит, $O'K\cdot O'C = 2$. \\
Пусть $\angle A'CO' = \gamma$, тогда $\sin\gamma = \frac{O'B}{CO'} = \frac{1}{CO'}$, а $A'K = 2\cdot OC\cdot\sin\gamma = \frac{2}{CO'} = O'K$. \\
По лемме о трезубце $O'$ -- центр вневписанной окружности треугольника $A'CB'$, то есть прямая $A'B'$ касается окружности $\Omega'$ в некоторой точке $C'$. Следовательно, $\angle A'O'A = \angle AO'C' + \frac12 \angle C'O'B = \angle CB'A' + \frac12\angle CA'B'$,  $\angle O'A'O = \angle O'A'B' + \angle B'A'O = \frac{\pi}{2} - \angle C'O'A' + \frac{\pi}{2} - \angle BCA = \pi - \angle BCA -\frac12\angle CA'B' = \angle CB'A' + \frac12\angle CA'B'$. \\
Так как $O'A = OA'$, $AO'A'O$ -- равнобедренная трапеция, и $AA' = OO' = \sqrt3$.
\begin{center}
\includegraphics[scale=0.6]{29.jpeg}\\
\end{center}

\subsection*{Задачи}

\task Продолжение биссектрисы угла $B$ треугольника $ABC$ пересекает описанную окружность в точке $M$; $O$ -- центр вписанной окружности, $O_1$ -- центр вневписанной окружности, касающейся стороны $AC$. Докажите, что точки $A, C, O$ и $O_1$ лежат на окружности с центром в точке $M$.

\task Докажите \textbf{внешнюю лемму о трезубце}:\\
Точка пересечения биссектрисы внешнего угла $A$ треугольника $ABC$ с его описанной окружностью равноудалена от точек $B$, $C$ и от центров $I_B$, $I_C$ вневписанных окружностей треугольника $ABC$, касающихся сторон $AC$ и $AB$, соответственно.

\task Докажите, что точка пересечения биссектрисы угла $ABC$ с серединным перпендикуляром к $AC$ лежит на описанной окружности треугольника $ABC$. 

\task Точка $I$ -- центр вписанной окружности треугольника $ABC$, точка $M$ -- середина стороны $AC$, а точка $W$ -- середина не содержащей $C$ дуги $AB$ описанной окружности. Оказалось, что $\angle AIM = 90^{\circ}$. В каком отношении $I$ делит отрезок $CW$?

\task Докажите, что биссектриса угла треугольника делит пополам угол между радиусом описанной около этого треугольника окружности и высотой, проведённой из вершины того же угла. 

\task Вписанная окружность треугольника $ABC$ касается сторон $AB$ и $AC$ в точках $D$ и $E$ соответственно, а $O$ -- центр описанной окружности треугольника $BCI$. Докажите, что $\angle ODB = \angle OEC$.

\taskk \textbf{Формула Эйлера.} Пусть $R$ и $r$ -- радиусы соответственно описанной и вписанной окружностей треугольника. Докажите, что расстояние $d$ между центрами этих окружностей может быть вычислено по формуле $d^2=R^2-2Rr$.

\taskk \textsl{Московская олимпиада по геометрии, 2013 г.} Дан треугольник $ABC$. На продолжениях сторон $AB$ и $CB$ за точку $B$ взяты точки $C_1$ и $A_1$ соответственно так, что $AC = A_1C = AC_1$. Докажите, что описанные окружности треугольников $ABA_1$ и $CBC_1$ пересекаются на биссектрисе угла $B$.

\taskkk \textsl{Всероссийская олимпиада по математике, 2014 г.} Треугольник $ABC$ ($AB > BC$) вписан в окружность $\Omega$. На сторонах $AB$ и $BC$ выбраны точки $M$ и $N$ соответственно так, что $AM = CN$. Прямые $MN$ и $AC$ пересекаются в точке $K$. Пусть $P$ -- центр вписанной окружности треугольника $AMK$, а $Q$ -- центр вневписанной окружности треугольника $CNK$, касающейся стороны $CN$. Докажите, что середина дуги $ABC$ окружности $\Omega$ равноудалена от точек $P$ и $Q$.



\subsection{Точки Жергонна и Нагеля}
\setcounter{tasknum}{0}
\setcounter{exnum}{0}

\textbf{Чевиана} треугольника -- это отрезок, соединяющий вершину треугольника с точкой на противоположной ей стороне. Если чевианы $AA_1,BB_1,CC_1$ треугольника $ABC$ пересекаются в одной точке, то выполнена \textbf{теорема Чевы}: $\frac{AB_1}{B_1C}\cdot\frac{CA_1}{A_1B}\cdot\frac{BC_1}{C_1A}=1$. Эту теорему можно доказать разными способами, один из них обсуждается в параграфе <<Геометрия масс>>.\\

\textbf{Теорема о точке Нагеля.} Пусть $A_1,B_1,C_1$ -- точки касания вневписанных окружностей соответственно со сторонами $BC,AC,AB$ треугольника $ABC$. Докажите, что чевианы $AA_1, BB_1,CC_1$ пересекаются в одной точке (эту точку называют точкой Нагеля).\\
\begin{center}
\includegraphics[scale=0.6]{3.jpeg}\\
\end{center}
\textsl{Доказательство.} Введем обозначения: $a=BC$, $b=AC$, $c=AB$, $p=\frac{a+b+c}{2}$ -- полупериметр треугольника $ABC$. Пусть $T$ -- точка касания первой из упомянутых вневписанных окружностей с продолжением стороны $AB$. Тогда $BA_1=AT-AB=p-c$, аналогично $A_1C=p-b,CB_1=p-a,B_1A=p-c,AC_1=p-b,C_1B=p-a$. Далее 
$$\frac{AB_1}{B_1C}\cdot\frac{CA_1}{A_1B}\cdot\frac{BC_1}{C_1A}=\frac{p-c}{p-a}\cdot\frac{p-b}{p-c}\cdot\frac{p-a}{p-b}=1$$
Отсюда, согласно теореме Чевы, упомянутые чевианы пересекаются в одной точке, что и требовалось доказать.\\

\textbf{Теорема о точке Жергонна.} Пусть $A_1,B_1,C_1$ -- точки касания вписанной в треугольник $ABC$ окружности со сторонами $BC,AC,AB$ соответственно. Докажите, что чевианы $AA_1, BB_1,CC_1$ пересекаются в одной точке (эту точку называют точкой Жергонна).

\textsl{Доказательство.} $AB_1=AC_1$, $BA_1=BC_1$, $CA_1=CB_1$ как отрезки касательных, проведенных из одной точки, -- значит, $\frac{AB_1}{B_1C}\cdot\frac{CA_1}{A_1B}\cdot\frac{BC_1}{C_1A}=1$. Отсюда, согласно теореме Чевы, упомянутые чевианы пересекаются в одной точке, что и требовалось доказать.\\


\subsection*{Задачи}

\task Высота параллелограмма, проведённая из вершины тупого угла, равна 2 и делит сторону параллелограмма пополам. Острый угол параллелограмма равен $30^\circ$. Найдите диагональ, проведённую из вершины тупого угла, и углы, которые она образует со сторонами.

\task В треугольнике $ABC$ биссектриса $AH$ пересекает высоты $BP$ и $CT$ в точках $K$ и $M$ соответственно, причём эти точки лежат внутри треугольника. Известно, что
$BK : KP = 2$ и $MT : KP = 3 : 2$. Найдите отношение площади треугольника $PBC$ к площади описанного около этого треугольника круга.

\taskk Пусть $S'$ -- окружность, гомотетичная с коэффициентом $\frac12$ вписанной окружности $\omega$ треугольника относительно точки Нагеля, а $S$ -- окружность, гомотетичная окружности $\omega$ с коэффициентом $-\frac12$ относительно точки пересечения медиан. \\Докажите, что: \\а) окружности $S$ и $S'$ совпадают; \\б) окружность $S$ касается средних линий треугольника; \\в) окружность $S'$ касается прямых, соединяющих попарно середины отрезков с концами в точке Нагеля и вершинах треугольника.

\taskk \textsl{Олимпиада по геометрии им. И.Ф.Шарыгина, 2018 г.} Постройте треугольник по точке Нагеля, вершине $A$ и основанию высоты, проведенной из этой вершины.

\taskkk \textsl{Олимпиада по геометрии им. И.Ф.Шарыгина, 2012 г.} Пусть $M$ и $I$ -- точки пересечения медиан и биссектрис неравнобедренного треугольника $ABC$, а $r$ -- радиус вписанной в него окружности. Докажите, что $MI = \frac r3$ тогда и только тогда, когда прямая $MI$ перпендикулярна одной из сторон треугольника.

\taskkk \textsl{Олимпиада по геометрии им. И.Ф.Шарыгина, 2016 г.} Диагонали вписанного четырёхугольника $ABCD$ пересекаются в точке $M$. Окружность $\omega$ касается отрезка $MA$ в точке $P$, отрезка $MD$ в точке $Q$ и описанной окружности $\Omega$ четырёхугольника $ABCD$ в точке $X$. Докажите, что $X$ лежит на радикальной оси описанных окружностей $\omega_Q$ и $\omega_P$ треугольников $ACQ$ и $BDP$.




\subsection{Теоремы Чевы и Менелая}
\setcounter{tasknum}{0}
\setcounter{exnum}{0}

\textbf{Теорема Чевы}. Чевианы $AA_1,BB_1,CC_1$ треугольника $ABC$ пересекаются в одной точке. Тогда $\frac{AB_1}{B_1C}\cdot\frac{CA_1}{A_1B}\cdot\frac{BC_1}{C_1A}=1$.

\textbf{Теорема Менелая}. На продолжении стороны $AB$ треугольника $ABC$ отмечена точка $C_1$ так, что точка $B$ расположена между точками $A$ и $C_1$. Прямая, проходящая через точку $C_1$, пересекает стороны $AC$ и $BC$ в точках $B_1$ и $A_1$ соответственно. Тогда $\frac{AB_1}{B_1C}\cdot\frac{CA_1}{A_1B}\cdot\frac{BC_1}{C_1A}=1$.\\


\subsection*{Задачи}

\task Докажите теорему Чевы, используя только теорему Менелая.

\task Через точку $P$, лежащую на медиане $CC_1$ треугольника $ABC$, проведены прямые $AA_1$ и $BB_1$ (точки $A_1$ и $B_1$ лежат на сторонах $BC$ и $CA$ соответственно). Докажите, что $A_1B_1 \parallel AB$.

\taskk \textsl{Олимпиада по геометрии им. И.Ф.Шарыгина, 2021 г.} Через точку внутри треугольника провели три чевианы. Оказалось, что длины шести отрезков, на которые они разбивают стороны треугольника, образуют в каком-то порядке геометрическую прогрессию. Докажите, что длины чевиан тоже образуют геометрическую прогрессию.

\taskk Докажите \textbf{обобщенную теорему Чевы}. \\
Пусть точки $A',B',C'$ лежат на прямых $BC,CA,AB$ соответственно (точки $A,B,C$ не лежат на одной прямой). Тогда прямые $AA',BB',CC'$ параллельны или пересекаются в одной точке тогда и только тогда, когда $\frac{BA'}{A'C}\cdot\frac{CB'}{B'A}\cdot\frac{AC'}{C'B}=1$.

\taskk В треугольнике $ABC$ сторона $AB$ равна $4$, угол $CAB$ равен $30^\circ$, а радиус описанной окружности равен $3$. Докажите, что высота, опущенная из вершины $C$ на сторону $AB$, меньше $3$.

\taskk Окружность $S$ касается окружностей $S_1$ и $S_2$ в точках $A_1$ и $A_2$. Докажите, что прямая $A_1A_2$ проходит через точку пересечения общих внешних или общих внутренних касательных к окружностям $S_1$ и $S_2$.

\taskkk \textsl{Теорема Дезарга.} Прямые $AA_1, BB_1, CC_1$ пересекаются в одной точке $O$. Докажите, что точки пересечения прямых $AB$ и $A_1B_1, BC$ и $B_1C_1, AC$ и $A_1C_1$ лежат на одной прямой.




\subsection{Вписанные четырехугольники. Теорема Птолемея}
\setcounter{tasknum}{0}
\setcounter{exnum}{0}

%%%%%%%%%%%%%%%%%%%%%%%%%%%%%%%%%%%%%%%%%%%%%

\subsection*{Задачи}

\task $ABCD$ -- вписанный четырехугольник, диагонали которого перпендикулярны. $O$ -- центр описанной окружности четырехугольника $ABCD$. $P$ -- точка пересечения диагоналей. Найдите сумму квадратов диагоналей, если известны длина отрезка $OP$ и радиус окружности $R$.

\task На сторонах $AB, BC, CD , DA$ выпуклого четырёхугольника $ABCD$ отметили точки $E, F, G, H$ соответственно.
Докажите, что описанные круги треугольников $HAE$, $EBF$, $FCG$ и $GDH$ покрывают четырёхугольник $ABCD$ целиком.

\task В треугольнике $ABC$ проведены биссектрисы $AD$ и $BE$, пересекающиеся в точке $O$. Известно, что $OE = 1$, а точки $C, D, E$ и $O$ лежат на одной окружности. Найдите стороны и углы треугольника $EDO$.

\task Биссектрисы двух углов вписанного четырёхугольника параллельны. Докажите, что сумма квадратов двух сторон четырёхугольника равна сумме квадратов двух других сторон.

\taskk \textsl{Московская олимпиада по геометрии, 2013 г.} Диагонали вписанного четырёхугольника $ABCD$ пересекаются в точке $O$. Описанные окружности треугольников $AOB$ и $COD$ пересекаются в точке $M$ на стороне $AD$. Докажите, что точка $O$ -- центр вписанной окружности треугольника $BMC$.

\taskk Четырёхугольник $ABCD$, диагонали которого взаимно перпендикулярны, вписан в окружность. Перпендикуляры, опущенные на сторону $AD$ из вершин $B$ и $C$, пересекают диагонали $AC$ и $BD$ в точках $E$ и $F$ соответственно. Найдите $EF$, если $BC = 1$.

\taskk \textsl{Московская олимпиада по геометрии, 2008 г.} Противоположные стороны выпуклого шестиугольника параллельны. Hазовём высотой такого шестиугольника отрезок с концами на прямых, содержащих противолежащие стороны и перпендикулярный им. Докажите, что вокруг этого шестиугольника можно описать окружность тогда и только тогда, когда его высоты можно параллельно перенести так, чтобы они образовали треугольник.

\taskkk Основание каждой высоты треугольника проектируется на боковые стороны треугольника. Докажите, что шесть полученных точек лежат на одной окружности.


\subsection{Площади многоугольников}
\setcounter{tasknum}{0}
\setcounter{exnum}{0}

%%%%%%%%%%%%%%%%%%%%%%%%%%%%%%%%%%%%%%%%%%%%%

\subsection*{Задачи}%%%%%%%%%%%%%%%%%%%%%%%%%%%%%%%%%%%%%%%%%%%%%%

\task

\task

\task

\task

\task

\task

\task

\task




\subsection{Степень точки относительно окружности. Радикальная ось}
\setcounter{tasknum}{0}
\setcounter{exnum}{0}

Будем называть \textbf{степенью точки} $A$ относительно окружности с центром $O$ и радиусом $R$ величину $AO^2-R^2$. Нетрудно видеть, что степень точки положительна, если точка расположена вне окружности, и отрицательна, если точка находится внутри окружности. 
\begin{center}
\includegraphics[scale=0.6]{13.jpeg}\\
\end{center}

Степень точки равна квадрату длины касательной, проведенной из этой точки к заданной окружности. Из этого, в частности, следуют две известные теоремы:\\
\textbf{Теорема о касательной и секущей.} Если из одной точки проведены к окружности касательная и секущая, то произведение всей секущей на её внешнюю часть равно квадрату касательной.\\
\textbf{Теорема о двух секущих.} Если из точки, лежащей вне окружности, проведены две секущие, то произведение одной секущей на её внешнюю часть равно произведению другой секущей на её внешнюю часть.\\
Предлагаем слушателю самостоятельно доказать эти теоремы.\\

\textbf{Радикальная ось} двух окружностей -- это геометрическое место точек, степени которых относительно двух заданных окружностей равны. 

\textbf{Теорема.} Радикальная ось является прямой. 

\textsl{Доказательство.} Поскольку степень точки относительно окружности равна $x^{2}+y^{2}+Ax+By+C$, где коэффициенты $A,B,C$ определяются через координаты центра и радиус окружности, то, приравняв степени точки относительно двух окружностей, получим $x^{2}+y^{2}+A_{1}x+B_{1}y+C_{1}=x^{2}+y^{2}+A_{2}x+B_{2}y+C_{2}$, откуда $(A_{1}-A_{2})x+(B_{1}-B_{2})y+(C_{1}-C_{2})=0$, а это уравнение прямой.
\begin{center}
\includegraphics[scale=0.6]{14.jpeg}\\
\end{center}


\subsection*{Задачи}

\task На плоскости даны окружность $S$ и точка $P$. Прямая, проведенная через точку $P$, пересекает окружность в точках $A$ и $B$. Докажите, что произведение $PA \cdot PB$ не зависит от выбора прямой.

\task Окружность задана уравнением $f(x, y) = 0$, где $f (x, y) = x^2 + y^2 + ax + by + c$. Докажите, что степень точки $(x_0, y_0)$ относительно этой окружности равна $f (x0, y0)$.

\task Расстояние между центрами окружностей больше суммы их радиусов.
Докажите, что середины отрезков четырёх общих касательных этих окружностей лежат на одной прямой.

\task \textbf{Радикальный центр трех окружностей.} На плоскости даны три окружности $S_1, S_2$ и $S_3$. Докажите, что если две радикальные оси этих окружностей пересекаются в точке $Q$, то третья радикальная ось также проходит через эту точку. \\
\textsl{Точка Q называется радикальным центром этих окружностей.}

\taskk На плоскости даны три попарно пересекающиеся окружности. Через точки пересечения каждых двух из них проведена прямая.
Докажите, что эти три прямые пересекаются в одной точке или параллельны.

\taskk Докажите, что радикальная ось двух пересекающихся окружностей проходит через точки их пересечения.

\taskk Даны две неконцентрические окружности $S_1$ и $S_2$. Докажите, что множеством центров окружностей, пересекающих обе эти окружности под прямым углом, является их радикальная ось, из которой (если данные окружности пересекаются) выброшена их общая хорда.

\taskkk Внутри выпуклого многоугольника расположено несколько попарно непересекающихся кругов различных радиусов. Докажите, что многоугольник можно разрезать на маленькие многоугольники так, чтобы все они были выпуклыми и в каждом из них содержался ровно один из данных кругов.

\taskkk \textsl{Олимпиада по геометрии имени И.Ф. Шарыгина, 2008 г.} Даны две окружности. Общая внешняя касательная касается их в точках $A$ и $B$. Точки $X , Y$ на окружностях таковы, что существует окружность, касающаяся данных в этих точках, причем одинаковым образом (внешним или внутренним). Найдите геометрическое место точек пересечения прямых $AX$ и $BY$.



\subsection{Прямая Симсона и окружность Эйлера}
\setcounter{tasknum}{0}
\setcounter{exnum}{0}

\textbf{Теорема Уоллеса-Симсона.} Основания перпендикуляров, опущенных из произвольной точки $S$ описанной окружности треугольника $ABC$ на его стороны, лежат на одной прямой.\\
\begin{center}
\includegraphics[scale=0.6]{12.jpeg}\\
\end{center}
\textsl{Доказательство.} Пусть $SA_1,SB_1,SC_1$ -- упомянутые перпендикуляры (см. рис.). Для доказательства теоремы достаточно показать, что $\angle B_1C_1A=\angle BC_1A_1$. Заметим, что точки $B_1,C_1,A,S$ лежат на одной окружности с диаметром $AS$ -- значит, $\angle B_1C_1A=\angle B_1SA$. Тогда из треугольника $B_1SA$ получим $\angle B_1AS=90^\circ-\angle B_1C_1A$. Из вписанности четырехугольника $CASB$ получим $\angle CAS+\angle CBS=180^\circ$ -- значит, $\angle SBC=90^\circ+\angle B_1C_1A$ и $\angle SBA_1=90^\circ-\angle B_1C_1A$.\\
Из прямоугольного треугольника $SBA_1$ получим $\angle BSA_1=\angle B_1C_1A$. Теперь заметим, что точки $C_1,B,A_1,S$ лежат на одной окружности с диаметром $BS$ -- значит, $\angle BSA_1=\angle A_1C_1B$, из чего следует истинность утверждения теоремы.\\

\textbf{Окружность Эйлера} (она же -- окружность девяти точек) треугольника -- это окружность, проходящая через середины сторон этого треугольника.

\textbf{Теорема.} Основания высот треугольника и середины отрезков, соединяющих его вершины с ортоцентром, лежат на его окружности Эйлера.
\begin{center}
\includegraphics[scale=0.8]{22.jpeg}\\
\end{center}
\textsl{Доказательство.} %%%%%%%%%%%%%%%%%%%%%%%%%%%%%%%%%%

%%%%%%%%%%%%%%%%%%%%%%%%%Теорема Тебо и/или Фейербаха

\subsection*{Задачи}

\taskk \textsl{Олимпиада по геометрии имени И.Ф. Шарыгина, 2019 г.} В треугольнике $ABC$ точка $O$ -- центр описанной окружности, $H$ -- ортоцентр, $M$ -- середина $AB$. Прямая $MH$ пересекает прямую, проходящую через $O$ и параллельную $AB$, в точке $K$, лежащей на описанной окружности треугольника. Точка $P$ -- проекция $K$ на $AC$. Докажите, что $PH\parallel BC$.

\taskk Точки $A, B$ и $C$ лежат на одной прямой, точка $P$ -- вне этой прямой. Докажите, что центры описанных окружностей треугольников $ABP, BCP, ACP$ и точка $P$ лежат на одной окружности.

\taskk В треугольнике $ABC$ проведены высоты $BB_1$ и $CC_1$. Докажите, что если $\angle A = 45^\circ$, то $B_1C_1$ -- диаметр окружности девяти точек треугольника $ABC$.

\taskk В треугольнике $ABC$ проведена биссектриса $AD$, и из точки $D$ опущены перпендикуляры $DB'$ и $DC'$ на прямые $AC$ и $AB$; точка $M$ лежит на прямой $B'C'$, причем $DM  \perp BC$. Докажите, что точка $M$ лежит на медиане $AA_1$.

\taskk \textsl{Олимпиада по геометрии имени И.Ф. Шарыгина, 2012 г.} Дан треугольник $ABC$. Рассматриваются прямые $l$, обладающие следующим свойством: три прямые, симметричные $l$ относительно сторон треугольника, пересекаются в одной точке. Докажите, что все такие прямые проходят через одну точку.

\taskk Углы треугольника $ABC$ удовлетворяют соотношению $\sin^2A + \sin^2B + \sin^2C = 1$. Докажите, что его описанная окружность и окружность девяти точек пересекаются под прямым углом.

\taskk Найдите геометрическом место ортоцентров (точек пересечения высот) всевозможных треугольников, вписанных в данную окружность.

\taskkk На окружности фиксированы точки $P$ и $C$; точки $A$ и $B$ перемещаются по окружности так, что угол $ACB$ остается постоянным. Докажите, что прямые Симсона точки $P$ относительно треугольников $ABC$ касаются фиксированной окружности.

\taskkk \textbf{Прямая Симсона вписанного четырехугольника}. а) Докажите, что проекции точки $P$ описанной окружности четырехугольника $ABCD$ на прямые Симсона треугольников $BCD, CDA, DAB$ и $BAC$ лежат на одной прямой.\\
б) Докажите, что аналогично по индукции можно определить прямую Симсона вписанного $n$-угольника как прямую, содержащую проекции точки $P$ на прямые Симсона всех $(n - 1)$-угольников, полученных выбрасыванием одной из вершин $n$-угольника.




\subsection{Точки Брокара и Лемуана}
\setcounter{tasknum}{0}
\setcounter{exnum}{0}

%%%%%%%%%%%%%%%%%%%%%%%%%%%%%%%%%%%%%%%%%%%%%%

\subsection*{Задачи}

\taskk На сторонах треугольника $ABC$ внешним образом построены подобные ему треугольники $CA_1B$, $CAB_1$ и $C_1AB$ (углы при первых вершинах всех четырех треугольников равны и т.д.). Докажите, что прямые $AA_1, BB_1$ и $CC_1$ проходят через точку Брокара.

\taskk Докажите, что угол Брокара не превосходит $30^\circ$.

\taskk Внутри треугольника $ABC$ взята точка $M$. Докажите, что один из углов $ABM, BCM$ и $CAM$ не превосходит $30^\circ$.

\taskk Выразите длину симедианы $AS$ через длины сторон треугольника $ABC$.

\taskk Касательные к описанной окружности треугольника $ABC$ в точках $B$ и $C$ пересекаются в точке $P$. Докажите, что прямая $AP$ содержит симедиану $AS$.

\taskkk Через точку $X$, лежащую внутри треугольника $ABC$, проведены три отрезка, антипараллельных его сторонам. Докажите, что эти отрезки равны тогда и только тогда, когда $X$ -- точка Лемуана.

\taskkk Пусть $P$ -- точка Брокара треугольника $ABC$; $R_1, R_2$ и $R_3$ -- радиусы описанных окружностей треугольников $ABP, BCP$ и $CAP$. Докажите, что $R_1R_2R_3 = R^3$, где $R$ -- радиус описанной окружности треугольника $ABC$.

\taskkk Пусть $A_1, B_1$ и $C_1$ -- проекции точки Лемуана $K$ на стороны треугольника $ABC$. Докажите, что $K$ -- точка пересечения медиан треугольника $A_1B_1C_1$.







\newpage
\setcounter{tasknum}{0}
\setcounter{exnum}{0}
\section{Геометрическое место точек}

\subsection{Понятие ГМТ}

\textbf{Геометрическое место точек (ГМТ)}, обладающих определенным свойством (или набором свойств), -- это множество \textsl{всех} точек плоскости (или пространства, если речь о нём), обладающих заданным свойством. Да, всё просто, но в задачах этого раздела будет важно не только найти это множество, но и доказать, что каждая его точка обладает этим свойством, а все остальные точки не обладают им.\\

Многие фигуры могут быть определены как геометрическое место точек, обладающих определенных свойством. Например, окружность -- это геометрическое место точек плоскости, находящихся на заданном расстоянии от данной точки. Через ГМТ можно определить эллипс, параболу, гиперболу, круг, сферу, шар и т.д.

\ex Найти все точки внутри угла, равноудаленные от его сторон. 
\sol Сторона угла -- это луч, а луч -- часть прямой. Расстояние от точки до прямой -- это длина перпендикуляра, опущенного из этой точки на данную прямую.\\
Пусть дан угол $A$ и точка $P$ внутри него, находящаяся на равных расстояниях от сторон угла $\angle A$. Опустим из точки $P$ перпендикуляры $PB$ и $PC$ на стороны нашего угла и обнаружим, что $\triangle APB=\triangle APC$ по катету и гипотенузе. Значит, $\angle PAB=\angle PAC$ -- углы, лежащие напротив равных сторон в равных треугольниках. Итак, $AP$ -- биссектриса угла $\angle BAC$.\\
\begin{center}
\includegraphics[scale=0.6]{23.jpeg}\\
\end{center}
Теперь докажем, что любая точка на биссектрисе угла равноудалена от сторон этого угла. Для этого из произвольной точки $T$ на биссектрисе угла $\angle A$ опустим перпендикуляры $TD$ и $TE$ на стороны этого угла. Треугольники $ATD$ и $ATE$ равны по гипотенузе и острому углу -- значит, $TD=TE$ -- стороны, лежащие напротив равных углов в равных треугольниках. Таким образом, точка $T$ равноудалена от сторон угла $\angle EAD$.\\
Итак, мы доказали, что если точка равноудалена от сторон угла, то она лежит на биссектрисе этого угла, и наоборот, если точка лежит на биссектрисе угла, то она равноудалена от сторон этого угла. Из этого заключаем, что искомое ГМТ -- биссектриса угла.

\ex Найти геометрическое место точек, из которых проведены касательные к данной окружности, имеющие заданную длину.
\sol Ранее было доказано, что касательная к окружности перпендикулярна радиусу, проведенному в точку касания. Пусть $O$ -- центр окружности, $P$ -- точка вне окружности, $A$ -- точка, в которой одна из касательных, проходящих через точку $P$, касается окружности. Тогда треугольник $APO$ -- прямоугольный и $PO^2=AP^2+AO^2$, а поскольку длины отрезков $AP$ и $AO$ известны заранее, то длина отрезка $PO$ определяется однозначно.\\
Итак, искомое ГМТ -- это множество точек, удаленных от точки $O$ на расстояние $PO$. Такое ГМТ представляет собой окружность с центром $O$ и радиусом $PO=\sqrt{AP^2+AO^2}$.

\ex Дана прямая и точка, не лежащая на ней. Где расположены точки плоскости, равноудаленные от данных прямой и точки?
\sol Введем прямоугольную систему координат так, что данная точка совпадет с началом координат, а заданная прямая будет перпендикулярна оси ординат. Пусть прямая проходит через точку $(0,d)$, причем $d\ne0$ согласно условию (иначе прямая проходила бы через точку).\\
Пусть $A(x,y)$ -- точка, находящаяся на равных расстояниях от начала координат и от нашей прямой. Расстояние от $A$ до начала координат равно $\sqrt{x^2+y^2}$, а от прямой -- $y-d$. Приравнивая эти расстояния, получим $x^2+y^2=(y-d)^2\Rightarrow y=-\frac{1}{2d}x^2+\frac d2$ -- уравнение параболы.\\
Нетрудно убедиться, что любая точка на этой параболе обладает требуемым свойством.
\begin{center}
\includegraphics[scale=0.6]{15.jpeg}\\
\end{center}

\subsection*{Задачи}

\task Найдите геометрическое место центров окружностей, имеющих радиус 5 и проходящих через заданную точку.

\task Найдите геометрическое место точек, равноудаленных от концов заданного отрезка.

\task Дан треугольник $ABC$. Найдите геометрическое место точек $M$ со свойством $AM\le BM\le CM$.

\task Найдите геометрическое место центров окружностей, проходящих через две заданные точки.

\task Найдите геометрическое место центров окружностей заданного радиуса, касающихся данной окружности.

\task Найдите геометрическое место точек пересечения срединных перпендикуляров треугольника с заданной стороной.

\task Найдите геометрическое место точек, из которых данный отрезок виден под заданным углом.

\task Даны точки $A,B$. Найдите геометрическое место точек $P$, для которых $AP^2-BP^2$ -- постоянная величина.

\task Найдите геометрическое место точек, из которых можно провести касательную к данной дуге окружности.

\taskk Точки $D$ и $E$ лежат соответственно на сторонах $AB$ и $BC$ треугольника $ABC$. Найдите геометрическое место середин отрезков $DE$.

\taskk Найдите геометрическое место середин хорд данной окружности, проходящих через заданную точку внутри этой окружности.

\taskk Биссектриса $AL$ и срединный перпендикуляр к стороне $BC$ треугольника $ABC$ пересекаются в точке $O$. Докажите, что если точка $O$ лежит внутри треугольника $ABC$, то этот треугольник равнобедренный.

\taskk Про четырехугольник $ABCD$ известно, что его площадь равна $S$ и стороны $AB$ и $CD$ не параллельны. Где внутри этого четырехугольника расположены точки $P$, для которых \\$S_{ABP}+S_{CDP}=\frac S2$?

\taskk На заданной окружности фиксирована точка $A$. Найдите геометрическое место точек $P$, делящих хорды с концом $A$ в отношении $1:2$, считая от точки $A$.

\taskk Даны точки $A,B$. Найдите геометрическое место точек $P$, таких, что $PA+PB=c$ для заданного $c$, такого, что $|AB|<c$.

\taskk Даны точки $A$ и $B$. Найдите геометрическое место точек, каждая из которых симметрична точке $A$ относительно какой-то прямой, проходящей через точку $B$.

\taskkk Даны точки $A,B$. Найдите геометрическое место точек $P$, для которых $AP:BP=k$ для заданного положительного $k$.
\textsl{\comment Это ГМТ называется \textbf{окружностью Аполлония}.}

\taskkk По двум прямым, пересекающимся в точке $O$, с равными скоростями движутся точки $A$ и $B$ соответственно. Докажите, что существует точка $C$, такая, что в любой момент времени $AC=BC$.




%\subsection{ГМТ -- прямая или отрезок}
%%%%%%%%%%%%%%%%%%%%%%%%%%%%%%%%%%%%%%%%%%%%

%\subsection{ГМТ -- окружность или часть окружности}
%%%%%%%%%%%%%%%%%%%%%%%%%%%%%%%%%%%%%%%%%%%%

\subsection{ГМТ с ненулевой площадью}

%%%%%%%%%%%%%%%%%%%%%%%%%%%%%%%%%%%%%%%%%%%%

\subsection*{Задачи}

\task Пусть $O$ -- центр прямоугольника $ABCD$. Найдите ГМТ $M$, для которых $AM\ge OM$, $BM\ge OM$, $CM \ge OM$ и $DM \ge OM$.

\task Среди поля проходит прямая дорога, по которой со скоростью $10$ км/ч едет автобус. Укажите все точки на поле, из которых можно догнать автобус, если бежать с такой же скоростью.

\task Изобразите множество середин всех отрезков, концы которых лежат на диагоналях данного квадрата.

\task Найдите ГМТ $X$, из которых можно провести касательные к данной дуге $AB$ окружности.

\task Дан отрезок $AB$. Найдите геометрическое место вершин $C$ остроугольных треугольников $ABC$.

\task Найдите геометрическое место точек, из которых данный отрезок виден под тупым углом.

\taskk Прямые $OA$ и $OB$ перпендикулярны. Найти геометрическое место концов $M$ таких ломаных $OM$ длины 1, которые каждая прямая, параллельная $OA$ или $OB$, пересекает не более чем в одной точке.

\taskk \textsl{Московская математическая олимпиада, 1955 г.} Найти геометрическое место середин отрезков с концами на двух различных непересекающихся окружностях, лежащих одна вне другой.

\taskk \textsl{Московская математическая олимпиада, 1963 г.} Найти множество центров тяжести всех остроугольных треугольников, вписанных в данную окружность.

\taskkk \textsl{Московская математическая олимпиада, 1972 г.} Озеро имеет форму невыпуклого $n$-угольника. Докажите, что множество точек озера, из которых видны все его берега, либо пусто, либо заполняет внутренность выпуклого $m$-угольника, где $m\le n$.




\subsection{Теорема Карно. Окружность Аполлония}

%%%%%%%%%%%%%%%%%%%%%%%%%%%%%%%%%%%%%%%%%%%%

\subsection*{Задачи}

\taskk Вневписанные окружности треугольника $ABC$ касаются сторон $BC, AC$ и $AB$ в точках $A_1, B_1$ и $C_1$ соответственно. Докажите, что перпендикуляры, восставленные к этим сторонам в точках соответственно $A_1, B_1$ и $C_1$, пересекаются в одной точке.

\taskk \textsl{Московская математическая олимпиада, 1954 г.} На двух лучах $l_1$ и $l_2$, исходящих из точки $O$, отложены отрезки $OA_1$ и $OB_1$ на луче $ l_1$ и $OA_2$ и $OB_2$ на луче $l_2$; при этом $ {\frac{OA_1}{OA_2}} \ne {\frac{OB_1}{OB_2}}$. Определить геометрическое место точек $S$ пересечения прямых $A_1A_2$ и $B_1B_2$ при вращении луча $l_2$ около точки $O$ (луч $l_1$ неподвижен).

\taskkk \textsl{Всероссийская математическая олимпиада, 2008 г.} Дан выпуклый четырёхугольник $ABCD$. Пусть $P$ и $Q$ -- точки пересечения лучей $BA$ и $CD$, $BC$ и $AD$ соответственно, а $H$ -- проекция $D$ на $PQ$. Докажите, что четырёхугольник $ABCD$ является описанным тогда и только тогда, когда вписанные окружности треугольников $ADP$ и $CDQ$ видны из точки $H$ под равными углами.

\taskkk Прямая $l$ пересекает две окружности в четырех точках. Докажите, что четырехугольник, образованный касательными в этих точках, описанный, причем центр его описанной окружности лежит на прямой, соединяющей центры данных окружностей.

\taskkk Докажите, что перпендикуляры, опущенные из центров вневписанных окружностей на соответственные стороны треугольника, пересекаются в одной точке.

\taskkk Треугольник $ABC$ правильный, $P$ -- произвольная точка. Докажите, что перпендикуляры, опущенные из центров вписанных окружностей треугольников $PAB, PBC$ и $PCA$ на прямые $AB, BC$ и $CA$, пересекаются в одной точке.

\taskkk Докажите, что если перпендикуляры, восставленные из оснований биссектрис треугольника, пересекаются в одной точке, то треугольник равнобедренный.




\subsection{Кривые второго порядка как ГМТ}

%%%%%%%%%%%%%%%%%%%%%%%%%%%%%%%%%%%%%%%%%%%%

\subsection*{Задачи}%%%%%%%%%%%%%%%%%%%%%%%%%%%%%%%%%%%%%%%%%

\task

\task

\task

\task

\task

\task

\task

\task


\newpage
\setcounter{tasknum}{0}
\setcounter{exnum}{0}
\section{Преобразования плоскости}

\textbf{Преобразование плоскости} -- это взаимооднозначное отображение плоскости на себя. Именно \textsl{взаимооднозначное}: это значит, что преобразование плоскости переводит разные точки плоскости в (возможно, другие) разные точки, при этом некоторые точки плоскости могут остаться на месте, т.е. <<перейти в себя>>. Важно, что у каждой точки плоскости должен быть прообраз.\\
\textbf{Композицией} преобразований будем называть результат их последовательного применения. 
В первую очередь нам будут интересны преобразования плоскости, сохраняющие углы. Это значит, что $\angle ABC=\angle A_1B_1C_1$ для любых $A,B,C$ и их образов $A_1,B_1,C_1$ соответственно.\\
Преобразования, сохраняющие углы, называются \textbf{конформными}.

В некоторых задачах удобно провести одно-два преобразования плоскости, после чего требуемый факт оказывается очевидным, либо значение нужной величины может быть вычислено гораздо быстрее.\\

\ex Точки $A,B$ расположены по одну сторону от прямой $l$. Найдите такую точку $M\in l$, что сумма длин отрезков $AM$ и $BM$ принимает наименьшее значение из всех возможных.
\sol Эта классическая задача отлично иллюстрирует преимущества применения преобразований плоскости (в данном случае, симметрии).\\
Построим точку $A'$, симметричную точке $A$ относительно прямой $l$. Поскольку точки $A'$ и $B$ расположены по разные стороны от этой прямой, отрезок $A'B$ пересекает $l$ в некоторой точке -- назовем ее $P$ (заметим, что $AP=A'P$) и докажем, что $AP+BP$ принимает наименьшее из возможных значений. \\Для этого предположим противное: пусть существует такая точка $T$ на прямой $l$, что $AT+BT<AP+BP$. Тогда ввиду $AT=A'T$ имеем $A'T+BT<A'P+BP=A'B$, что нарушает неравенство треугольника и, следовательно, приводит к противоречию.\\
Значит, наше предположение было неверно, и $P$ -- искомая точка.\\
\begin{center}
\includegraphics[scale=0.6]{5.jpeg}\\
\end{center}


\subsection{Движения плоскости}

\textbf{Движением} называется преобразование плоскости, сохраняющее расстояния. То есть для любых точек $A,B$ и их образов $A_1,B_1$ выполнено $AB=A_1B_1$. Движение плоскости всегда сохраняет тип объекта: прямая переходит в прямую, окружность -- в окружность и т.д. Из определения следует, что прообраз фигуры равен ее образу.\\

Можно разделить движения плоскости на несколько типов:
\begin{itemize}
\item параллельный перенос на заданный вектор;
\item поворот вокруг заданной точки на заданный угол;
\item центральная симметрия относительно заданной точки;
\item осевая симметрия относительно заданной прямой (оси);
\item скользящая симметрия (композиция осевой симметрии и параллельного переноса на вектор, параллельный оси симметрии).
\end{itemize}

\ex Докажите, что при параллельном переносе окружность переходит в окружность.
\sol Пусть дана окружность с центром в точке $O$, а параллельный перенос осуществляется на вектор $\overrightarrow{v}$. Выберем на этой окружности произвольную точку $A$. Образы точек $O$ и $A$ -- соответственно такие точки $O'$ и $A'$, что $\overrightarrow{OO'}=\overrightarrow{AA'}=\overrightarrow{v}$. Тогда $OO'A'A$ -- параллелограмм, и $OA=O'A'$, т.е. точка $A'$ лежит на окружности радиуса $OA$ с центром в точке $O'$. Ввиду произвольности выбора точки $A$ делаем вывод, что образы всех точек исходной окружности образуют окружность того же радиуса, что и требовалось доказать.

\ex Докажите, что противоположные стороны шестиугольника, образованного сторонами треугольника и касательными к его вписанной окружности, параллельными сторонам, равны.\\
\begin{center}
\includegraphics[scale=0.6]{6.jpeg}\\
\end{center}
\sol Пусть дан треугольник, в котором $O$ -- центр вписанной окружности. Выполним центральную симметрию относительно точки $O$, в результате чего вписанная окружность перейдет в себя (поскольку ее центр совпадает с центром симметрии), а образы сторон треугольника будут параллельны своим прообразам и останутся касательными к вписанной окружности. Пересечение внутренностей исходного треугольника и его образа образует шестиугольник, стороны которого попарно параллельны. Кроме того, точка $O$ -- центр симметрии этого шестиугольника, что доказывает равенство его противоположных сторон.

\ex Точка $M$ лежит на диаметре $AB$ окружности. Хорда $CD$ проходит через $M$ и пересекает $AB$ под углом $45^\circ$. Докажите, что сумма $CM^2+DM^2$ не зависит от выбора точки $M$.\\
\begin{center}
\includegraphics[scale=0.6]{7.jpeg}\\
\end{center}
\sol Пусть $D'$ -- точка, симметричная $D$ относительно диаметра $AB$, при этом $DM=D'M$. Тогда $\angle CMD'=90^\circ$ и $CM^2+DM^2=CM^2+D'M^2=CD'^2$ по теореме Пифагора. Далее заметим, что $DD'\perp AB$, откуда $\angle CDD'=45^\circ$ -- вписанный угол, опирающийся на дугу $CD'$. Поскольку этот угол не зависит от выбора точки $M$, длина хорды $CD'$ также не зависит от этого выбора, что и завершает доказательство.

\subsection*{Задачи}

\task Докажите, что при повороте окружность переходит в окружность.

\task Докажите, что четырехугольник, имеющий центр симметрии, является параллелограммом.

\task Докажите, что середины сторон правильного многоугольника образуют правильный многоугольник. 

\task Существует ли фигура, не имеющая осей симметрии, но переходящая в себя при некотором повороте?

\task Через центр квадрата проведены две перпендикулярные прямые. Докажите, что их точки пересечения со сторонами квадрата образуют квадрат.

\task Дан невыпуклый четырёхугольник. Докажите, что найдётся выпуклый четырёхугольник того же периметра, но большей площади.

\taskk Двое игроков поочерёдно выкладывают на прямоугольный стол пятаки. Монету разрешается класть только на свободное место. Проигрывает тот, кто не может сделать очередной ход. Докажите, что первый игрок всегда может выиграть.

\taskk На биссектрисе внешнего угла $C$ треугольника $ABC$ взята точка $M$, отличная от $C$. Докажите, что $MA+MB>CA+CB$.

\taskk На сторонах $BC$ и $CD$ параллелограмма $ABCD$ построены внешним образом правильные треугольники $BCP$ и $CDQ$. Докажите, что треугольник $APQ$ правильный.

\taskk Докажите, что при повороте на угол $\alpha$ с центром в начале координат точка $(x,y)$ переходит в точку $(x\cos\alpha-y\sin\alpha,x\sin\alpha+y\cos\alpha)$.

\taskk Верно ли, что если фигура имеет две оси симметрии, то она имеет центр симметрии?

\taskkk Вокруг квадрата описан параллелограмм. Докажите, что перпендикуляры, опущенные из вершин параллелограмма на стороны квадрата, образуют квадрат.





\subsection{Композиция движений. Теорема Шаля}
\setcounter{tasknum}{0}
\setcounter{exnum}{0}

Композиция некоторых функций (движений, преобразований,...) -- это результат их последовательного применения. Зачастую бывает, что композиция некоторых движений может быть сведена к одному движению. Так, например,
\begin{itemize}
\item композиция параллельных переносов -- это параллельный перенос;
\item композиция поворотов с общим центром -- поворот; 
\item композиция симметрий относительно параллельных прямых -- симметрия.
\end{itemize}
Вышеприведенные факты интуитивно очевидны, и их доказательство мы оставляем слушателю. Следующие же факты уже не столь очевидны:

\textbf{Теорема.} Композиция двух симметрий относительно пересекающихся осей -- это поворот вокруг точки их пересечения на удвоенный угол между ними. 

\textbf{Теорема.} Композиция параллельного переноса на вектор $\overrightarrow{a}\ne\overrightarrow{0}$ и осевой симметрии относительно прямой $l\perp\overrightarrow{a}$, является осевой симметрией с осью, параллельной $l$.

\textbf{Теорема.} Композиция двух поворотов с различными центрами на углы соответственно $\alpha$ и $\beta$ является поворотом на угол $\alpha+\beta$, если $\alpha+\beta\ne 360^\circ\cdot k$, и параллельным переносом в противном случае.

\textbf{Теорема.} Композиция параллельного переноса и поворота на ненулевой (а значит, и не кратный $360^\circ$) угол $\alpha$ является поворотом на угол $\alpha$.\\


\textbf{Лемма (<<о трех гвоздях>>).} Докажите, что всякое движение плоскости однозначно задается тремя точками, не лежащими на одной прямой, и их образами.

Прежде чем перейти к классификации движений плоскости, необходимо разобраться в понятии \textbf{ориентации} плоскости. Его можно определить по крайней мере двумя разными способами:\\
1. Выберем на плоскости две различные системы координат. Если поворот от оси абсцисс до оси ординат на кратчайший угол осуществляется против часовой стрелки, то такую ориентацию будем называть \textsl{правой}, в противном случае -- \textsl{левой}. 
\begin{center}
\includegraphics[scale=0.8]{orient.png}\\
\end{center}
2. Выберем на плоскости два равных треугольника -- $\triangle ABC$ и $\triangle A_1B_1C_1$. Если проход по вершинам в алфавитном порядке осуществляется у этих треугольников в одном направлении (например, по часовой стрелке), то будем называть эти треугольники одинаково ориентированными, а преобразования плоскости, переводящие $\triangle ABC$ в $\triangle A_1B_1C_1$ -- \textsl{сохраняющими ориентацию}.\\

Интуитивно понятно, что каждое движение плоскости либо сохраняет ее ориентацию, либо меняет ее на противоположную. 

\textbf{Теорема Шаля (о классификации движений).} Любое сохраняющее ориентацию движение плоскости представляет собой поворот или параллельный перенос, а движение, не сохраняющее ориентацию плоскости, является осевой или скользящей симметрией.


\subsection*{Задачи}

\taskk Докажите, что композиция двух центральных симметрий является параллельным переносом.

\taskk Докажите, что композиция параллельного переноса и центральной симметрии является центральной симметрией.

\taskk Дан треугольник $ABC$. Точка $M$, расположенная внутри треугольника, движется параллельно стороне $BC$ до пересечения со стороной $CA$, затем параллельно $AB$ до пересечения с $BC$, затем параллельно $AC$ до пересечения с $AB$ и т. д. Докажите, что через некоторое число шагов траектория движения точки замкнётся.

\taskkk Докажите, что композицию чётного числа симметрий относительно прямых нельзя представить в виде композиции нечётного числа симметрий относительно прямых.




\subsection{Аффинные преобразования}
\setcounter{tasknum}{0}
\setcounter{exnum}{0}

\textbf{Аффинным} называется преобразование плоскости, при котором прямые переходят в прямые. Здесь уже не обязательно сохраняются расстояния и углы, поэтому образы фигур могут быть <<деформированными>>. Группа аффинных преобразований, переводящих фигуры в им подобные, называется \textbf{преобразованиями подобия}.\\
Движения плоскости формально тоже относятся к аффинным преобразованиям, но в этом параграфе обсуждаться не будут, разве что в составе композиции с другими преобразованиями.\\

\ex Докажите, что при аффинном преобразовании параллельные прямые переходят в параллельные.
\sol Предположим, что образы параллельных прямых имеют общую точку. Аффинное преобразование является взаимооднозначным, значит, у этой точки есть точка-прообраз, принадлежащая обеим исходным прямым, что невозможно, поскольку они параллельны.\\
Значит, наше предположение было неверно, и образы параллельных прямых параллельны, что и требовалось доказать.\\

Одно из сравнительно редких аффинных преобразований -- \textbf{растяжение плоскости} с заданным коэффициентом $k\ne0$ относительно выбранной прямой. Это преобразование, при котором образом каждой точки $A$ является такая точка $A'$, что $\overrightarrow{OA}=k\overrightarrow{OA'}$, где $O$ -- проекция точки $A$ на упомянутую прямую, т.е. основание перпендикуляра, опущенного из точки $A$ на эту прямую. Если $|k|<1$, то такое растяжение называют \textbf{сжатием}, а при $k<0$ растяжение сопровождается симметрией относительно прямой.


\subsection*{Задачи}

\task Докажите \textbf{обобщенную теорему Фалеса}:
Точки $A,B$ лежат на одной стороне угла $\angle Q$, а точки $C,D$ -- на другой стороне так, что $AC\parallel BD$. Докажите, что $\frac{QA}{AB}=\frac{QC}{CD}$.

\task Две окружности касаются в точке $K$. Прямая, проходящая через точку $K$, пересекает эти окружности в точках $A$ и $B$. Докажите, что касательные к окружностям, проведённые через точки $A$ и $B$, параллельны.

\task Две окружности касаются в точке $K$. Через точку $K$ проведены две прямые, пересекающие первую окружность в точках $A$ и $B$, вторую -- в точках $C$ и $D$. Докажите, что $AB \parallel CD$.

\task Дан квадрат $ABCD$. Точки $P$ и $Q$ лежат соответственно на сторонах $AB$ и $BC$, причём $BP = BQ$. Пусть $H$ -- основание перпендикуляра, опущенного из точки $B$ на отрезок $PC$. Докажите, что $\angle DHQ = 90^\circ$.

\task Выпуклый многоугольник обладает следующим свойством: если все его стороны отодвинуть на расстояние $1$ во внешнюю сторону, то полученные прямые образуют многоугольник, подобный исходному. Докажите, что этот многоугольник описанный.

\task \textbf{Лемма о трех гвоздях}. Докажите, что аффинное преобразование плоскости однозначно задается тремя точками, не лежащими на одной прямой, и их образами.

\task Докажите, что любое аффинное преобразование можно представить в виде композиции двух растяжений и преобразования подобия.




\subsection{Гомотетия}
\setcounter{tasknum}{0}
\setcounter{exnum}{0}

Известное аффинное преобразование -- \textbf{гомотетия}, относящаяся к преобразованиям подобия. Это преобразование, переводящее точку $A$ в такую точку $A'$, что $\overrightarrow{OA}=k\overrightarrow{OA'}$ для фиксированной точки $O$ (центр гомотетии) и $k\ne0$ (коэффициент гомотетии).\\
\begin{center}
\includegraphics[scale=0.6]{16.jpeg}\\
\end{center}
\textbf{Поворотной гомотетией} называют композицию гомотетии и поворота вокруг ее центра.\\

\ex В трапеции точка пересечения диагоналей равноудалена от прямых, на которых лежат боковые стороны. Докажите, что трапеция равнобедренная.
\sol Рассмотрим трапецию $ABCD$ с точкой пересечения диагоналей $O$, равноудаленной от боковых сторон $AB$ и $CD$, продолжения которых пересекаются в точке $P$. Сначала докажем, что точки $P,O$ и центры оснований трапеции лежат на одной прямой. Для этого выполним гомотетию с коэффициентом $\frac{BC}{AD}$ и центром в точке $P$ -- тогда $AD$ перейдет в $BC$ и центр $AD$ -- в центр $BC$. Значит, центры оснований лежат на одной прямой с точкой $P$. Теперь выполним гомотетию с коэффициентом $-\frac{BC}{AD}$ и центром в точке $O$ -- тогда снова центр $AD$ перейдет в центр $BC$ -- значит, центры оснований лежат на одной прямой с точкой $O$. В итоге заключаем, что точки $P,O$ и центры оснований трапеции лежат на общей прямой, что и требовалось доказать.\\
Теперь докажем, что $AB=CD$. Прямая $PO$, как доказано выше, проходит через середину $AD$. Кроме того, эта же прямая является биссектрисой угла $\angle APD$ согласно условию задачи. Значит, эта прямая содержит медиану и биссектрису треугольника $APD$, что делает его равнобедренным, откуда $\triangle BPC$ -- тоже равнобедренный, что влечет $AB=CD$, что и требовалось доказать.

\ex Четырёхугольник разрезан диагоналями на четыре треугольника. Докажите, что их точки пересечения медиан образуют параллелограмм.
\sol Сначала докажем, что середины сторон выпуклого четырехугольника (назовем его $ABCD$) являются вершинами параллелограмма. Отрезок с концами в серединах сторон $AB$ и $BC$ является средней линией треугольника $ABC$ -- значит, он параллеле $AC$ и равен половине этой диагонали. Значит, две стороны четырехугольника, вершины которого являются серединами сторон $ABCD$, параллельны и равны, что является признаком параллелограмма.\\
Пусть $O$ -- точка пересечения диагоналей $ABCD$. Выполним гомотетию с коэффициентом $\frac23$ и центром в точке $O$ -- тогда вершины упомянутого параллелограмма перейдут в точки пересечения медиан треугольников $ABO,BCO,CDO,DAO$ благодаря свойству медиан. Поскольку гомотетия -- это преобразования подобия, делаем вывод, что точки пересечения медиан упомянутых треугольников также являются вершинами параллелограмма, что и требовалось доказать.\\
\begin{center}
\includegraphics[scale=0.6]{8.jpeg}\\
\end{center}

\subsection*{Задачи}

\task Докажите, что при гомотетии окружность переходит в окружность.

\task На плоскости даны точки $A$ и $B$ и прямая $l$. По какой траектории движется точка пересечения медиан треугольников $ABC$, если точка $C$ движется по прямой $l$?

\taskk Окружности $S_1$ и $S_2$ пересекаются в точках $A$ и $B$. При поворотной гомотетии с центром $A$, переводящей $S_1$ в $S_2$, точка $M_1$ окружности $S_1$ переходит в точку $M_2$. Докажите, что прямая $M_1M_2$ проходит через точку $B$.

\taskk Докажите, что композиция двух гомотетий с коэффициентами $k_1$ и $k_2$, где $k_1k_2\ne1$, является гомотетией с коэффициентом $k_1k_2$, причём её центр лежит на прямой, соединяющей центры этих гомотетий. Исследуйте случай $k_1k_2=1$.

\taskk Даны две неконцентрические окружности $S_1$ и $S_2$. Докажите, что существуют ровно две поворотные гомотетии с углом поворота $90^\circ$, переводящие $S_1$ в $S_2$.




\subsection{Инверсия}
\setcounter{tasknum}{0}
\setcounter{exnum}{0}

Инверсия не является аффинным преобразованием, потому что может перевести прямую в окружность, а окружность -- в прямую. Однако, если считать прямую \textsl{обобщенной окружностью} (т.е. окружностью бесконечного радиуса), то инверсия прямой -- это прямая.\\
Пусть дана окружность с центром в точке $O$ и радиусом $R$. \textbf{Инверсия относительно окружности} -- это такое преобразование, которое переводит точку $A$ плоскости в точку $A'$ на луче $OA$, причем $OA\cdot OA'=R^2$. Величину $R^2$ называют \textbf{степенью инверсии}.\\
Оговоримся: инверсия не является в полном смысле преобразованием плоскости, поскольку не определена для самой точки $O$. <<Прищурившись>>, можно сказать, что точка $O$ перейдет в некую <<бесконечно удаленную точку>> плоскости, которая, в свою очередь, перейдет в точку $O$. Чтобы не усложнять этот параграф сверх меры, мы будем считать, что в этом смысле инверсия является преобразованием плоскости, дополненной бесконечно удаленной точкой.\\
Очевидно, что точки, лежащие на окружности инверсии, перейдут в себя. Заметим также, что если применить инверсию относительно одной и той же окружности дважды, то все точки плоскости перейдут в самих себя.\\

\ex Доказать, что инверсия является конформным преобразованием.
\sol Во-первых, заметим, что прямая, проходящая через точку $O$ (центр окружности инверсии), перейдет в себя. Действительно, каждая точка этой прямой останется лежать на этой же прямой согласно определению инверсии, а обратимость этого преобразования дает нам право утверждать, что и прообразы всех точек этой прямой лежат на ней же.\\
Теперь возьмем произвольные точки $A,B$ плоскости, отличные от точки $O$, и их образы $A',B'$ соответственно. Тогда $OA \cdot OA'=OB\cdot OB'$, откуда $\frac{OA}{OB}=\frac{OA'}{OB'}$, т.е. треугольники $OAB$ и $OA'B'$ подобны ввиду $\angle AOB=\angle A'OB'$. Значит, соответствующие углы этих треугольников равны, что и доказывает требуемое утверждение.\\
\begin{center}
\includegraphics[scale=0.6]{9.jpeg}\\
\end{center}

\ex Доказать, что при инверсии с центром $O$ прямая, не проходящая через $O$, переходит в окружность, проходящую через $O$.
\sol Опустим на прямую $l$, не проходящую через $O$, перпендикуляр $OH$ (образ точки $H$ -- точка $H'$). Пусть $A$ -- произвольная точка прямой $l$, ее образ обозначим $A'$. Поскольку $OH\perp AH$, имеем $OA'\perp A'H'$, из чего следует, что точка $A'$ лежит на окружности с диаметром $OH'$. Ввиду произвольности выбора точки $A$ делаем вывод, что прямая $l$ при инверсии перешла в окружность с диаметром $OC'$, т.е. проходящую через точку $O$, что и требовалось доказать.\\

Раз уж прямые и окружности -- это обобщенные окружности, можно определить \textbf{угол между окружностями} -- это угол между касательными к этим окружностям, проведенными через их общую точку.\\

\ex Никакие три из четырех точек $A, B, C, D$ не лежат на одной прямой. Докажите, что угол между описанными окружностями треугольников $ABC$ и $ABD$ равен углу между описанными окружностями треугольников $ACD$ и $BCD$.
\sol Сделаем инверсию с центром $A$. Интересующие нас углы будут равны соответственно углу между прямыми $B'C'$ и $B'D'$ и углу между прямой $C'D'$ и описанной окружностью треугольника $B'C'D'$. Оба этих угла равны половине дуги $C'D'$, следовательно, равны друг другу, что и требовалось доказать.

\subsection*{Задачи}

\task Докажите, что касающиеся окружности (либо окружность и прямая) переходят при инверсии в касающиеся окружности или в окружность и прямую, или в пару параллельных прямых.

\task Докажите, что при инверсии окружность, перпендикулярная окружности инверсии, переходит в себя.

\task Докажите, что при инверсии с центром $O$ окружность, не проходящая через $O$, переходит в окружность.

\task Точки $A'$ и $B'$ -- образы точек $A$ и $B$ при инверсии относительно некоторой окружности. Докажите, что точки $A$, $B$, $A'$ и $B'$ лежат на одной окружности.

\taskk Докажите, что инверсия с центром в вершине $A$ равнобедренного треугольника $ABC$ ($AB=AC$) и степенью $AB^2$ переводит основание $BC$ треугольника в дугу $BC$ описанной окружности.

\taskk Докажите, что если при инверсии относительно некоторой окружности с центром $O$ окружность $S$ переходит в окружность $S'$, то $O$ -- один из центров гомотетии окружностей $S$ и $S'$.

\taskk Докажите, что любые две окружности можно при помощи инверсии перевести в пару равных окружностей.

\taskk Докажите, что две непересекающиеся окружности $S_1$ и $S_2$ (или окружность и прямую) можно при помощи инверсии перевести в пару концентрических окружностей.

\taskkk В сегмент вписываются всевозможные пары касающихся окружностей. Найдите множество их точек касания.

\taskkk \textbf{Теорема о бабочке}. Пусть через точку $M$, являющуюся серединой хорды $PQ$ некоторой окружности, проведены две произвольные хорды $AB$ и $CD$ той же окружности. Пусть хорды $AD$ и $BC$ пересекают хорду $PQ$ в точках $X$ и $Y$. Тогда $M$ является серединой отрезка $XY$.



\subsection{Изогональное и изотомическое сопряжения}

Вы уже знаете, что существуют разные отображения плоскости на себя: те, что определяются точкой, окружностью, прямой и т.д. Но что, если преобразование плоскости определяется... треугольником?\\

%%%%%%%%%%%%%%%%%%%%%%%%% Изогональное сопряжение

Пусть дан треугольник $ABC$, у которого $A_0,B_0,C_0$ -- середины стороны $BC,AC,AB$ соответственно. Пусть также на плоскости выбрана произвольная точка $P$, не лежащая на прямых, содержащих стороны $\triangle ABC$. Тогда рассмотрим прямые $AP,BP,CP$. Пусть они пересекают прямые, содержащие противолежащие стороны треугольника, соответственно в точках $A_1,B_1,C_1$ (если прямые окажутся параллельными, точкой пересечения считается бесконечно удалённая точка прямой). Согласно теореме Чевы, $\frac{AC_1}{C_1B}=\frac{BA_1}{A_1C}=\frac{CB_1}{B_1A}=1$. Если теперь точки $A_1,B_1,C_1$ симметрично отразить относительно $A_0,B_0,C_0$ соответственно, получатся точки $A_2,B_2,C_2$ (бесконечно удалённая точка переходит сама в себя). Поскольку $AC_1=BC_2$, $AC_2=BC_1$ и так же для остальных пар точек, получаем $1=\frac{AC_1}{C_1B}=\frac{BA_1}{A_1C}=\frac{CB_1}{B_1A}=\frac{BC_2}{C_2A}=\frac{CA_2}{A_2B}=\frac{AB_2}{B_2C}$ и, согласно той же теореме Чевы, прямые $AA_2,BB_2,CC_2$ пересекаются в одной точке $P'$. Эта точка называется \textbf{изотомически сопряжённой} точке $P$ относительно треугольника $ABC$.\\
Изотомическое сопряжение устанавливает взаимно-однозначное соответствие между точками плоскости с исключёнными прямыми $AB,BC,AC$. На этих прямых соответствие не является взаимно-однозначным, так любой точке прямой $BC$ соответствует вершина $A$ и т. д.\\
\begin{center}
\includegraphics[scale=0.8]{11.jpeg}\\
\end{center}

\ex Точки $M$ и $N$ расположены соответственно на сторонах $AB$ и $AC$ треугольника $ABC$, причём $AM : MB = 1 : 2$, $AN : NC = 3 : 2$.  Прямая $MN$ пересекает продолжение стороны $BC$ в точке $F$. Найдите $CF : BC$.
\begin{center}
\includegraphics[scale=0.6]{31.jpeg}\\
\end{center}
\sol Воспользуемся теоремой Менелая: $\frac{AM}{BM}\cdot\frac{BF}{CF}\cdot\frac{CN}{AN}=1$. Используя $\frac{AM}{BM}=\frac12$ и $\frac{CN}{AN}=\frac23$, получим $BF=3\cdot CF$, откуда $\frac{CF}{BC}=\frac12$.

\ex Через точку внутри треугольника провели три чевианы. Оказалось, что длины шести отрезков, на которые они разбивают стороны треугольника, образуют в каком-то порядке геометрическую прогрессию. Докажите, что длины чевиан тоже образуют геометрическую прогрессию.

\sol Будем считать, что длина наименьшего из шести отрезков равна 1. Тогда длины остальных равны $q,q^2,q^3,q^4,q^5$, где $q\ge1$ -- знаменатель прогрессии. По теореме Чевы, произведение каких-то трех из этих отрезков равно произведению трех остальных, т.е. $\sqrt{q^{15}}$. Это возможно только при $q=1$. Значит, данный треугольник равносторонний, а чевианы являются его медианами, т.е. их длины равны, что и требовалось доказать.

\subsection*{Задачи}

\taskk 

\taskk 

\taskk 

\taskk 

\taskk Докажите, что неподвижными точками (то есть переходящими сами в себя) изотомического сопряжения являются центроид треугольника $ABC$ и точки, симметричные вершинам треугольника относительно середин противолежащих сторон.

\taskk Докажите, что точки Жергонна и Нагеля изотомически сопряжены. 

%\subsection{Задачи для практики}
%\setcounter{tasknum}{0}
%\setcounter{exnum}{0}
%
%\task
%
%\task
%
%\task
%
%\task
%
%\task
%
%\task
%
%\task
%
%\task
%
%\task
%
%\task
%
%\task
%
%\task
%
%\task
%
%\task
%
%\task
%
%\task





\newpage
\setcounter{tasknum}{0}
\setcounter{exnum}{0}
\section{Геометрические построения}

В этой главе мы будем строить фигуру, обладающую определенным набором свойств. Зачем же это нужно? Дело в том, что, выполняя построение, мы анализируем признаки и свойства фигур, а также инструментов, используемых при построении -- это важный этап освоения геометрии, и навыки, которыми мы при этом пользуемся, помогут решать множество задач, в том числе прикладных.\\

Каждая задача на построение может быть решена согласно следующей схеме:
\begin{enumerate}
\item представляем, что фигура построена, и исследуем ее свойства; 
\item воображаем построение в обратном порядке: <<как мы могли получить эту \\точку/отрезок/окружность/...?>>;
\item описываем шаги построения в прямом порядке;
\item доказываем, что изложенная последовательность шагов приводит к требуемому результату (чаще бывает, что доказательство тривиально, поэтому его пропускают);
\item исследуем вводные данные и количество возможных решений, чтобы ответить на вопросы <<при каких значениях исходных величин решение существует/единственно?>>, <<сколько решений имеет задача?>>.\\
\end{enumerate}

Какими инструментами мы будем пользоваться? Классический набор -- это циркуль и линейка, но бывает и иначе: например, многие построения можно выполнить только циркулем, либо только линейкой, либо даже т.н. двусторонней линейкой и т.д.\\
От набора инструментов, конечно, зависит и способ построения, и его сложность, а иногда и выполнимость. Например, знаменитая задача о трисекции угла (деления произвольного угла на три равные части), как доказано еще в XIX веке, не может быть решена при помощи циркуля и линейки, однако может быть решена при помощи невсиса (<<скользящей линейки>>), что было доказано еще Архимедом.\\
Инструменты построения мы будем считать \textsl{идеальными}, т.е. обладающими абсолютной точностью, но и сами построения тоже должны быть выполнены с абсолютной точностью.


\subsection{Простейшие построения циркулем и линейкой}

Циркулем будем считать инструмент, с помощью которого можно проводить окружности произвольного радиуса, либо радиуса, равного заданному (или уже построенному) отрезку. Кроме того, с помощью циркуля можно откладывать отрезки известной (либо произвольной) длины на уже построенной прямой.\\
С помощью линейки мы можем проводить произвольные прямые, либо прямые, проходящие через две заданные точки. Также можно провести произвольную прямую, проходящую через одну заданную точку.\\
Кроме того, мы можем отмечать произвольную точку плоскости \\(прямой/отрезка/окружности/дуги/...), либо точку, лежащую на пересечении уже построенных фигур.\\

\ex От прямой $l$ отложите угол, равный данному.
\sol Пусть даны угол $ABC$ (точки $A$ и $C$ выбраны произвольно на разных сторонах угла) и прямая $l$. Сначала отметим на прямой $l$ произвольную точку $B_1$, затем на этой прямой отложим от нее отрезок $B_1C_1$, равный $BC$. Построим окружности с центрами в точках $B_1,C_1$ и радиусами $AB,AC$ соответственно. Поскольку $AB+AC>BC$ (иначе треугольник $ABC$ не мог бы существовать), построенные окружности пересекутся в двух точках -- назовем одну из них $A_1$. Треугольники $ABC$ и $A_1B_1C_1$ равны по трем сторонам, поэтому угол $\angle A_1B_1C_1$ -- требуемый.\\

\ex Разделить данный отрезок пополам.
\sol Пусть дан отрезок $AB$. Построим окружности с центрами в точках $A$ и $B$ с равными радиусами, равными, например, $AB$. Поскольку сумма радиусов больше расстояния между центрами окружностей, а разность радиусов (онна равна $0$) меньше этого расстояния, делаем вывод, что окружности пересекутся в двух точках -- назовем их $C$ и $D$. Прямая $CD$ пересекает $AB$ в точке $M$ -- докажем, что эта точка является искомой серединой отрезка $AB$.\\
\begin{center}
\includegraphics[scale=0.5]{17.jpeg}\\
\end{center}
Действительно, треугольники $ABC$ и $ABD$ равны и являются равнобедренными, а значит, ввиду симметрии чертежа относительно прямой $CD$, углы $\angle AMC$ $\angle BMC$ равны и, являясь смежными, равны $90^\circ$ -- значит, $CM$ -- высота равнобедренного $\triangle AMC$, проведенная из вершины, противолежащей основанию, следовательно, $CM$ -- медиана этого треугольника, что завершает построение.\\
Из построения очевидно, что задача имеет решение для любого невырожденного отрезка $AB$.\\


Заметим, что кроме деления отрезка пополам, мы решили еще одну задачу -- задачу построения срединного перпендикуляра к отрезку.

\ex Постройте биссектрису данного угла.
\sol Пусть дан угол с вершиной $A$. построим окружность произвольного радиуса с центром $A$ -- она пересечет стороны угла в точках $B$ и $C$. Согласно алгоритму из примера 2 найдем середину отрезка $BC$ -- точку $M$. Луч $AM$ является искомой биссектрисой -- докажем это.\\
Действительно, треугольники $ABM$ и $ACM$ равны по трем сторонам -- значит, $\angle MAB=\angle MAC$, что завершает построение.

\subsection*{Задачи}

\task Опустите перпендикуляр из данной точки на данную прямую.

\task Постройте прямую, параллельную данной и проходящую через заданную точку, не лежащую на данной прямой.

\task Постройте квадрат со стороной, равной данному отрезку.

\task Постройте правильный шестиугольник с заданной стороной.

\task Постройте касательную к данной окружности, проходящую через заданную точку вне окружности.

\task Постройте прямоугольный треугольник по катету и гипотенузе.

\task Постройте отрезок длины $\sqrt n$ по заданному натуральному $n$ и заданному отрезку длины $1$.

\task Даны отрезки длины $a$, $b$, $c$. Постройте отрезки с длинами $\frac{ab}{c}$, $\sqrt{ab}$.

\task Постройте прямоугольный треугольник по гипотенузе и проекции на гипотенузу одного из катетов.

\taskk Постройте прямоугольный треугольник по гипотенузе и отношению катетов.

\taskk Постройте треугольник по двум углам и периметру (т.е. дан отрезок, длина которого равна периметру требуемого треугольника).

\taskk Постройте треугольник по двум сторонам и высоте, опущенной на третью.

\taskk Постройте треугольник по стороне и медианам, проведенным к двум другим сторонам.

\taskk Внутри угла даны точки $A$ и $B$. Постройте окружность, проходящую через эти точки и высекающую на сторонах угла равные отрезки.

\taskk Постройте ромб, две стороны которого лежат на двух данных параллельных прямых, а две другие проходят через две данные точки.



%\subsection{Построение треугольника по трем элементам}
%%%%%%%%%%%%%%%%%%%%%%%%%%%%%%%%%%%%%%%%%%%%



\subsection{Построение линейкой. Построение циркулем}

%%%%%%%%%%%%%%%%%%%%%%%%%%%%%%%%%%%%%%%%%%%%

\subsection*{Задачи}%%%%%%%%%%%%%%%%%%%%%%%%%%%%%%%%%%%%%%%%%

\task

\task

\task

\task

\task

\task

\task 

\task 




\subsection{Построение двусторонней линейкой. Построение угольником}
\setcounter{tasknum}{0}
\setcounter{exnum}{0}

Двусторонняя линейка позволяет делать всё то же, что и обычная линейка, но еще с ее помощью можно 
\begin{itemize}
\item провести прямую, параллельную заданной и находящуюся от нее на фиксированном расстоянии, равном ширине линейки;
\item через две данные точки (если расстояние между ними не меньше ширины линейки) провести пару параллельных прямых, находящихся друг от друга на расстоянии ширины линейки.
\end{itemize} 

Еще один из доступных инструментов -- прямой угол (бесконечный <<угольник>>). Он позволяет делать то же, что и линейка, но еще и 
\begin{itemize}
\item строить перпендикуляр к данной прямой, проходящий через заданную точку;
\item можно расположить угольник так, чтобы вершина его прямого угла лежала на данной прямой, а стороны проходили через две заданные точки (если такое положение угольника возможно при заданных положениях прямой и точек), и провести лучи, соответствующие сторонам угла.
\end{itemize}


\subsection*{Задачи}

\task Даны две параллельные прямые. С помощью линейки разделите пополам данный отрезок, лежащий на одной из упомянутых прямых.

\taskk Даны две параллельные прямые. С помощью линейки разделите отрезок, лежащий на одной из них, на $n$ равных частей.

\taskk С помощью двусторонней линейки разделите пополам данный отрезок.

\taskk С помощью прямого угла разделите данный отрезок пополам.

\taskk С помощью прямого угла отложите от данной прямой угол, равный заданному.

\taskk С помощью двусторонней линейки постройте перпендикуляр к данной прямой, проходящий через заданную точку.

\taskk С помощью прямого угла постройте прямую, проходящую через заданную точку параллельно данной прямой.

\taskk Даны окружность, ее диаметр $AB$ и точка $P$. С помощью линейки опустите из точки $P$ перпендикуляр на $AB$.

\taskkk Докажите, что любое построение, которое можно выполнить циркулем и линейкой, можно выполнить одним циркулем.




\subsection{Задачи для практики}
\setcounter{tasknum}{0}
\setcounter{exnum}{0}


\task Докажите, что если длины диагоналей четырехугольника равны $d_1,d_2$, то его площадь может быть вычислена по формуле $S=\frac12d_1d_2\sin\phi$, где $\phi$ -- угол между диагоналями.

\task Внутри параллелограмма $ABCD$ взята точка $O$ так, что $\angle OAD = \angle OCD$. Докажите, что $\angle OBC = \angle ODC$.

\task Докажите, что длины отрезков, на которые инцентр делит биссектрису треугольника, относятся как сумма длин сторон угла, из которого выходит биссектриса, к третьей стороне треугольника.

\task Докажите, что всякий четырехугольник с осью симметрии является либо вписанным, либо описанным.

\task Найдите геометрическое место точек пересечения медиан прямоугольного треугольника с заданной гипотенузой.

\task Пусть $P$ -- середина стороны $AB$ выпуклого четырёхугольника $ABCD$. Докажите, что если площадь треугольника $PCD$ равна половине площади четырёхугольника $ABCD$, то $BC \parallel AD$.

\task Известно, что из $n$ отрезков длины $a_1,a_2,a_3,\ldots,a_n$ можно составить $n$-угольник. Докажите, что из этих отрезков можно составить выпуклый $n$-угольник.

\taskk С помощью циркуля и линейки постройте точку, инверсную данной точке относительно заданной окружности.

\taskk С помощью циркуля и линейки разделите угол $67^\circ$ на 67 равных частей.

\taskk Пользуясь только неравенством треугольника, докажите, что из $n$ отрезков с длинами $a_1,a_2,a_3,\ldots,a_n$ можно составить $n$-угольник тогда и только тогда, когда длина наибольшего из этих отрезков меньше суммы длин всех остальных. 

\taskk С помощью циркуля и линейки найдите центр заданной окружности.

\taskk Внутри выпуклого многоугольника выбраны точки $A,B$. Докажите, что найдется вершина $V$ многоугольника со свойством $VA<VB$.

\taskk Постройте окружность, проходящую через две данные точки и касающуюся данной окружности (или прямой).

\taskk Постройте окружность, касающуюся данной окружности $S$ и перпендикулярную двум данным окружностям $S_1$ и $S_2$.

\taskkk Докажите, что инверсия относительно окружности Аполлония точек $A$ и $B$ меняет эти точки местами.

\taskkk Верно ли, что при любом преобразовании подобия, не являющемся параллельным переносом, найдется точка, совпадающая со своим образом?

\taskkk Постройте правильный пятиугольник с помощью циркуля и линейки.

\taskkk Докажите, что с помощью циркуля и линейки можно построить любой угол, величина которого кратна $3^\circ$.

\taskkk Постройте треугольник по его точке Нагеля, вершине $A$ и основанию высоты, проведенной из этой вершины.







\newpage
\part{Векторы и координатные методы}
\setcounter{section}{0}

\section{Векторы}

\subsection{Арифметические операции с векторами}
\setcounter{tasknum}{0}
\setcounter{exnum}{0}

Вот два эквивалентных (равносильных) определения \textsl{свободного} вектора:\\
\textbf{Вектор} -- это
\begin{itemize}
\item направленный отрезок, т.е. отрезок, концы которого неравнозначны. Например, $\overrightarrow{AB}$ и $\overrightarrow{BA}$ -- разные векторы.
\item пара чисел $(a,b)$ для фиксированной системы координат, совпадающая с координатами конца вектора, если его начало совпадает с началом координат (это определение вектора на плоскости; вектор в трехмерном пространстве отличается от плоского вектора наличием третьей координаты). 
\end{itemize}

Вектор называется \textbf{свободным}, если его можно перемещать параллельно самому себе. В дальнейшем мы будем говорить только о таких векторах, но перед этим вот пример закрепленного (<<несвободного>>) вектора -- вектор силы, приложенной к материальному телу: очевидно, что если приложить силу к другой точке, то картина физического явления может измениться.\\

Вектор характеризуется длиной (модулем) и направлением. Исключение -- нулевой вектор (т.е. тот, у которого начало и конец совпадают, и которому соответствует пара $(0,0)$): его направление не определено. Длина вектора равна длине соответствующего отрезка, и выражается числом $\sqrt{a^2+b^2}$, где $(a,b)$ -- координаты вектора.\\
\begin{center}
\includegraphics[scale=0.6]{18.jpeg}\\
\end{center}
Здесь и далее координаты вектора заданы в т.н. \textsl{ортонормированной системе координат}: её оси попарно перпендикулярны, а единичные отрезки на разных осях равны по длине.\\

Ненулевые векторы называются \textbf{коллинеарными}, если можно расположить их на параллельных прямых. Понятно, что <<расположить>> вектор на прямой -- значит перенести его параллельно самому себе так, чтобы оба его конца лежали на заданной прямой. Конечно, это не всегда можно сделать: мы не можем менять угол между вектором и прямой и, если изначально этот угол был ненулевым, то расположить вектор на прямой не удастся. \\Проще говоря, ненулевые векторы $\overrightarrow{AB}$ и $\overrightarrow{CD}$ коллинеарны, если $AB\parallel CD$. \\
Коллинеарные векторы делятся на \textbf{сонаправленные} (направленные <<в одну сторону>>) и \textbf{противоположно направленные}.\\
Пусть даны ненулевые векторы $(a,b)$ и $(c,d)$. Тогда эти векторы коллинеарны, если $ad=bc$. Можно сказать иначе: векторы $(a,b)$ и $(c,d)$ коллинеарны, если найдется $k\ne0$, такое, что $(ka,kb)=(c,d)$. При этом если $k>0$, то векторы сонаправлены, а если $k<0$, то противоположно направлены.\\
Для векторов в пространстве есть понятие компланарности: \textbf{компланарными} называются три (или более трёх) вектора, которые параллельны одной плоскости.\\

Вот арифметические операции, которые мы будем проделывать с векторами:
\begin{itemize}
\item \textbf{умножение на число}. Представим, что вектор $\overrightarrow{AB}$ (с координатами $(x,y)$) требуется умножить на число $k$. Если $k=0$, то получаем нулевой вектор, а если $k\ne0$, то длина вектора умножается на $|k|$, и направление меняется на противоположное, если $k<0$. Координаты нового вектора равны $(kx,ky)$;
\item \textbf{сложение векторов}. $\overrightarrow{AB}+\overrightarrow{BC}=\overrightarrow{AC}$ -- <<правило параллелограмма>>. В координатах сложение векторов выглядит вполне интуитивно: $(a,b)+(c,d)=(a+c,b+d)$. \textbf{Вычитание} вектора определяется как прибавление противоположного (т.е. предварительно умноженного на $-1$): $\overrightarrow{AC} -\overrightarrow{BC}=\overrightarrow{AC}+(-1)\cdot\overrightarrow{BC}=\overrightarrow{AC}+\overrightarrow{CB}=\overrightarrow{AB}$;
\begin{center}
\includegraphics[scale=0.6]{19.jpeg}\\
\end{center}
\item \textbf{скалярное произведение}. Это число, равное произведению длин векторов на косинус угла между ними: $\overrightarrow{AB}\cdot \overrightarrow{AC}=|\overrightarrow{AB}|\cdot|\overrightarrow{AC}|\cdot\cos\angle BAC$. В прямоугольной системе координат скалярное произведение вычисляется так: $(a,b)\cdot(c,d)=ac+bd$. Зачем оно нам нужно? С его помощью можно очень быстро найти угол между векторами (а значит, между прямыми и даже плоскостями) -- для этого делим скалярное произведение векторов на произведение их длин: $\cos\angle BAC=\frac{\overrightarrow{AB}\cdot \overrightarrow{AC}}{|\overrightarrow{AB}|\cdot|\overrightarrow{AC}|}$. Если хотя бы один из векторов -- нулевой, то возникнет проблема ввиду того, что направление нулевого вектора не определено. По знаку скалярного произведения можно безошибочно определить, является ли угол между векторами острым (тогда скалярное произведение положительно) или тупым (если скалярное произведение отрицательно). Если векторы перпендикулярны, то их скалярное произведение равно $0$, как и косинус прямого угла.
\end{itemize}

Аналогичным образом можно ввести те же операции для векторов в пространстве. Геометрически всё выглядит так же, а изменения в координатах небольшие и связаны с введением третьей координаты (по оси аппликат, которая перпендикулярна осям абсцисс ($Ox$) и ординат ($Oy$) и часто обозначается $Oz$).\\
Модуль вектора $(a,b,c)$ можно найти по формуле $|(a,b,c)|=\sqrt{a^2+b^2+c^2}$ (эту формулу часто ошибочно называют <<теоремой Пифагора в пространстве>>), а скалярное произведение вычисляется так: $(a,b,c)\cdot(u,v,w)=a\cdot u+b\cdot v+ c\cdot w$.\\
Можно отметить, что нам не важна ориентация координатных осей: можно ввести их направленными в любую сторону с условием их попарной перпендикулярности. Однако, в некоторых задачах высшей математики ориентация системы координат имеет значение -- например, в задачах, где важен знак \textbf{векторного произведения} векторов (не путайте его со скалярным произведением!).

Свойства упомянутых операций (кроме векторного произведения) аналогичны таковым у привычных операций с числами: коммутативность (<<от перемены мест слагаемых...>>), ассоциативность ($\overrightarrow{a}+(\overrightarrow{b}+\overrightarrow{c})=(\overrightarrow{a}+\overrightarrow{b})+\overrightarrow{c}$), дистрибутивность ($\overrightarrow{a}\cdot(\overrightarrow{b}+\overrightarrow{c})=\overrightarrow{a}\cdot\overrightarrow{b}+\overrightarrow{a}\cdot\overrightarrow{c}$).\\

Еще одна полезная операция -- проецирование одного вектора ($\overrightarrow{a}$) на другой ненулевой вектор ($\overrightarrow{b}$): 
\begin{center}
\includegraphics[scale=0.8]{24.jpeg}\\
\end{center}
\begin{center}
Пр$_{\overrightarrow{b}}\overrightarrow{a}=\frac{|\overrightarrow{a}|\cdot\cos\angle(\overrightarrow{a},\overrightarrow{b})}{|\overrightarrow{b}|}\cdot\overrightarrow{b}$\\
|Пр$_{\overrightarrow{b}}\overrightarrow{a}|=\frac{\overrightarrow{a}\overrightarrow{b}}{|\overrightarrow{b}|}=|\overrightarrow{a}|\cdot\cos\angle(\overrightarrow{a},\overrightarrow{b})$
\end{center}


\ex Найти косинус угла между векторами $(1,0)$ и $(4,-3)$.
\sol Сначала вычислим длины этих векторов: $|(1,0)|=\sqrt{1^2+0^2}=1$; $|(4,-3)|=\sqrt{4^2+(-3)^2}=5$. Теперь вычислим их скалярное произведение: $(1,0)\cdot(4,-3)=1\cdot4+0\cdot(-3)=4$. Осталось найти косинус угла $\phi$ между векторами: $\cos\phi=\frac{(1,0)\cdot(4,-3)}{|(1,0)|\cdot|(4,-3)|}=\frac{4}{1\cdot5}=\frac45$.\\

\ex Точки $K,L$ -- соответственно середины сторон $AB$ и $BC$ квадрата $ABCD$. Докажите, что $AL\perp KD$.
\begin{center}
\includegraphics[scale=0.6]{20.jpeg}\\
\end{center}
\sol Введем систему координат следующим образом: начало координат -- точка $A$, направление оси абсцисс совпадает с $\overrightarrow {AD}$, направление оси ординат -- с $\overrightarrow {AB}$, а сторону квадрата примем равной $2a$. Запишем координаты некоторых точек: $A(0,0)$; $L(a,2a)$; $K(0,a)$; $D(2a,0)$.\\
Теперь найдем координаты векторов $\overrightarrow {AL}$ и $\overrightarrow {KD}$: $\overrightarrow {AL}=L-A=(a,2a)-(0,0)=(a,2a)$; $\overrightarrow {KD}=D-K=(2a,0)-(0,a)=(2a,-a)$. Скалярное произведение этих (кстати, ненулевых) векторов равно $(a,2a)\cdot(2a,-a)=a\cdot2a+2a\cdot(-a)=0$ -- значит, векторы перпендикулярны, что и требовалось доказать.\\

Здесь мы использовали <<формулу>> для вычисления координат вектора по координатам его концов: $\overrightarrow {AB} = B-A$. Пожалуй, требуется уточнить, что имелось в виду. Для этого обозначим начало координат за $O$ (<<origin>>) и вспомним координатное определение вектора: <<вектор -- это пара чисел $(a,b)$ для фиксированной системы координат, совпадающая с координатами конца вектора, если его начало совпадает с началом координат>>. Значит, координаты вектора $\overrightarrow {OA}$ совпадают с координатами точки $A$, а координаты $\overrightarrow {OB}$ -- с координатами точки $B$. Осталось найти разность векторов: $\overrightarrow {OB}-\overrightarrow {OA}=\overrightarrow {AB}$. Именно последнее равенство мы имеем в виду, когда пишем $\overrightarrow {AB}=B-A$.\\
\begin{center}
\includegraphics[scale=0.6]{21.jpeg}\\
\end{center}
Отметим, что векторы $\overrightarrow {OA}$ и $\overrightarrow {OB}$ называются \textbf{радиус-векторами} точек $A$ и $B$.

\ex Найдите угол между диагоналями параллелограмма, стороны которого относятся как $3:5$, а косинус угла между этими сторонами равен $\frac{\sqrt{33}}{15}$.
\sol Последовательно обозначим вершины параллелограмма буквами $A,B,C,D$, причем $\alpha=\angle BAC$ и (согласно условию) $\cos\alpha=\frac{\sqrt{33}}{15}$. Пусть $AB=3x$, $AD=5x$. Введем систему координат: начало координат совпадет с точкой $A$, ось абсцисс сонаправлена с вектором $\overrightarrow{AD}$, а ось ординат перпендикулярна оси абсцисс и направлена так, что ее положительное направление образует острый угол с вектором $\overrightarrow{AB}$. Тогда вектор $\overrightarrow{AB}$ будет иметь координаты $(3x\cos\alpha,3x\sin\alpha)$, вектор $\overrightarrow{AD}$ -- координаты $(5x,0)$. На диагоналях параллелограмма лежат векторы $\overrightarrow{AC}=\overrightarrow{AB}+\overrightarrow{AD}=(5x+3x\cos\alpha,3x\sin\alpha)$ и $\overrightarrow{BD}=\overrightarrow{AD}-\overrightarrow{AB}=(5x-3x\cos\alpha,-3x\sin\alpha)$.\\
Скалярное произведение этих векторов равно $\overrightarrow{AC}\cdot\overrightarrow{BD}=25x^2-9x^2\cos^2\alpha-9x^2\sin^2\alpha=25x^2-9x^2\cdot(\sin^2\alpha+\cos^2\alpha)=16x^2$. Модуль вектора $\overrightarrow{BD}$ равен $\sqrt{25x^2-30x^2\cos^2\alpha+9x^2\cos^2\alpha+9x^2\sin^2\alpha}=x\cdot\sqrt{34-30\cos^2\alpha}$, аналогично $|\overrightarrow{AC}|=x\cdot\sqrt{34+30\cos^2\alpha}$.\\
Тогда косинус угла между векторами $\overrightarrow{AC}$ и $\overrightarrow{BD}$ (т.е. между диагоналями параллелограмма) равен $\frac{16x^2}{x^2\cdot\sqrt{1156-900\cos^2\alpha}}=\frac{16}{\sqrt{1024}}=\frac12$. Острый угол, косинус которого равен $\frac12$, составляет $60^\circ$ -- это и есть ответ на вопрос задачи.\\

\ex $ABCD$ -- трапеция с основаниями $AD$ и $BC$, причем $AD=3\cdot BC$, а точка $P$ -- середина ее средней линии. Пусть $\overrightarrow{a}=\overrightarrow{AD}$, $\overrightarrow{b}=\overrightarrow{AB}$. Выразите через $\overrightarrow{a},\overrightarrow{b}$ векторы $\overrightarrow{CD},\overrightarrow{AP}$.
\begin{center}
\includegraphics[scale=0.8]{25.jpeg}\\
\end{center}
\sol Из условия следует, что $\overrightarrow{BC}=\frac13\overrightarrow{a}$. Далее $\overrightarrow{CD}=\overrightarrow{a}-\overrightarrow{b}-\frac13\overrightarrow{a}=\frac23\overrightarrow{a}-\overrightarrow{b}$.\\
$\overrightarrow{AP}=\overrightarrow{AT}+\overrightarrow{TP}$, где $T$ -- середина стороны $AB$. $|\overrightarrow{TP}|=\frac12\cdot\frac{|\overrightarrow{AD}|+|\overrightarrow{BC}|}{2}$ и $\overrightarrow{TP}\parallel\overrightarrow{a}$ (из свойств средней линии трапеции), откуда $\overrightarrow{TP}=\frac13\overrightarrow{a}$. Учитывая $\overrightarrow{AT}=\frac12\overrightarrow{b}$, получим $\overrightarrow{AP}=\frac13\overrightarrow{a}+\frac12\overrightarrow{b}$.\\

\subsection*{Задачи}

\task Запишите координаты точки $B$, если известны координаты $A(-5,3)$ и $\overrightarrow{AB}(2,-4)$.

\task Найдите косинус угла между векторами, заданными их координатами: $(12,-5)$; $(-20,12)$.

\task Дан вектор $\overrightarrow{v}(3,8)$. Найдите координаты вектора $\overrightarrow{w}$ длины $10$, составляющего с вектором $\overrightarrow{v}$ угол $150^\circ$. 

\task Известно, что векторы $(2x,3)$ и $(8,3x)$ коллинеарны. Найдите $x$.

\task Даны точки $A(2,-7)$ и $B(-7,-4)$. Найдите координаты точки $P$, которая делит отрезок $AB$ в отношении $2:3$, считая от точки $B$. 

\task $ABCD$ -- параллелограмм, а точки $M$ -- середина $BC$. Пусть $\overrightarrow{a}=\overrightarrow{AC}$, $\overrightarrow{b}=\overrightarrow{AB}$. Выразите через $\overrightarrow{a}$ и $\overrightarrow{b}$ векторы $\overrightarrow{CD},\overrightarrow{AD},\overrightarrow{NA}$.

\task \textbf{Теорема косинусов.} Используя скалярное произведение, докажите, что для треугольника $ABC$ выполнено $AB^2=BC^2+AC^2-2\cdot BC\cdot AC\cdot\cos\angle ACB$.

\task Докажите, что если $M$ -- \textbf{центроид} (точка пересечения медиан) треугольника $ABC$, то для произвольной точки $K$ плоскости выполнено $\overrightarrow{KM}=\frac13(\overrightarrow{KA}+\overrightarrow{KB}+\overrightarrow{KC})$.

\task Докажите, что из медиан произвольного треугольника можно составить треугольник.

\task Стороны треугольника $ABC$ параллельны медианам треугольника $A_1B_1C_1$. Докажите, что стороны треугольника $A_1B_1C_1$ параллельны медианам треугольника $ABC$.

\taskk $ABCD$ -- прямоугольник, на сторонах $AB$ и $BC$ которого отмечены точки $M$ и $N$ соответственно. Известно, что $AB=CN=\frac25 BC$ и $CM\perp DN$. В каком отношении точка $M$ делит отрезок $AB$?

\taskk При каких $k$ векторы $(2,k,5),(k+1,3,-6),(2,1,0)$ компланарны?

\taskk Докажите, что для каждого $t\in\mathbb R$ найдется такая точка $M$ на прямой $AB$, что для произвольной точки $P$ выполнено $\overrightarrow{PM}=t\cdot\overrightarrow{PA}+(1-t)\cdot\overrightarrow{PB}$. 

\taskk Докажите, что для каждой точки $M$ на прямой $AB$ с фиксированной точкой $P\in AB$ найдется такое $t\in\mathbb R$, что $\overrightarrow{PM}=t\cdot\overrightarrow{PA}+(1-t)\cdot\overrightarrow{PB}$.

\taskk Докажите, что если диагонали четырехугольника перпендикулярны, то диагонали любого другого четырехугольника с теми же длинами сторон перпендикулярны.

%\subsection{Скалярное произведение. Проекция}
%%%%%%%%%%%%%%%%%%%%%%%%%%%%%%%%%%%%%%%%%%%%

%\subsection{Геометрические неравенства}
%%%%%%%%%%%%%%%%%%%%%%%%%%%%%%%%%%%%%%%%%%%%

%\subsection{Векторное и смешанное произведения векторов}
%%%%%%%%%%%%%%%%%%%%%%%%%%%%%%%%%%%%%%%%%%%%



\newpage
\section{Метод координат. Элементы геометрии масс}

\subsection{Уравнения прямой, окружности, плоскости, сферы}
\setcounter{tasknum}{0}
\setcounter{exnum}{0}

Это очень мощный метод решения геометрических задач на плоскости и в пространстве. Если вы решили его использовать, то вот какие этапы предстоит пройти:
\begin{enumerate}
\item ввести систему координат;
\item определить координаты точек и уравнения прямых (а то и плоскостей), важных для ответа на вопрос задачи. Если координаты некоторых точек неизвестны, нужно <<обозвать>> их буквами, т.е. ввести переменные;
\item получить уравнение (реже -- выражение) или систему уравнений с введенными переменными, и найти их значения.
\end{enumerate} 

В большинстве случаев проще работать с прямоугольной системой координат, в которой единичные отрезки на разных осях равны между собой (т.н. \textbf{ортонормированная} система координат).

Вместе с преимуществами метод координат имеет и недостатки: во-первых, малейшая ошибка при определении координат точек или при составлении уравнений рушит все дальнейшие усилия -- значит, при решении этим методом нужна хорошая концентрация и самопроверка. Во-вторых, решение ряда задач этим методом нерационально: на него уходит много времени, а объем вычислений иногда может неприятно удивить. Поэтому, если вы представляете, как решить задачу другим методом, то, возможно, стоит его попробовать, прежде чем использовать метод координат.\\
Как же понять, имеет ли смысл использовать метод координат? Вот несколько признаков, которые могут на это указать:
\begin{itemize}
\item в условии задачи есть фигура с прямым углом. Да, вы уже поняли: угол между координатными осями тоже прямой, поэтому легко ввести систему координат и определить координаты как минимум трех точек -- вершины прямого угла и двух <<соседних>> с ней вершин. Например, если дан прямоугольник (или прямоугольный параллелепипед), все стороны которого известны;
\item требуется найти экстремальное (т.е. наименьшее или наибольшее) значение величины, зависящей от одного параметра. Вводим систему координат и переменную, задающую параметр, затем строим выражения для функции, задающей требуемую величину. Остается определить экстремальное значение этой функции.
\end{itemize}
Не факт, что именно метод координат -- наиболее подходящий в конкретном случае, но с опытом решения множества задач вы научитесь определять, когда его следует применить.

Прямая на плоскости с введенной системой координат однозначно задается точкой и направляющим вектором. Действительно, если мы знаем вектор $\overrightarrow{a}(p,q)$, параллельный нашей прямой, и точку $M(x_0,y_0)$, через которую она проходит, то про любую другую точку $A$ можем точно сказать, лежит ли она на нашей прямой: если вектор $\overrightarrow{MA}$ коллинеарен $\overrightarrow{a}$, то точка $A$ лежит на прямой, в противном случае -- не лежит. Ясно, что вектор $\overrightarrow{a}$ определяется с точностью до коллинеарности (т.е. его можно заменить на коллинеарный вектор, и ничего не изменится), а точка $M$ может быть заменена на произвольную точку $A(x,y)$ с условием $\overrightarrow{MA}\parallel\overrightarrow{a}$. Для выполнения этого условия необходимо $q(x-x_0)=p(y-y_0)$, откуда получаем $py-qx-py_0+qx_0=0$. Это почти уравнение прямой, но давайте ради красоты переименуем некоторые параметры: $$ax+by+c=0$$
Полученное уравнение называют \textbf{общим уравнением прямой}, иногда добавляя, что $a^2+b^2\ne0$, чтобы подчеркнуть, что $a$ и $b$ не могут одновременно равняться нулю. Если $b\ne0$, то можно переписать это уравнение в виде $y=-\frac ab x -\frac cb$ -- \textbf{уравнение прямой с угловым коэффициентом}, которое чаще записывают так: $y=kx+b$.

С помощью уравнения с угловым коэффициентом можно задать любую прямую, если она не перпендикулярна оси абсцисс -- в этом случае в общем уравнении будет $b=0$, и делить на $b$ нельзя.

Заметим, что в общем уравнении прямой параметры $a$ и $b$ выражаются через координаты направляющего вектора следующим образом: $a=-q$, $b=p$. Если мы перемножим вектор $\overrightarrow{n}=(a,b)$ на вектор $\overrightarrow{a}$, то получим $(a,b)\cdot(p,q)=(-q,p)\cdot(p,q)=-qp+pq=0$ -- это означает, что $\overrightarrow{n}\perp\overrightarrow{a}$. Вектор $\overrightarrow{n}(a,b)$ называется \textbf{нормальным вектором} прямой, задающейся уравнением $ax+by+c=0$, и тоже определяется с точностью до коллинеарности.

Если две прямые параллельны, то их нормальные векторы коллинеарны и направляющие векторы тоже коллинеарны, а угловые коэффициенты равны (либо обе прямые перпендикулярны оси абсцисс). \\
\textbf{Уравнение прямой в отрезках}: $\frac xa +\frac yb=1$ -- так можно задать прямую, не проходящую через начало координат и не параллельную ни одной из координатных осей, если она проходит через точки $(a,0)$ и $(0,b)$.\\
\textbf{Уравнение прямой, проходящей через две заданные точки}: $\frac{x-x_1}{x_2-x_1}=\frac{y-y_1}{y_2-y_1}$ -- упростив это равенство, можно получить уравнение прямой, проходящей через точки $(x_1,y_1)$ и $(x_2,y_2)$. Если $x_1=x_2$, то уравнение запишется так: $x=x_1$. Аналогично, если $y_1=y_2$, то получим $y=y_1$.\\

Как прямая задается точкой и вектором, так и плоскость в пространстве может быть задана точкой и парой неколлинеарных векторов, либо тремя точками, не лежащими на общей прямой. Существуют разные способы записать уравнение плоскости, среди них есть
\begin{itemize}
\item \textbf{общее уравнение плоскости}: $ax+by+cz+d=0$, где $a^2+b^2+c^2\ne0$. Здесь $(a,b,c)$ -- координаты нормального вектора;
\item \textbf{уравнение плоскости в отрезках}: $\frac xa+\frac yb+\frac zc=1$, где $(a,0,0)$, $(0,b,0)$, $(0,0,c)$ -- точки пересечения плоскости с координатными осями (если таковые существуют). \\
\end{itemize}

\ex Запишите уравнение прямой, проходящей через точки $(2,-2)$ и $(0,7)$.
\sol Используем уравнение прямой, проходящей через две заданные точки: $\frac{x-2}{0-2}=\frac{y-(-2)}{7-(-2)}$, откуда получим $\frac{x-2}{-2}=\frac{y+2}{9}\Rightarrow 9(x-2)=-2(y+2)\Rightarrow y=-4.5x+7$. Можно сделать проверку, подставив в полученное уравнение координаты исходных точек.

\ex Запишите уравнение прямой, проходящей через точку $(6,-3)$ перпендикулярно прямой, заданной уравнением $2y-5x+1=0$.
\sol Нормальный вектор искомой прямой должен быть перпендикулярен вектору $(2,-5)$ -- нормальному вектору прямой $2y-5x+1=0$. Один из таких векторов имеет координаты $(5,2)$ (можно убедиться, что $(5,2)\cdot(2,-5)=0$ -- значит, эти векторы перпендикулярны). Отсюда делаем вывод, что уравнение искомой прямой записывается как $5x+2y+c=0$ для некоторого $c$. Найдем его, подставив координаты точки $(6,-3)$: $5\cdot6+2\cdot(-3)+c=0\Rightarrow c=-24$. Ответ: $5x+2y-24=0$.

\ex Найдите расстояние от точки $P(5,1)$ до прямой $f$, заданной уравнением $3x+y+3=0$.
\sol Сначала запишем уравнение прямой $h$, проходящей через точку $P$ перпендикулярно прямой $f$: $x-3y-2=0$. Теперь найдем точку $K$ пересечения прямых $f$ и $h$, решив систему уравнений:
\begin{equation*} 
\begin{cases}
   3x+y+3=0\\
   x-3y-2=0
 \end{cases}
\end{equation*}
Получим $K(-0.7,-0.9)$. Осталось найти длину отрезка $KP$: $|KP|=\sqrt{(5-(-0.7))^2+(1-(-0.9))^2}=\sqrt{5.7^2+1.9^2}=1.9\sqrt{10}$.

\subsection*{Задачи}

\task Даны точки $A(1,4)$, $B(3,0)$, $C(-2,4)$. Запишите уравнения прямых, содержащих стороны треугольника $ABC$.

\task Запишите уравнения прямых, содержащих медианы треугольника $ABC$ из предыдущей задачи.

\task Найдите координаты точки пересечения прямых, заданных уравнениями $x+4y-6=0$ и $3x+4y+2=0$.

\task Прямые $a$ и $b$ проходят через точку $(4,-1)$, причем прямая $a$ параллельна прямой $c$, заданной уравнением $2x-y-5=0$, а прямая $b$ перпендикулярна $c$. Запишите уравнения прямых $a$ и $b$. 

\task При каких значениях параметра $p$ система уравнений 
\begin{equation*} 
\begin{cases}
   3x+(p^2-1)y-4=0\\
   px+8y-p=1
 \end{cases}
\end{equation*}
не имеет решений?

\taskk Вычислите площадь треугольника $ABC$ из задачи №1 двумя разными способами.

\taskk Докажите, что расстояние от точки $M(x_0,y_0)$ до прямой $l$, заданной уравнением $ax+by+c=0$, может быть вычислено по формуле $$d(M,l)=\frac{ax_0+by_0+c}{\sqrt{a^2+b^2}}$$

\taskk Докажите, что расстояние от точки $M(x_0,y_0,z_0)$ до плоскости $\theta$, заданной уравнением $ax+by+cz+d=0$, равно $$d(M,\theta)=\frac{ax_0+by_0+cz_0+d}{\sqrt{a^2+b^2+c^2}}$$

\taskk Докажите, что площадь треугольника с координатами вершин $(0,0)$, $(x_1,y_1)$, $(x_2,y_2)$ равна $\frac12 |x_1y_2-x_2y_1|$.

\taskk Докажите, что если координаты вершин треугольника рациональны, то координаты центра его описанной окружности тоже рациональны.




\subsection{Геометрия масс}
\setcounter{tasknum}{0}
\setcounter{exnum}{0}

Из курса физики мы знаем, что \textbf{центр тяжести} тела (фигуры, множества точек, ...) -- это точка, к которой приложена сила тяжести, действующая на тело. Точнее, \textsl{можно считать}, что сила тяжести приложена к этой точке, а на самом деле сила тяжести действует на любой объект в поле тяготения, обладающий массой, т.е. на каждый элемент тела, обладающий массой. Нам будет важно, что за центр масс можно <<подвесить>> тело в любом положении, и это положение не изменится без вмешательства извне.

Вообще говоря, центр тяжести и \textbf{центр масс} -- это разные понятия, но в случае равномерно распределенной массы эти точки совпадают. Это как раз наш случай: мы будем считать, что масса равномерно распределена по всем точкам наших фигур, а центр масс будем иногда называть \textbf{барицентром} фигуры.

Во-первых, у любой фигуры существует центр масс, и он единственен -- докажем это для системы из конечного числа точек.\\
$A_1,A_2,\ldots,A_n$ -- точки, $m_1,m_2,\ldots,m_n$ -- их массы. Выберем произвольные точки $P$ и $M$. Тогда $m_1\overrightarrow{MA_1}+\cdots+m_n\overrightarrow{MA_n}=(m_1+\cdots+m_n)\overrightarrow{MP}+m_1\overrightarrow{PA_1}+\cdots+m_n\overrightarrow{PA_n}$.\\
Точка $M$ является центром масс тогда и только тогда, когда $m_1\overrightarrow{MA_1}+\cdots+m_n\overrightarrow{MA_n}=0$ (условие равновесия системы точек), откуда $(m_1+\cdots+m_n)\overrightarrow{MP}+m_1\overrightarrow{PA_1}+\cdots+m_n\overrightarrow{PA_n}=0$ и $$\overrightarrow{PM}=\frac{m_1\overrightarrow{PA_1}+\cdots+m_n\overrightarrow{PA_n}}{m_1+\cdots+m_n}$$
Из последней формулы следует единственность центра масс. Опустим доказательство для бесконечного числа точек, поскольку оно требует предельного перехода, а понятия предела и сходимости ряда уже не входят в школьную программу.\\

Докажем (тоже для конечного числа точек), что если часть точек заменить одной точкой, которая расположена в их центре масс и которой приписана масса, равная сумме их масс, то центр масс системы точек останется прежним.\\
Пусть $M$ -- центр масс точек $A_1,\ldots,A_n,B_1,\ldots,B_m$ с массами $a_1,\ldots,a_n,b_1,\ldots,b_m$ соответственно, при этом $B$ -- центр масс системы точек $B_1,\ldots,B_m$. Тогда $a_1\overrightarrow{MA_1}+\cdots+a_n\overrightarrow{MA_n}+b_1\overrightarrow{MB_1}+\cdots+b_m\overrightarrow{MB_m}$, причем $b_1\overrightarrow{BB_1}+\cdots+b_m\overrightarrow{BB_m}=0$. Вычитая второе равенство из первого, получим $$a_1\overrightarrow{MA_1}+\cdots+a_n\overrightarrow{MA_n}+(b_1+\cdots+b_m)\overrightarrow{MB}=0,$$
т.е. точка $M$ является центром масс системы точек $A_1,\ldots,A_n,B$ с массами соответственно \\$a_1,\ldots,a_n,(b_1+\cdots+b_m)$, что и требовалось доказать.\\

\ex Доказать, что центр масс точек $A$ и $B$ с массами $a$ и $b$ делит отрезок $AB$ в отношении $b:a$, считая от точки $A$.
\sol Пусть $M$ -- центр масс точек $A$ и $B$. Тогда $a\overrightarrow{MA}+b\overrightarrow{MB}=0$ -- значит, векторы $\overrightarrow{MA}$ и $\overrightarrow{MB}$ коллинеарны и противоположно направлены, следовательно, точка $M$ лежит на отрезке $AB$. Далее получим $a|\overrightarrow{MA}|=b|\overrightarrow{MB}|$, откуда $\frac{MA}{MB}=\frac ba$, что и требовалось доказать.

\subsection*{Задачи}

\task Докажите, что медианы треугольника пересекаются в одной точке и делятся ею в отношении $2:1$, считая от вершины.

\task Назовем \textsl{медианой тетраэдра} отрезок, соединяющий вершину тетраэдра с точкой пересечения медиан противоположной грани. Докажите, что медианы тетраэдра пересекаются в одной точке, и найдите отношение, в котором эта точка делит их.

\task Докажите теорему Чевы при помощи группировки масс:\\
\textbf{Теорема Чевы}. Точки $A_1,B_1,C_1$ соответственно лежат на сторонах $BC,CA,AB$ треугольника $ABC$ так, что отрезки $AA_1,BB_1,CC_1$ пересекаются в одной точке. Докажите, что $\frac{AC_1}{C_1B}\cdot\frac{BA_1}{A_1C}\cdot\frac{CB_1}{B_1A}=1$. 

\task Докажите, что если у фигуры есть центр симметрии, то он совпадает с центром масс фигуры.
\textsl{\comment Из этого следует, что если у фигуры не может быть более одного центра симметрии.}

\task Докажите, что если у фигуры есть ось симметрии, то центр масс фигуры расположен на этой оси.
\textsl{\comment Из этого следует, что если у фигуры есть несколько осей симметрии, то они пересекаются в одной точке.}

\taskk Дан треугольник $ABC$ и точка $P$ внутри него. Докажите, что можно подобрать массы $a,b,c$ точек $A,B,C$ так, что точка $P$ будет центром масс треугольника $ABC$.
\textsl{\comment Тройка чисел $a,b,c$ называется \textbf{барицентрическими координатами} точки $P$. Если $a+b+c=1$, то $(a:b:c)$ называется \textbf{абсолютными барицентрическими координатами} точки $P$ относительно треугольника $ABC$.}

\taskk Докажите, что абсолютные барицентрические координаты точки определены однозначно.


%\subsection{Задачи для практики}
%\setcounter{tasknum}{0}
%\setcounter{exnum}{0}
%
%\task
%
%\task
%
%\task
%
%\task
%
%\task
%
%\task
%
%\task
%
%\task
%
%\task
%
%\task
%
%\task
%
%\task
%
%\task
%
%\task
%
%\task
%
%\task



\newpage
\part{Стереометрия}

\setcounter{section}{0}

\section{Многогранники}

\subsection{Прямые и плоскости в пространстве. Проекции и сечения}
\setcounter{tasknum}{0}
\setcounter{exnum}{0}

\subsection*{Задачи}

\task Высота прямоугольного треугольника $ABC$, опущенная на гипотенузу, равна $9.6$. Из вершины $C$ прямого угла восставлен к плоскости треугольника $ABC$ перпендикуляр $CM=28$. Найдите расстояние от точки $M$ до гипотенузы $AB$.

\task Основание пирамиды $SABCD$ -- четырёхугольник $ABCD$. Постройте прямую пересечения плоскостей $ABS$ и $CDS$.

\task Дан куб $ABCDA_1B_1C_1D_1$ с ребром $a$. Найдите расстояние между прямыми $AA_1$ и $BD_1$ и постройте их общий перпендикуляр.

\task Существует ли четырёхугольная пирамида, у которой две противоположные боковые грани перпендикулярны плоскости основания?

\task Верно ли, что в пространстве углы с соответственно перпендикулярными сторонами равны или составляют в сумме $180^\circ$?

\taskk Боковая грань правильной четырёхугольной пирамиды образует с плоскостью основания угол $45^\circ$. Найдите угол между апофемой пирамиды и плоскостью соседней грани.

\taskk В основании четырёхугольной пирамиды $SABCD$ лежит параллелограмм $ABCD$. Известно, что плоскости треугольников $ASC$ и $BSD$ перпендикулярны друг другу. Найдите площадь грани $ASD$, если площади граней $ASB , BSC$ и $CSD$ равны соответственно 5, 6 и 7.

\taskk На скрещивающихся прямых $l$ и $m$ взяты отрезки $AB$ и $CD$ соответственно. Докажите, что объём пирамиды $ABCD$ не зависит от положения отрезков $AB$ и $CD$ на этих прямых. Найдите этот объём, если $AB = a$, $CD = b$, а угол и расстояние между прямыми $l$ и $m$ равны соответственно $\alpha$ и $c$.

\taskkk Основанием пирамиды $ABCEH$ служит выпуклый четырехугольник $ABCE$, который диагональю $BE$ делится на два равновеликих треугольника. Длина ребра $AB$ равна 1, длины ребер $BC$ и $CE$ равны между собой. Сумма длин ребер $AH$ и $EH$ равна $ \sqrt{2}$. Объем пирамиды равен $\frac16$. Найдите радиус шара, имеющего наибольший объем среди всех шаров, помещающихся в пирамиде $ABCEH$.




\subsection{Многогранные углы}
\setcounter{tasknum}{0}
\setcounter{exnum}{0}

\subsection*{Задачи}

\task Докажите, что выпуклый четырёхгранный угол можно пересечь плоскостью так, чтобы в сечении получился параллелограмм.

\task Верно ли, что в сечении любого трёхгранного угла плоскостью можно получит правильный треугольник?

\task На какое наименьшее число непересекающихся трёхгранных углов можно разбить пространство?

\task Найдите двугранные углы трёхгранного угла, плоские углы которого равны $90^\circ$, $90^\circ$ и $\alpha$. 

\task Все плоские углы трёхгранного угла прямые. Докажите, что любое его сечение, не проходящее через вершину, есть остроугольный треугольник.

\taskk Сколько существует различных пирамид, все рёбра которых равны 1?

\taskk Докажите, что сумма угловых величин всех двугранных углов тетраэдра больше $360^\circ$.

\taskkk \textsl{Московская математическая олимпиада, 1975 г.} Можно ли разместить в пространстве четыре свинцовых шара и точечный источник света так, чтобы каждый исходящий из источника света луч пересекал хотя бы один из шаров?

\taskkk \textsl{Московская математическая олимпиада, 1989 г.} На рёбрах произвольного тетраэдра выбрано по точке. Через каждую тройку точек, лежащих на рёбрах с общей вершиной, проведена плоскость. Докажите, что если три из четырёх проведённых плоскостей касаются вписанного в тетраэдр шара, то и четвёртая плоскость также его касается.


\subsection{Классификация многогранников. Платоновы тела}
\setcounter{tasknum}{0}
\setcounter{exnum}{0}

\subsection*{Задачи}

\task Существует ли выпуклый многогранник, имеющий 12 рёбер, которые соответственно равны и параллельны 12 диагоналям граней куба?

\task Найдите объём правильного октаэдра (правильного восьмигранника), ребро которого равно $a$.

\task Можно ли расположить на плоскости\\
  а) 4 точки так, чтобы каждая из них была соединена отрезками с тремя другими (без пересечений)?\\
  б) 6 точек и соединить их непересекающимися отрезками так, чтобы из каждой точки выходило ровно 4 отрезка?

\task \textsl{Олимпиада по геометрии имени И.Ф.Шарыгина, 2010 г.} Среди вершин двух неравных икосаэдров можно выбрать шесть, являющихся вершинами правильного октаэдра. Найдите отношение рёбер икосаэдров.

\task Можно ли вписать октаэдр в куб так, чтобы вершины октаэдра находились на рёбрах куба?

\taskk Можно ли вписать октаэдр в додекаэдр так, чтобы каждая вершина октаэдра была вершиной додекаэдра?

\taskk \textsl{Турнир городов, 2005 г.} Муравей ползает по замкнутому маршруту по рёбрам додекаэдра, нигде не разворачиваясь назад. Маршрут проходит ровно два раза по каждому ребру. Докажите, что некоторое ребро муравей оба раза проходит в одном и том же направлении.

\taskkk \textsl{Московская математическая олимпиада, 1997 г.} Можно ли разбить правильный тетраэдр с ребром 1 на правильные тетраэдры и октаэдры, длины ребер каждого из которых меньше $\frac{1}{100}$?



\subsection{Развертки и сечения многогранников}
\setcounter{tasknum}{0}
\setcounter{exnum}{0}

\subsection*{Задачи}

\task Постройте сечение куба, представляющее собой правильный шестиугольник.

\task Может ли сечением куба оказаться правильный пятиугольник?

\task Основание пирамиды $PABCD$ -- параллелограмм $ABCD$. На рёбрах $AB$ и $PC$ взяты соответственно точки $K$ и $M$, причём $AK:KB = CM:MP = 1:2$. В каком отношении плоскость, проходящая через точки $K$ и $M$ параллельно прямой $BD$, делит объём пирамиды $PABCD$?

\task \textsl{Турнир городов, 1988 г.} Можно ли нарисовать на поверхности кубика Рубика такой замкнутый путь, который проходит через каждый квадратик ровно один раз (через вершины квадратиков путь не проходит)?

\task \textsl{Московская математическая олимпиада, 1959 г.} Доказать, что не более одной вершины тетраэдра обладает тем свойством, что сумма любых двух плоских углов при этой вершине больше $180^\circ$.

\taskk Плоскости диагональных сечений пирамиды, основанием которой является параллелограмм, взаимно перпендикулярны. Докажите, что суммы квадратов площадей противоположных боковых граней равны между собой.

\taskkk \textsl{Московская олимпиада по геометрии, 2006 г.} Oснованием пирамиды служит выпуклый четырехугольник. Oбязательно ли существует сечение этой пирамиды, не пересекающее основание и являющееся вписанным четырехугольником?

\taskkk \textsl{Турнир городов, 2019 г.} Может ли в сечении какого-то тетраэдра двумя разными плоскостями получиться два квадрата: один -- со стороной, не большей 1, а другой -- со стороной, не меньшей 100?



\subsection{Тетраэдры}
\setcounter{tasknum}{0}
\setcounter{exnum}{0}

\subsection*{Задачи}

\task Найдите объем правильного тетраэдра с ребром $a$. 

\task Высота треугольной пирамиды проходит через точку пересечения высот треугольника основания. Докажите, что противоположные рёбра пирамиды попарно перпендикулярны.

\task Существует ли тетраэдр, высоты которого равны 1, 2, 3 и 6?

\task Дан тетраэдр, у которого периметры всех граней равны между собой. Докажите, что сами грани равны между собой.

\taskk \textsl{Турнир городов, 2008 г.} Внутри некоторого тетраэдра взяли произвольную точку $X$. Через каждую вершину тетраэдра провели прямую, параллельную отрезку, соединяющему $X$ с точкой пересечения медиан противоположной грани. Докажите, что четыре полученные прямые пересекаются в одной точке.

\taskk \textsl{Олимпиада по геометрии имени И.Ф.Шарыгина, 2014 г.} Докажите, что для любого тетраэдра его самый маленький двугранный угол (из шести) не больше чем двугранный угол правильного тетраэдра.

\taskk \textsl{Олимпиада по геометрии имени И.Ф.Шарыгина, 2005 г.} К граням тетраэдра восстановлены перпендикуляры в их точках пересечения медиан. Докажите, что проекции трёх перпендикуляров на четвёртую грань пересекаются в одной точке.

\taskk \textsl{Всероссийская олимпиада по математике, 2001 г.} Докажите, что если у тетраэдра два отрезка, идущие из концов некоторого ребра в центры вписанных окружностей противолежащих граней, пересекаются, то отрезки, выпущенные из концов скрещивающегося с ним ребра в центры вписанных окружностей двух других граней, также пересекаются.

\taskkk \textsl{Всероссийская олимпиада по математике, 2008 г.} Каждую грань тетраэдра можно поместить в круг радиуса 1. Докажите, что весь тетраэдр можно поместить в шар радиуса $\frac{3}{2\sqrt2}$.



\section{Цилиндры, конусы, сферы}

\subsection{Сфера и шар}
\setcounter{tasknum}{0}
\setcounter{exnum}{0}

\subsection*{Задачи}

\task В пространстве дана плоскость П и точки A и B по одну сторону от П (AB не параллельно П). Рассматриваются сферы, проходящие через точки A и B, касающиеся плоскости П. Докажите, что точки касания этих сфер и плоскости П лежат на одной окружности.

\task Внутренняя точка $A$ шара радиуса $r$ соединена с поверхностью шара тремя отрезками прямых, имеющими длину $l$ и проведёнными под углом $\alpha$ друг к другу. Найдите расстояние точки $A$ от центра шара.

\taskk \textsl{Всероссийская олимпиада по математике, 2014 г.} Есть полусферическая ваза, закрытая плоской крышкой. В вазе лежат четыре одинаковых апельсина, касаясь вазы, и один грейпфрут, касающийся всех четырёх апельсинов. Верно ли, что все четыре точки касания грейпфрута с апельсинами обязательно лежат в одной плоскости? (Все фрукты являются шарами.)

\taskk \textsl{Всероссийская олимпиада по математике, 2015 г.} В пространстве расположены 2016 сфер, никакие две из них не совпадают. Некоторые из сфер -- красного цвета, а остальные -- зелёного. Каждую точку касания красной и зелёной сферы покрасили в синий цвет. Найдите наибольшее возможное количество синих точек.

\taskk Три параллельные прямые касаются в точках $A , B$ и $C$ сферы радиуса 4 с центром в точке $O$. Найдите угол $BAC$, если известно, что площадь треугольника $OBC$ равна 4, а площадь треугольника $ABC$ больше 16.

\taskk Три шара радиуса $r$ лежат на нижнем основании правильной треугольной призмы, причём каждый из них касается двух других шаров и двух боковых граней призмы. На этих шарах лежит четвёртый шар, который касается всех боковых граней и верхнего основания призмы. Найдите высоту призмы.

\taskk На сфере, радиус которой равен 2, расположены три окружности радиуса 1, каждая из которых касается двух других. Найдите радиус окружности меньшей, чем данная, которая также расположена на данной сфере и касается каждой из данных окружностей.

\taskk Можно ли точку в пространстве закрыть четырьмя шарами (т.е. построить четыре шара так, чтобы любой луч, выходящий из заданной точки, пересекал хотя бы один шар)? 

\taskkk Четыре сферы радиуса 1 попарно касаются. Найдите радиус сферы, касающейся всех четырёх сфер. 

\taskkk \textsl{Московская математическая олимпиада, 1963 г.} Доказать, что на сфере нельзя так расположить три дуги больших окружностей в $300^\circ$ каждая, чтобы никакие две из них не имели ни общих точек, ни общих концов. Большая окружность -- это окружность, полученная в сечении сферы плоскостью, проходящей через ее центр.


\subsection{Конус и цилиндр}
\setcounter{tasknum}{0}
\setcounter{exnum}{0}

\subsection*{Задачи}

\task Радиус основания цилиндра равен $r$. Плоскость пересекает боковую поверхность цилиндра, не пересекает его оснований и образует угол $\alpha$ с плоскостью основания. Найдите площадь сечения цилиндра этой плоскостью.

\task Дан прямой круговой конус и точка $O$. Найти геометрическое место вершин конусов, равных данному, с осями, параллельными оси данного конуса, и содержащих внутри данную точку $O$.

\task Три шара одинакового радиуса попарно касаются друг друга и некоторой плоскости. Основание конуса расположено в этой плоскости. Все три сферы касаются боковой поверхности конуса внешним образом. Найдите угол при вершине осевого сечения конуса, если высота конуса равна диаметру шара.

\taskk Высота цилиндра равна $3r$. Внутри цилиндра расположены три сферы радиуса $r$, причём каждая сфера касается двух других и боковой поверхности цилиндра. Две сферы касаются нижнего основания цилиндра, а третья сфера -- верхнего основания. Найдите радиус основания цилиндра.

\taskk В конусе расположены два одинаковых шара радиуса $r$, касающиеся основания конуса в точках, симметричных относительно центра основания. Каждый из шаров касается боковой поверхности конуса и другого шара. Найдите угол между образующей конуса и основанием, при которой объем конуса наименьший.

\taskk \textsl{Окружная олимпиада (Москва), 2008 г.} Найдите угол при вершине осевого сечения прямого кругового конуса, если известно, что существуют три образующие боковой поверхности конуса, попарно перпендикулярные друг другу.

\taskk \textsl{Олимпиада по геометрии имени И.Ф.Шарыгина, 2007 г.} На плоскости лежат три трубы (круговые цилиндры одного размера в обхвате 4 м). Две из них лежат параллельно и, касаясь друг друга по общей образующей, образуют над плоскостью тоннель. Третья, перпендикулярная к первым двум, вырезает в тоннеле камеру. Найдите площадь границы этой камеры.

\taskkk Два равных конуса с общей вершиной касаются друг друга и некоторой плоскости $\alpha$. Пусть $l$ -- прямая, по которой пересекаются плоскости оснований конусов. Найдите угол между прямой $l$ и плоскостью $\alpha$, если высота каждого конуса равна 2, а радиус основания равен 1.

\taskkk На плоскости лежат три равных конуса с общей вершиной. Каждый из них касается двух рядом лежащих. Найдите угол при вершине каждого конуса.




%\subsection{Конические сечения. Шары Данделена}
%\setcounter{tasknum}{0}
%\setcounter{exnum}{0}
%
%\subsection*{Задачи}
%
%\task
%
%\task
%
%\task
%
%\task
%
%\task
%
%\task
%
%\task
%
%\task




\newpage
\part{Элементы теории чисел}
\setcounter{section}{0}

В этой главе мы познакомимся с \textbf{теорией чисел} -- разделом математики, изучающим \textbf{целые числа} и их свойства. Опуская формальное определение натурального числа и считая натуральными числа $1,2,3,\ldots$ (часто для этого множества используется обозначение $\mathbb N$), дадим определение множества целых чисел (обозначается как $\mathbb Z$): это множество, каждый элемент которого -- либо натуральное число, либо число, противоположное натуральному, либо $0$. \\Итак, $\mathbb N=\{1,2,3,\ldots\}$; $\mathbb Z=\{\ldots,-2,-1,0,1,2,\ldots\}$.

\section{Делимость и остатки}


\subsection{Делимость, признаки делимости. Простые числа}
\setcounter{tasknum}{0}
\setcounter{exnum}{0}

Будем говорить, что целое число $a$ \textbf{делится} на целое число $b\ne0$, если существует такое целое $k$, что $a=bk$ (разумеется, $0$ делится на любое целое число, кроме самого себя). Тогда для каждого целого числа можно выделить конечное подмножество целых чисел, на которые оно делится -- множество его делителей. Например, множество делителей числа $12$ выглядит так: $\{-12,-6,-4,-3,-2,-1,1,2,3,4,6,12\}$.

В этом разделе нас будут интересовать в первую очередь натуральные числа. Если для натурального $a$ множество его \textsl{натуральных} делителей состоит всего из двух чисел -- $\{1,a\}$, -- будем говорить, что число $a$ -- \textbf{простое}. Если у натурального числа больше двух натуральных делителей, то будем называть его \textbf{составным}. Заметим, что число $1$ не является ни простым, ни составным: у него всего один делитель.\\

Но что, если $a$ не делится на $b\ne0$? Тогда можно говорить о ненулевом остатке от деления $a$ на $b$.\\
\textbf{Остаток от деления} целого $a$ на натуральное число $b$ -- это такое наименьшее целое неотрицательное $r$, что существует целое $k$, при котором $a=bk+r$. Обратите внимание: из определения следует, что $0\le r<b$ . Так, например, $17$ дает остаток $3$ при делении на $7$, поскольку $3$ -- это наименьшее значение неотрицательного целого $r$, при котором $17-r$ делится на $7$. \\

Наверняка читатель заметил, что, определяя делимость $a$ на $b$, мы говорили лишь о существовании частного, но не о его значении. И правда, во многих задачах теории чисел нас будет интересовать лишь возможность деления, но не его результат. Итак, что надо выяснить, делится ли $a$ на $b$, не выполняя деления -- здесь нам пригодятся признаки делимости. Перечислим некоторые из них, пригодные к записи числа в десятичной системе счисления (подробнее о системах счисления будет рассказано в параграфе 1.3):
\begin{itemize}
\item \textbf{на 2}: натуральное число делится на 2 тогда и только тогда, когда последняя (справа) цифра в его записи делится на 2;
\item \textbf{на 3}: натуральное число делится на 3 тогда и только тогда, когда сумма его цифр делится на 3;
\item \textbf{на 5}: натуральное число делится на 5 тогда и только тогда, когда последняя (справа) цифра в его записи делится на 5;
\item \textbf{на 9}: натуральное число делится на 9 тогда и только тогда, когда сумма его цифр делится на 9;
\item \textbf{на 10}: натуральное число делится на 10 тогда и только тогда, когда его последняя цифра -- 0;
\item \textbf{на 100}: натуральное число делится на 100 тогда и только тогда, когда последние две его цифры -- нули.
\end{itemize}

\ex Докажите, что если $a$ $b$ делятся на $c$, то $a+b$ тоже делится на $c$.\\
\textsl{Доказательство.} По определению делимости имеем $a=cx,b=cy$ для некоторых целых $x,y$. Таким образом, $a+b=cx+cy=c(x+y)$, т.е. $a+b$ представимо в виде $ck$ для целого $k=x+y$, что по определению означает, что $a+b$ делится на $c$.\\

\ex Доказать признак делимости на 4: натуральное число $n$ кратно 4 тогда и только тогда, когда число $m$, образованное последними двумя цифрами числа $n$ в их исходном порядке, кратно 4. (Так, например, число 14172 кратно 4, поскольку 72 кратно 4)\\
\textsl{Доказательство.} Представим число $n$ в виде $n=(n-m)+m$. Тогда число $n-m$ в десятичной системе счисления оканчивается на <<00>>, т.е. делится на 100, а значит, делится на 4. Тогда, согласно примеру 1, число $(n-m)+m$ делится на 4 тогда и только тогда, когда $m$ делится на 4, что и требовалось доказать.

\ex Известно, что $14x + 13y$ делится на 11 при некоторых целых $x$ и $y$. Докажите, что $19x + 9y$ также делится на 11 при тех же $x$ и $y$.\\
\textsl{Доказательство.} Заметим, что $19x + 9y = 11(3x + 2y) + (-1)\cdot(14x + 13y)$. Первое слагаемое кратно 11 по определению, а второе -- по условию. Значит, их сумма тоже кратна 11, что и требовалось доказать.


\subsection*{Задачи}

\task Докажите, что если $a,b$ кратны $c$, то $a-b$ кратно $c$.

\task Докажите, что если $a$ кратно $c$, то $ab$ кратно $c$ для любого целого $b$.

\task Сумма трёх различных наименьших делителей некоторого числа $A$ равна 8. На сколько нулей может оканчиваться число $A$?

\task Пусть $a,b$ -- целые числа. Докажите, что если $a^2+9ab+b^2$ делится на 11, то и $a^2-b^2$ делится на 11.

\task Сумма двух натуральных чисел равна 201. Докажите, что произведение этих чисел не может делиться на 201.

\task Докажите, что если произведение целых $a,b$ делится на простое число $p$, то либо $a$, либо $b$ (либо они оба) кратны $p$.

\taskk Есть неограниченное число монет по 6, 9 и 20 рублей. Какую наибольшую сумму в целое число рублей нельзя набрать, используя только эти монеты?

\taskk \textsl{Турнир им. Ломоносова, 2019 г.} \\Пусть $a,b,c,d,n$ -- натуральные числа. Докажите, что если числа $(a-b)(c-d)$ и $(a-c)(b-d)$ делятся на $n$, то и число $(a-d)(b-c)$ делится на $n$.

\taskk \textsl{Окружная олимпиада (Москва), 2016 г.} \\Сумма двух целых чисел равна $S$. Маша умножила левое число на целое число $a$, правое -- на целое число $b$, сложила эти произведения и обнаружила, что полученная сумма делится на $S$. Алёша, наоборот, левое число умножил на $b$, а правое -- на $a$. Докажите, что и у него аналогичная сумма разделится на $S$.

\taskk Докажите, что ни при каком натуральном $m$ число $1998^m-1$ не делится на $1000^m-1$.



\subsection{Основная теорема арифметики}
\setcounter{tasknum}{0}
\setcounter{exnum}{0}

Исследуя какой-либо объект, мы рассматриваем его как совокупность составляющих его элементов. Поэтому вполне естественно ожидать, что натуральные числа тоже представимы в виде некоторого сочетания элементов, а в контексте делимости не менее естественно ожидать, что упомянутые элементы будут множителями. О таком представлении натурального числа говорит\\

\textbf{Основная теорема арифметики.} Каждое натуральное число, кроме $1$, представимо в виде произведения простых чисел, причем такое представление единственно с точностью до перестановки множителей.\\
(Упомянутое представление натурального числа называется \textbf{каноническим}, и его легко подобрать для небольших чисел -- например, $84=2\cdot2\cdot3\cdot7$.) 

\textsl{Доказательство.} Воспользуемся методом математической индукции. Очевидно, для числа $2$ утверждение теоремы верно. Докажем существование разложения числа $n$ на простые множители, если предположить, что аналогичное уже доказано для всех натуральных чисел, меньших $n$. Если $n$ -- простое, то существование доказано (это следует из определения простого числа). Если $n$ -- составное, то оно может быть представлено в виде произведения $a\cdot b$, причем числа $a,b$ больше $1$, но меньше $n$ -- существование их разложения следует из индукционной гипотезы.\\

Теперь докажем единственность разложения. Для простого числа единственность очевидна. Для составного числа идея для доказательства заключается в использовании метода <<от противного>>: предположим, что число $n$ имеет два различных разложения. Рассмотрим простые числа $p_0$ и $q_0$, являющиеся наименьшими в первом и втором из этих разложений соответственно, и воспользуемся леммой:\\

\textbf{Лемма.} Если разложение числа $n$ на простые множители единственно, то каждый его простой делитель должен входить в это разложение.

\textsl{Докажите эту лемму самостоятельно!}\\

Далее рассматривается число $n-p_0q_0$, которое, в свою очередь, является натуральным и меньшим $n$. Из предположения индукции и вышеуказанной леммы следует, что $p_0q_0$ является делителем этого числа, а значит, первое разложение на множители делится на $q_0$. Никакое простое число не может встретиться в обоих разложениях сразу, так как иначе на него можно было бы сократить и получить различные разложения на простые множители числа, меньшего $n$, что неверно по предположению индукции. Теорема доказана.

\ex Найдите наименьшее натуральное $n$, при котором $2022!$ не делится на $662^n$. 
\sol $662 = 2\cdot 331$, причем 331 -- простое число. $2022!=1\cdot2\cdot3\cdot\cdots2021\cdot2022$, причем среди этих множителей кратны 331 только 331, 662, 993, 1324, 1655, 1986 -- всего 6 чисел, и ни одно из них не кратно $331^2$. Значит, в разложение $2022!$ на простые множители число 331 входит в степени 6, при этом число 2 входит в это разложение в степени, большей 6 (поскольку $2022!$ кратно $2^7=128$) -- значит, $2022!$ делится на $662^6$, но не кратно $662^7$.\\
Ответ: 7.

\ex Найдите наименьшее натуральное число, половина которого -- квадрат, треть -- куб, а пятая часть -- пятая степень.
\sol Пусть $n$ -- искомое число. Тогда $n$ кратно 2, 3, 5 и, согласно основной теореме арифметики, $n=2^a3^b5^cd$, где $a,b,c,d$ -- натуральные числа, причем $d$ не кратно ни 2, ни 3, ни 5. При этом $2^{a-1}3^b5^cd$ -- квадрат, т.е. $a-1,b,c$ -- четные, и $d$ -- квадрат. Поскольку $\frac n3$ -- куб, имеем $a,b-1,c$ кратны 3, и $d$ -- куб. Аналогично, $a,b,c-1$ кратны 5 и $d$ -- пятая степень натурального числа.\\
Поскольку требуется найти наименьшее натуральное число, удовлетворяющее описанным условиям, полагаем $d=1$ и находим наименьшие натуральные $a,b,c$: так, наименьшее $a$, являющееся нечетным, но кратное 3 и 5 -- это 15; наименьшее $b$, кратное 2 и 5 и дающее остаток 1 при делении на 3 -- это 10; наименьшее $c$, кратное 2 и 3 и дающее остаток 1 при делении на 5 -- это 6.\\
Ответ: $2^{15}\cdot3^{10}\cdot5^6$

\subsection*{Задача}

\task Разложите 111111 на простые множители.

\task Произведение двух натуральных чисел, не кратных 10, равно $10^n$. Найдите сумму этих чисел.

\task Произведение числа на сумму его цифр равно 2008. Найдите это число.

\task \textsl{Московская математическая регата, 2013 г.} \\При каких натуральных $n$ число $n^2-1$ является степенью простого числа?

\task \textsl{Московская математическая регата, 2013 г.} \\Перемножили несколько натуральных чисел и получили 224, причём самое маленькое число было ровно вдвое меньше самого большого. Сколько чисел перемножили?

\task Даны натуральные числа $a$ и $b$, причём $a<1000$. Докажите, что если $a^{21}$ делится на $b^{10}$, то $a^2$ делится на $b$.

\taskk \textsl{Турнир городов, 2012 г.} \\Пусть $C(n)$ -- количество различных простых делителей числа n. (Например,  $C(10)=2$, $C(11)=1$, $C(12)=2$.)
Конечно или бесконечно число таких пар натуральных чисел $(a, b)$, что $a\ne b$ и $C(a+b)=C(a)+C(b)$?

\taskk Докажите, что $n!$ не делится на $2^n$ ни при каком натуральном $n$.

\taskkk \textbf{Формула Лежандра.} Пусть $n!=p_1^{k_1}p_2^{k_2}\cdots p_m^{k_m}$ ($p_i$ -- попарно различные простые числа). Докажите, что $k_i=[\frac{n}{p_i}]+[\frac{n}{p_i^2}]+[\frac{n}{p_i^3}]+\cdots$.\\
Квадратными скобками обозначена целая часть числа, т.е. наибольшее целое число, не превосходящее данного. Например, $[12.8]=12$, $[\pi]=3$.

\taskkk Докажите, что для любых неотрицательных целых $m,n$ число $\frac{(2m)!\cdot (2n)!}{m!n!(m+n)!}$ является целым.



\subsection{Представление числа. Позиционные системы счисления}
\setcounter{tasknum}{0}
\setcounter{exnum}{0}

Для записи чисел всегда использовались те или иные значки. Но давать каждому числу свой значок стало нерационально: чем больше чисел требуется при вычислениях, тем больше нужно значков, а чем больше значков, тем труднее ими оперировать. Значит, множество значков нужно ограничить, но как тогда записать любое из огромного количества чисел? Возникла идея использования последовательностей значков, причем таких, чтобы с их помощью записывать достаточно большие числа, используя сравнительно небольшое количество значков.\\

Итак, значки -- \textsl{цифры} -- используются как алфавит для записи чисел, и в разные времена в разных частях света и в разных культурах использовали самые разные алфавиты: когда-то числа записывали палочками (одна палочка -- $1$, две палочки -- $2$), затем стали использовать буквы (например, с тильдой над ними) или специальные символы. Эти способы записи объединены общим принципом: одна и та же буква обозначает одно и то же независимо от своего положения в записи числа -- такие \textsl{системы счисления} называют непозиционными. В настоящее же время общепринятыми являются \textbf{позиционные} системы счисления, т.е. такие, в которых одна и та же цифра играет разную роль, будучи использованной в том или другом месте записи числа. Например, в привычной нам десятичной системе записи $330$ и $303$ обозначают разные числа несмотря на то, что используют один и тот же набор цифр.\\

Разберемся в позиционной записи числа подробнее на примере записи $1047$ в десятичной системе счисления: $$1047=1\cdot1000+0\cdot100+4\cdot10+7\cdot1=1\cdot10^3+0\cdot10^2+4\cdot10^1+7\cdot10^0$$
Вообще, запись $x_m=\overline{a_na_{n-1}\ldots a_2a_1a_0}$ (здесь целые $a_0,a_1,\ldots,a_n$ -- цифры: каждая из них не меньше нуля и строго меньше натурального $m>1$) говорит нам: число $x$ в $m$-ичной системе счисления записывается цифрами $a_n,a_{n-1},\ldots ,a_2,a_1,a_0$ и равно $$x=a_n\cdot m^n+a_{n-1}\cdot m^{n-1}+\cdots+a_2m^2+a_1m+a_0$$

Привычная нам десятичная система счисления -- лишь одна из множества позиционных систем. Читатель наверняка слышал о двоичной ($m=2$ с цифрами $0,1$), шестнадцатеричной ($m=16$, и в качестве цифр используются значки $0,1,2,\ldots,9,A,B,C,D,E,F$) и других системах счисления, основанных на том же принципе.

\ex В какой системе счисления справедливо равенство $3\cdot4=10$?
\sol 3 и 4 -- это цифры. Значит, искомое основание $m$ системы счисления больше 4. при этом $3\cdot4=12$ в десятичной системе счисления -- значит, число 12 в $m$-ичной системе счисления записывается как 10, отсюда $12=1\cdot m^1+0\cdot m^0$, откуда $m=12$.\\
Ответ: в двенадцатеричной.\\

\ex Вася задумал три различные цифры, отличные от нуля. Петя записал все возможные двузначные числа, в десятичной записи которых использовались только эти цифры. Сумма записанных чисел равна 231. Найдите цифры, задуманные Васей.
\sol Пусть $а, b, c$ -- три цифры, задуманные Васей. Существует девять двузначных чисел, в десятичной записи которых используются только эти цифры: $\overline{ab},\overline{ac},\overline{ba},\overline{ca},\overline{bc},\overline{cb},\overline{aa},\overline{bb},\overline{cc}$. Найдем их сумму, разложив каждое из чисел в виде суммы разрядных слагаемых: $(10a + a) + (10b + b) + (10c + c) + (10a + b) + (10b + a) + (10a + c) + (10c + a) + (10b + c) + (10c + b) = 33a + 33b + 33c = 33(a + b + c)$. По условию, $33(a + b + c) = 231$, то есть $a + b + c = 7$. Существует единственная тройка различных и отличных от нуля цифр, сумма которых равна 7, -- это 1, 2, 4.\\
Ответ: 1, 2, 4.\\

\ex Какое наименьшее число гирь требуется, чтобы с их помощью можно было взвесить на чашечных весах без стрелок любое целое число граммов от 1 до 120?
\sol Пусть взвешиваемый груз лежит на левой чаше весов. Сначала покажем, что четырех гирь недостаточно. Для каждой гири есть 3 варианта расположения: на правой чаше, на левой чаше, либо ни на одной из них. Таким образом, если гирь не более чем 4 (пусть их ровно 4, а в случае необходимости добавим недостающее число гирь массой по 1 грамму), мы имеем $3^4=81$ -- таково наибольшее количество различных по массе грузов, которые можно взвесить. Очевидно, мы не сможем взвесить любой груз от 1 до 120 граммов.\\
Теперь докажем, что 5 гирь с массами соответственно 1, 3, 9, 27 и 81 грамм будет достаточно.\\
\textsl{Лемма.} Любое число в троичной системе счисления может быть представлено в виде разности двух чисел, каждое из которых записывается в этой системе счисления без использования цифры 2.\\
\textsl{Предлагаем читателю самостоятельно доказать эту лемму. Помните, что правила вычисления <<в столбик>> работают в любой системе счисления!}\\
Итак, пусть взвешиваемый грух имеет массу $n$ граммов. Запишем $n$ в виде разности двух чисел, записываемых в троичной системе без использования цифры 2: $n=a-b$. Очевидно, мы всегда можем добиться того, чтобы цифры в разных разрядах записи чисел $a,b$ были различны. Теперь поставим на правую чашу весов гири, сумма масс которых равна $a$, а на левую -- те, сумма масс которых равна $b$. Это всегда можно сделать, поскольку массы гирь, являясь степенями <<тройки>>, записываются в троичной системе как единица с (возможно) следующими после нее нулями.\\
Поскольку сумма масс гирь больше 120, мы можем взвесить любой груз массой до 120 граммов.\\
Ответ: 5.


\subsection*{Задачи}

\task Найти все натуральные числа, равные (в десятичной системе счисления) удвоенной сумме своих цифр.

\task Найти числа, равные (в $m$-ичной системе счисления) удвоенной сумме своих цифр.

\task Четырехзначное число начинается с цифры 6. Эту цифру переставили в конец числа. Полученное число оказалось на 1152 меньше исходного. Найдите исходное число.

\taskk \textsl{Турнир городов, 1985 г.} \\Натуральное число $n$ записано в десятичной системе счисления. Известно, что если какая-то цифра входит в эту запись, то $n$ делится нацело на эту цифру (0 в записи не встречается). Какое максимальное число различных цифр может содержать эта запись?

\taskk \textsl{Московская математическая олимпиада, 1945 г.} \\Двузначное число в сумме с числом, записанным теми же цифрами, но в обратном порядке, даёт полный квадрат. Найти все такие числа.

\taskk Докажите, что представление числа в позиционной системе счисления с фиксированным основанием -- единственно.

\taskk \textsl{Турнир городов, 1987 г.} \\Докажите, что существует число, сумма цифр квадрата которого более, чем в 1000 раз превышает сумму цифр самого числа.

\taskk \textsl{Московская математическая олимпиада, 1968 г.} \\Двухсотзначное число $89252525\ldots2525$ умножено на число $444x18y27$ ($x$ и $y$ -- неизвестные цифры). Оказалось, что 53-я цифра полученного числа (считая справа) есть 1, а 54-я -- 0. Найти $x$ и $y$.


\subsection{Взаимная простота. Признаки делимости на составные числа. Признак Паскаля}
\setcounter{tasknum}{0}
\setcounter{exnum}{0}

Наибольший общий делитель двух целых $a,b$ -- это буквально наибольшее целое $c$, являющееся делителем и $a$, и $b$. При этом мы считаем, что среди чисел $a,b$ хотя бы одно не равно нулю, иначе наибольший общий делитель (кратко -- НОД) не определен. Аналогично определяется наибольший общий делитель набора из более чем двух целых чисел.\\

Два целых числа называют \textbf{взаимно простыми}, если их наибольший общий делитель равен $1$. Понятие взаимно простых чисел будец центральным при определении признаков делимости на составные числа: так, если натуральное число делится и на 10, и на 100, это не означает, что оно делится на $10\cdot100=1000$ (приведите контрпример!). 

\ex Сформулировать и доказать признак делимости на 15.

\sol Для того, чтобы число делилось на 15, необходимо и достаточно, чтобы оно делилось на 3 и 5. Значит, число делится на 15 тогда и только тогда, когда в его десятичной записи последняя цифра кратна 5, и сумма всех цифр кратна 3.

\textbf{Признак Паскаля.} Пусть нам дано натуральное $A$, которое записывается как $\overline{a_na_{n-1}\ldots a_1a_0}$ в десятичной системе счисления. Сформулируем и докажем признак делимости на натуральное $m$. Для этого найдем\\
$r_1$ -- остаток от деления 10 на $m$;\\
$r_2$ -- остаток от деления $10r_1$ на $m$;\\
$r_3$ -- остаток от деления $10r_2$ на $m$;\\
...\\
$r_n$ -- остаток от деления $10r_{n-1}$ на $m$.\\
Остатков $r_i$ при делении на $m$ получится не более чем $m$ штук -- значит, последовательность $\{r_i\}$ будет циклической: ее элементы будут повторяться, начиная с некоторого момента. 
$$A\pmod m\equiv \overline{a_na_{n-1}\ldots a_1a_0}\pmod m\equiv \overline{a_na_{n-1}\ldots a_1}\cdot10+a_0\pmod m\equiv $$ $$ \overline{a_na_{n-1}\ldots a_1}\cdot r_1+a_0\pmod m\equiv \overline{a_na_{n-1}\ldots a_2}\cdot r_2+a_1r_1+a_0\pmod m\equiv$$ $$\cdots \equiv a_nr_n+a_{n-1}r_{n-1}+\cdots+a_1r_1+a_0\pmod m$$
Таким образом, $A$ имеет тот же остаток при делении на $m$, что и число $a_nr_n+a_{n-1}r_{n-1}+\cdot+a_1r_1+a_0r_0$. Для единообразия будем считать $r_0=1$.\\

\ex Сформулируйте и докажите признак делимости на 3.

\sol Заметим, что для любого натурального $k$ число $10^k$ дает остаток 1 при делении на 3 -- значит, в признаке Паскаля для $m=3$ получим $r_1=r_2=\cdots=1$. Значит, любое натуральное $A=\overline{a_na_{n-1}\ldots a_1a_0}$ дает тот же остаток при делении на 3, что и сумма его цифр $a_n+a_{n-1}+\cdots+a_1+a_0$.\\
Итак, для того, чтобы число делилось на 3, необходимо и достаточно, чтобы сумма цифр его десятичной записи делилась на 3.\\

\ex Сформулируйте и докажите признак делимости на 7.

\sol Применим признак Паскаля для $m=7$:\\
$r_1=3$ -- остаток от деления 10 на $7$;\\
$r_2=2$ -- остаток от деления $10r_1=30$ на $7$;\\
$r_3=6$ -- остаток от деления $10r_2=20$ на $7$;\\
$r_4=4$ -- остаток от деления $10r_3=60$ на $7$;\\
$r_5=5$ -- остаток от деления $10r_4=40$ на $7$;\\
$r_6=1$ -- остаток от деления $10r_5=50$ на $7$.\\
Итак, цикл замкнулся: $r_6=r_0=1$. Получаем, что число $A=\overline{a_na_{n-1}\ldots a_1a_0}$ дает тот же остаток при делении на 7, что и число $a_0+3a_1+2a_2+6a_3+4a_4+5a_5+a_6+\cdots$.\\


\subsection*{Задачи}

\task Докажите, что если $a,b$ -- взаимно простые числа, и $c$ делится на каждое из них, то $c$ делится на $ab$.

\task Сформулируйте и докажите признак делимости на 6.

\task Сформулируйте и докажите признак делимости на 60.

\task Докажите, что НОД$(a,b)\cdot$НОК$(a,b)=a\cdot b$.

\task Сформулируйте и докажите признак делимости двоичного числа на $2^n$ для произвольного натурального $n$.

\taskk Дано шестизначное число $\overline{abcdef}$, причем $\overline{abc}-\overline{def}$ кратно $7$. Докажите, что и само число кратно $7$.

\taskk Докажите, что если необходимый и достаточный признак делимости, выражающийся через свойства цифр числа, не зависит от порядка цифр, то это признак делимости на 3 или на 9.

\taskkk \textsl{Московская математическая олимпиада, 1967 г.} Задано такое натуральное число $A$, что для любого натурального $N$, делящегося на $A$, число $\overleftarrow{N}$ тоже делится на A. ($\overleftarrow{N}$ -- число, состоящее из тех же цифр, что и $N$, но записанных в обратном порядке). Доказать, что $A$ является делителем числа $99$.




\subsection{Алгоритм Евклида}
\setcounter{tasknum}{0}
\setcounter{exnum}{0}

Алгоритм Евклида -- это простой способ найти наибольший общий делитель двух целых чисел. Итак, пусть нам даны целые $a,b$ (для определенности, пусть $a>b$). Будем строить последовательность $r_0,r_1,r_2,\ldots,r_{n+1}$ следующим образом:
\begin{itemize}
\item $r_0=a$, $r_1=b$ (первый <<шаг>> алгоритма);
\item для каждого натурального $k>1$ на $k$-м шаге получим $r_k$ -- остаток от деления $r_{k-2}$ на $r_{k-1}$;
\item $r_{n+1}=0$;
\item $r_n$ -- искомый наибольший общий делитель $a$ и $b$.
\end{itemize}
Для доказательства корректности этого алгоритма достаточно показать, что если $a=bq+r$, то НОД$(a,b)=$НОД$(b,r)$.

\textsl{Доказательство.} Пусть $k$ -- какой-либо делитель $a$ и $b$, тогда $a=uk, b=vk$, где $u,v$ -- целые числа. Тогда $k$ является делителем $r=a-bq=(u-vq)k$, причем обратное утверждение тоже верно: если $k$ делит $b,r$, то $k$ делит $a,b$. Поскольку это верно для любого делителя $k$, это верно и для наибольшего делителя чисел $a,b$, что и требовалось доказать.

Из самой конструкции алгоритма Евклида следует 

\textbf{Лемма Безу.} Даны целые взаимно простые $a,b$. Тогда существуют такие целые $x,y$, что $ax+by=1$.\\
\textsl{Доказательство.} Очевидно, для сущестования решения уравнения $ax+by=1$ необходимо и достаточно, чтобы для некоторого целого $x$ выражение $ax-1$ было кратно $b$. Рассмотрим остатки $t_k$ при делении $ak-1$ на $b$ при $k=0,1,\ldots,(|b|-1)$; предположим, что среди $t_k$ нет нуля. Тогда среди этих остатков есть не более $(|b|-1)$ различных чисел (поскольку всего существует $|b|$ остатков при делении на $b$, и по предположению мы исключаем 0), при том, что этих остатков $|b|$ штук -- согласно принципу Дирихле, найдутся такие $u,v$, что $au-1$ и $av-1$ дают равные остатки при делении на $b$. Не ограничивая общности, будем считать, что $0\le u<v<|b|$.\\
Рассмотрим их разность: $au-1-(av-1)=a(v-u)$ -- она кратна $b$. Но это невозможно, т.к. $a$ взаимно просто с $b$, и $0<v-u<|b|$ -- получили противоречие, значит, предположение о том, что среди остатков нет нуля, ложно.\\
Таким образом, лемма доказана.\\

\ex При помощи алгоритма Евклида найти НОД(228,132).
\sol Итак, $r_0=228$, $r_1=132$. Далее $r_2$ -- это остаток от деления $r_0$ на $r_1$, т.е. 96 -- значит, $r_2=96$; $r_3$ -- остаток от деления $r_1$ на $r_2$, т.е. $r_3=36$; продолжая, получим $r_4=24$, $r_5=12$, $r_6=0$ -- значит, НОД(228,132)=12.\\
Ответ: 12.\\

С алгоритмом Евклида связана еще одна интереснейшая конструкция -- цепная дробь. Чтобы лучше понять, что это такое, давайте рассмотрим дробь из предыдущего примера: $\frac{132}{228}$. Очевидно, $$\frac{132}{228}=\frac{1}{1+\frac{96}{132}}=\frac{1}{1+\frac{1}{1+\frac{36}{96}}}=\frac{1}{1+\frac{1}{1+  \frac{1}{2+ \frac{24}{36}}  }} = \frac{1}{1+\frac{1}{1+  \frac{1}{2+ \frac{1}{1+\frac{12}{24}}}  }} = \frac{1}{1+\frac{1}{1+  \frac{1}{2+ \frac{1}{1+\frac{1}{2}}}  }}$$
Наверняка вы заметили, что на каждом шаге мы получали те же частные и остатки, что возникали при работе алгоритма Евклида.\\

Вообще, \textbf{цепная дробь} -- это выражение вида $$[a_0;a_1,a_2,\ldots] =a_0+\frac{1}{a_1+\frac{1}{a_2+\cdots}}$$ для целого $a_0$ и натуральных $a_1, a_2,\ldots$. Любое вещественное число можно представить в виде цепной дроби, конечной или бесконечной. Очевидно, для рациональных чисел такая дробь будет конечной, т.е. последовательность $a_0,a_1,a_2,\ldots$ содержит конечное число членов. Однако, представление числа в виде цепной дроби неоднозначно: например, $\frac{7}{4}=1+\frac{1}{1+\frac{1}{3}}=[1;1,3]$ и в то же время $\frac74 = 1+\frac{1}{1+\frac{1}{2+\frac{1}{1}}}=[1;1,2,1]$.\\

\ex Один прибор делает синие пометки на длинной ленте через каждые $m$ см, другой -- красные пометки через каждые $n$ см ($m$ и $n$ -- взаимно простые). Верно ли, что какая-то синяя пометка окажется на расстоянии не больше 1 см от какой-то красной?
\sol Согласно лемме Безу, найдутся такие целые $x,y$, что $mx+ny=1$, при этом среди $x,y$ ровно одно отрицательно (не ограничивая общности, положим $y<0$). Таким образом, синяя пометка под номером $x$ окажется на расстоянии 1 см от красной пометки с номером $-y$.\\
Ответ: да, верно.


\subsection*{Задачи}

\task Используя алгоритм Евклида, найдите НОД(264,88).

\task Решите уравнение в натуральных числах: $$x+\frac{1}{y+\frac{1}{z}}=\frac{10}{7}$$

\task С 1 сентября четыре школьника начали посещать кинотеатр. Первый бывал в нем каждый четвертый день, второй -- каждый пятый, третий -- каждый шестой и четвертый -- каждый девятый. Когда второй раз все школьники встретятся в кинотеатре?

\task Докажите, что $18n+3$ и $27n+4$ взаимно просты при любом натуральном $n$.

\task Докажите, что дробь $\frac{n^2-n+1}{n^2+1}$ несократима при любом натуральном $n$.

\task Найдите все натуральные $n$, при которых число $\frac{n^3+n+1}{n^2-n+1}$ -- целое.

\taskk Докажите, что для нечетных $a,b,c$ выполнено равенство: НОД$(a,b,c)=$НОД$(\frac{a+b}{2},\frac{b+c}{2},\frac{c+a}{2})$.



\subsection{Неравенства в целых числах}
\setcounter{tasknum}{0}
\setcounter{exnum}{0}

\ex Известно, что доля учеников класса, успешно сдавших экзамен, заключена между $86,2\%$ и $87,9\%$. Чему равно наименьшее возможное число учеников в этом классе?
\sol Пусть $n$ -- число учеников в классе, $a$ -- число сдавших экзамен. Тогда, согласно условию задачи, $0.862<\frac an<0.879$, откуда $\frac{1000}{879}<\frac na<\frac{1000}{862}\Rightarrow 1+\frac{69}{431}<\frac{n-a}{a}<1+\frac{121}{879}\Rightarrow 6+\frac{52}{69}<\frac{a}{n-a}<7+\frac{32}{121}$. Отсюда $\frac{a}{n-a}=7\Rightarrow \frac{a}{n}=\frac78$. Заключаем, что искомое $n$ равно 8.\\
Ответ: 8.

\ex \textsl{Турнир городов, 2005 г.} \\Можно ли уместить два точных куба между двумя соседними точными квадратами?
\sol Пусть для натуральных $a,b,n$ выполнено $n^2<a^3<b^3<(n+1)^2$. Тогда $n^2<a^4$, а значит, $n<a^2$. Таким образом, $b^3\ge(a+1)^3>a^3+2a^2+1>n^2+2n+1=(n+1)^2$.\\
Ответ: нет, нельзя.


\subsection*{Задачи}

\task Найдите число сторон правильного $n$-угольника, если его внутренний угол $\alpha$ удовлетворяет неравенствам $158^{\circ}<\alpha<160^{\circ}$.

\task Найдите число сторон выпуклого $n$-угольника, если каждый его внутренний угол $\alpha$ удовлетворяет неравенствам $151^{\circ}\le\alpha\le153^{\circ}$.

\taskk На выборах кандидат получил больше $39.221\%$ и меньше $39.284\%$ голосов. Найдите наименьшее возможное число избирателей на этих выборах.

\taskk Найти наименьшее значение выражения $|36^k-5^l|$ для всевозможных натуральных $k,l$.

\taskkk \textsl{<<Физтех>>, 2020 г.}\\
Найдите количество пар целых чисел $(x,y)$,удовлетворяющих системе
\begin{equation*}
 \begin{cases}
   y>2^x+3\cdot2^{65}\\
   y\le70+(2^{64}-1)x
 \end{cases}
\end{equation*}

\taskkk \textsl{Всероссийская олимпиада по математике, 2014 г.} \\Натуральные $a,x,y$, большие 100, таковы, что $y^2-1=a^2(x^2-1)$. Какое наименьшее значение может принимать дробь $\frac ax$?


\newpage
\section{Некоторые теоретико-числовые теоремы}

\subsection{Арифметика остатков. Сравнения по модулю и их свойства}
\setcounter{tasknum}{0}
\setcounter{exnum}{0}

Будем говорить, что целые числа $x,y$ \textbf{сравнимы по модулю} натурального $c$, если $x,y$ дают равные остатки при делении на $c$. Это записывается так: $x\equiv y\pmod c$. Другими словами, существует целое $k$, при котором $x-y=ck$.\\
Из определения следует, что 
\begin{itemize}
\item любые числа сравнимы по модулю $1$;
\item если $x\equiv y\pmod c$, $y\equiv z\pmod c$, то $x\equiv z\pmod c$;
\item если $x\equiv y\pmod c$ и натуральное $d$ является делителем $c$, то $x\equiv y\pmod d$;
\item если $x$ и $y$ сравнимы по модулям $c_1,c_2,\ldots,c_n$, то они сравнимы по модулю $[c_1,c_2,\ldots,c_n]$ (наименьшее общее кратное чисел $c_1,c_2,\ldots,c_n$).
\end{itemize}

Сравнения можно складывать и перемножать: если $x_1\equiv y_1\pmod c$ и $x_2\equiv y_2\pmod c$, то $x_1+x_2\equiv y_1+y_2\pmod c$, $x_1x_2\equiv y_1y_2\pmod c$.\\
\textsl{Доказательство.} Из формулировки утверждения следует, что $x_1-y_1=mc$, $x_2-y_2=nc$ для некоторых целых $m,n$. Тогда $x_1+x_2-(y_1+y_2)=mc+ nc=(m+ n)c$, из чего следует, что $x_1+x_2\equiv y_1+y_2\pmod c$. Аналогично, $x_1-x_2-(y_1+y_2)=mc- nc=(m- n)c$, откуда $x_1-x_2\equiv y_1-y_2\pmod c$.\\
Также $x_1x_2-y_1y_2=x_1x_2-y_1y_2+x_1y_2-x_1y_2=x_1(x_2-y_2)+y_2(x_1-y_1)=x_1\cdot nc+ y_2\cdot mc=(x_1\cdot n+ y_2\cdot m)\cdot c$, из чего следует $x_1x_2\equiv y_1y_2\pmod c$. Утверждение доказано.\\

\ex Докажите, что $n^2+1$ не кратно 3 ни при каком натуральном $n$. 
\textsl{Доказательство.} Пусть $n$ дает остаток $r$ при делении на 3. Тогда $n\equiv r\pmod3\Rightarrow n^2\equiv r^2\pmod 3\Rightarrow n^2+1\equiv r^2+1\pmod 3$. перебирая возможные значения $r$ (0, 1, 2), убеждаемся в том, что $n^2+1$ не кратно 3, что и требовалось доказать.

\ex Существует ли натуральное число вида 111...1, кратное 2021?
\sol Введем обозначение: пусть $a_n$ -- это число, десятичная запись которого состоит из $n$ <<единиц>>. Рассмотрим числа $a_1,a_2,\ldots,a_{2022}$ -- согласно принципу Дирихле, в этом списке найдутся по крайней мере два числа $a_k,a_l$ (пусть $k>l$), дающие равные остатки при делении на 2021. Тогда $a_k-a_l=11....100....00=a_{k-l}\cdot10^l$. Поскольку числа 2021 и 10 взаимно просты, число $a_{k-l}$ кратно 2021.\\
Ответ: да, существует.

\ex \textsl{Турнир городов, 2014 г.} \\Назовём натуральное число ровным, если в его записи все цифры одинаковы (например: 4, 111, 999999).
Докажите, что любое $n$-значное число можно представить как сумму не более чем $n+1$ ровных чисел.
\textsl{Доказательство.} Пусть  $A_n = 1\ldots1$  ($n$ единиц). Докажем по индукции более сильное утверждение: \textsl{любое число $a\le A_n$ можно представить как сумму не более чем $n$ ровных чисел.}\\
База ($n = 1$) очевидна.\\
Шаг индукции. Число $A_n+1$ -- ровное. Если же $a\le A_{n+1}-1=10\cdot A_n$, то $a$ можно записать в виде $qA_n+r$, где $0\le q\le 9$, $0\le r\le A_n$. Число $qA_n$ -- ровное, а $r$ можно представить как сумму не более чем $n$ ровных чисел по предположению индукции, что завершает доказательство.


\subsection*{Задачи}

\task Найдите все возможные остатки при делении точного куба на 7.

\task Может ли сумма трех последовательных натуральных чисел быть простым числом?

\task При каких натуральных $n$ число $6^n+1$ кратно $7$?

\task Найдите последнюю цифру десятичной записи числа $243^{17}\cdot1376^{1375}\cdot2022^{2023}$.

\taskk Докажите, что если $x^2+y^2$ делится на 7 для целых $x,y$, то и сами числа $x,y$ кратны 7.

\taskk Докажите, что $n^3-n$ делится на 6.

\taskk Натуральные числа $m$ и $n$ таковы, что $m > n$, $m$ не делится на $n$ и имеет от деления на $n$ тот же остаток, что и $m + n$ от деления на $m – n$. Найдите отношение $m : n$.

\taskk \textsl{Турнир городов, 1996 г.}\\
Существует ли такое шестизначное число $A$, что среди чисел $A, 2A, \ldots, 500000A$ нет ни одного числа, оканчивающегося шестью одинаковыми цифрами?

\taskkk Для натуральных $n$ и $k$ (причем $0 < k \le n$) даны $k$ чисел, взаимно простых с $n$. Докажите,  что среди сумм некоторых из этих $k$ чисел встретится не менее $k$ различных остатков от деления на $n$. 

\taskkk \textsl{Московская математическая олимпиада, 2021 г.}\\
Пусть $p$ и $q$ -- взаимно простые натуральные числа. Лягушка прыгает по числовой прямой, начиная в точке 0, каждый раз либо на $p$ вправо, либо на $q$ влево. Однажды лягушка вернулась в 0. Докажите, что для любого натурального $d<p+q$ найдутся два числа, посещенные лягушкой и отличающиеся на $d$.


\subsection{Китайская теорема об остатках}
\setcounter{tasknum}{0}
\setcounter{exnum}{0}

Эта теорема помогает установить существование и единственность (с точностью до сравнимости по модулю) решения системы сравнений.\\

\textbf{Китайская теорема об остатках.} Даны попарно взаимно простые $a_1,a_2,\ldots,a_n$ и набор целых $r_1,r_2,\ldots,r_n$, таких, что $0\le r_i<a_i$ для всех $i=1,2,\ldots,n$. Тогда система сравнений 
\begin{equation*}
 \begin{cases}
   N\equiv r_1\pmod {a_1}\\
   N\equiv r_2\pmod {a_2}\\
   \ldots \\
   N\equiv r_n\pmod {a_n}
 \end{cases}
\end{equation*}
имеет единственное с точностью до сравнения по модулю $a_1a_2\cdots a_n$ решение.\\

\textsl{Доказательство.} Проведем доказательство индукцией по $n$. При $n=1$ утверждение теоремы очевидно. Пусть оно верно для $n=k-1$, тогда найдется его решение $m$. Введем обозначение $d=a_1a_2\cdots a_{k-1}$ и выберем произвольное $a_k$, взаимно простое с каждым из $a_i$ для всех $i=1,2,\ldots,k-1$. Рассмотрим множество $M=\{m,m+d,m+2d,\ldots m+(a_k-1)\cdot d\}=\{m+t\cdot d\}_{0\le t<a_k}$ и покажем, что все $t\in \{0,1,\ldots,a_k-1\}$ являются остатками при делении элементов $M$ на $a_k$.\\
Пусть это не так, т.е. найдется $r_k<a_k$, которое не принадлежит множеству остатков при делении элементов $M$ на $a_k$. Поскольку количество этих элементов равно $a_k$, а возможных остатков при делении элементов $M$ на $a_k$ не более чем $a_k-1$ (ведь ни одно число не дает остаток $r_k$ по предположению), то, согласно принципу Дирихле, найдутся два числа, имеющие равные остатки (пусть им соответствуют $t_1,t_2$). Тогда их разность $(m+t_1d)-(m+t_2d)=(t_1-t_2)d$ делится на $a_k$, что невозможно, т.к. $t_1,t_2<a_k$ и $d$ взаимно просто с $a_k$. Получили противоречие, а значит, среди рассматриваемых чисел найдется $N=m+td$, которое при делении на $a_k$ дает нужный остаток $r_k$ -- такое $N$ является решением системы сравнений.\\
Теперь докажем единственность. Действительно, пусть $N_1,N_2$ -- решения данной системы сравнений. Тогда $N_1\equiv N_2\equiv r_i\pmod {a_i}$ для всех $i=1,2,\ldots,n$, то есть $N_1-N_2$ делится на все $a_i$, что доказывает сравнение $N_1\equiv N_2\pmod {a_1a_2\cdots a_n}$.\\
Теорема доказана.


\subsection*{Задачи}

\task Решите систему сравнений
\begin{equation*}
 \begin{cases}
   x\equiv 3\pmod {5}\\
   x\equiv 11\pmod {17}
 \end{cases}
\end{equation*}

\task Найдите такое наименьшее чётное натуральное число $a$, что $a + 1$ делится на 3, $a + 2$ -- на 5, $a + 3$ -- на 7, $a + 4$ -- на 11, $a + 5$ -- на 13.

\task На столе лежат книги, которые надо упаковать. Если их связать в одинаковые пачки по 4, по 5 или по 6 книг, то каждый раз останется одна лишняя книга, а если связать по 7 книг в пачку, то лишних книг не останется. Какое наименьшее количество книг может быть на столе?

\taskk В китайской натурофилософии выделяются пять первоэлементов природы -- дерево, огонь, металл, вода и земля, которым соответствуют пять цветов -- синий (или зелёный), красный, белый, чёрный и жёлтый. В восточном календаре с древних времен используется 12-летний животный цикл так, что каждому из 12 годов в цикле соответствует одно из животных. Кроме того, каждый год проходит под покровительством одной из стихий и окрашивается в один из цветов:\\
  годы, оканчивающиеся на 0 и 1 -- годы металла (цвет белый);\\
  годы, оканчивающиеся на 2 и 3 -- это годы воды (цвет чёрный);\\
  годы, оканчивающиеся на 4 и 5 -- годы дерева (цвет синий);\\
  годы, оканчивающиеся на 6 и 7 -- годы огня (цвет красный);\\
  годы, оканчивающиеся на 8 и 9 -- годы земли (цвет жёлтый).\\
В 60-летнем календарном цикле каждое животное возникает пять раз. С помощью китайской теоремы об остатках объясните, почему оно все пять раз бывает разного цвета.

\taskk Генерал хочет построить для парада своих солдат в одинаковые квадратные каре (конечно, в каре должно быть более одного человека), но он не знает сколько солдат (от 1 до 37) находится в лазарете. Докажите, что у генерала может быть такое количество солдат, что он, независимо от заполнения лазарета, сумеет выполнить свое намерение. Например войско из 9 человек можно поставить в виде квадрата $3\times 3$, а если один человек болен, то в виде двух квадратов $2\times2$.

\taskk При каких целых $n$ число $n^2 + 3n + 1$ делится на 55?

\taskkk Докажите, что для каждого натурального $n$ найдутся $n$ последовательных натуральных чисел, не являющихся степенями простых чисел.



\subsection{Некоторые арифметические функции}
\setcounter{tasknum}{0}
\setcounter{exnum}{0}

В теории чисел часто используют специальные арифметические функции, определенные на натуральных числах и принимающие (чаще всего) натуральные значения. Некоторые из этих функций нам пригодятся, но сначала познакомимся с их определениями.\\
$\tau(n)$ -- функция числа натуральных делителей натурального $n$.\\
$\sigma(n)$ -- сумма натуральных делителей натурального $n$.\\
$\phi(n)$ (функция Эйлера) -- количество натуральных чисел, меньших $n$ и взаимно простых с ним. При этом по определению принимают $\phi(1)=1$.
%%%%%%%%%%%%%%%%%%%%%%%%%%%%%%%%%%%%%%%%%%%%


\subsection*{Задачи}

\task Найдите три натуральных числа, имеющих нечетное число делителей.

\task Дано каноническое разложение $n=p_1^{k_1}p_2^{k_2}\cdots p_s^{k_s}$ (простые числа $p_i$ попарно различны). Найдите $\tau(n)$.

\task Докажите, что для простого $p$ и $n=p^k$ выполнено $\phi(n)=p^k-p^{k-1}$.

\task Решите уравнение: $\phi(x)=\frac x3$

\taskk Докажите, что для взаимно простых $m,n$ выполнено $\tau(mn)=\tau(m)\tau(n)$.

\taskk Докажите, что для взаимно простых $m,n$ выполнено $\sigma(mn)=\sigma(m)\sigma(n)$.

\taskk Докажите, что для взаимно простых $m,n$ выполнено $\phi(mn)=\phi(m)\phi(n)$.

\taskk Дано каноническое разложение $n=p_1^{k_1}p_2^{k_2}\cdots p_s^{k_s}$ (простые числа $p_i$ попарно различны). Найдите $\phi(n)$.

\taskk Докажите, что число делителей $n$ не превосходит $2\sqrt n$.




%\subsection{Функция Эйлера}
%%%%%%%%%%%%%%%%%%%%%%%%%%%%%%%%%%%%%%%%%%%%%%%%%%%

\subsection{Теорема Эйлера и малая теорема Ферма}
\setcounter{tasknum}{0}
\setcounter{exnum}{0}

\textbf{Теорема Эйлера.} Если НОД$(a,m)=1$, то $$a^{\phi(m)}\equiv 1\pmod m$$
\textsl{Доказательство.} Пусть $x_1,x_2,\ldots,x_{\phi(m)}$ -- все различные натуральные числа, меньшие $m$ и взаимно простые с ним.
Поскольку $a$ взаимно просто с $m$, и $x_{i}$ взаимно просто с $m$, то и $x_i a$ также взаимно просто с $m$, то есть $x_i a\equiv x_j\pmod m$ для некоторого $j$.
Действительно, пусть это не так, тогда существуют такие $i_1 \ne i_2$, что $x_{i_1}a\equiv x_{i_2}\pmod m$, или $(x_{i_1}-x_{i_2})a\equiv 0\pmod m$.
Так как $a$ взаимно просто с $m$, последнее сравнение равносильно $x_{i_1}\equiv x_{i_2}\pmod m$, что противоречит попарной различности $x_i$ по модулю $m$.\\
Рассмотрим все возможные произведения $x_i a$ для всех $i$ от $1$ до $\phi(m)$ и отметим, что все остатки $x_i a$ при делении на $m$ различны.
Перемножим все сравнения вида $x_i a\equiv x_j\pmod m$ и получим $x_1\cdots x_{\phi(m)}a^{\phi(m)}\equiv x_1\cdots x_{\phi(m)}\pmod m$, или $$x_1\cdots x_{\phi(m)}(a^{\phi(m)}-1)\equiv 0\pmod m$$
Поскольку $x_1\cdots x_{\phi(m)}$ взаимно просто с $m$, получим $a^{\phi(m)}\equiv 1\pmod m$, что и требовалось доказать.

\textbf{Малая теорема Ферма.} Пусть $p$ -- простое число; $a$ -- натуральное, не кратное $p$. Тогда $$a^{p-1}\equiv 1\pmod p$$
Докажите эту теорему самостоятельно, используя утверждение теоремы Эйлера.

\subsection*{Задачи}

\task Найдите остаток от деления $7^{100}$ на 101.

\task Верно ли, что $300^{3000}-1$ кратно 1001?

\task Докажите, что $7^{120}-1$ кратно 143.

\task Докажите, что число $30^{239}+239^{30}$ -- составное.

\task $p$ -- простое число. Сколько существует способов раскрасить вершины правильного $p$-угольника в $a$ цветов? (Раскраски, которые можно совместить поворотом, считаются одинаковыми.)

\taskk Докажите, что если сумма трех целых чисел кратна 30, то сумма пятых степеней этих чисел также кратна 30.

\taskk Найдите три последние цифры числа $7^{2020}$.

\taskk Докажите, что для всякого простого $p>5$ число $1\ldots1$ ($p-1$ единиц в десятичной записи) кратно $p$.

\taskkk \textsl{IMO, 2003 г.} Пусть $p$ -- простое число. Докажите, что при некотором простом $q$ все числа вида $n^p-p$ не кратны $q$.


\newpage
\section{Диофантовы уравнения}

Во многих задачах требуется найти количество определенных объектов, удовлетворяющих условию, либо необходимо, чтобы решение было кратно определенному числу, либо было полным квадратом и т.п. Значит, ответами на вопросы таких задач должны быть целые числа, а то и натуральные. Дело может осложняться и тем, что количество условий (читай -- <<уравнений>>) уступает количеству неизвестных и может иметь целое пространство решений над вещественными числами, и это пространство может быть многомерным и очень сложным по структуре. А что если перед нами кубическое уравнение с несколькими неизвестными? В таких условиях на первый план выходит не умение проводить алгебраические преобразования, а соображения из теории чисел.\\

Итак, \textbf{диофантовы уравнения -- это уравнения в целых числах}: мы ищем только целые корни, и для этого нужно использовать специальные методы. Математики озадачились диофантовыми уравнениями много столетий назад. Вот лишь несколько примеров: 
\begin{itemize}
\item пифагоровы тройки -- целочисленные решения уравнения $a^2+b^2=c^2$ -- интересовали математиков уже в древнем Вавилоне ок. 4 тыс. лет назад;
\item великая теорема Ферма, ставящая вопрос об отсутствии ненулевых целочисленных решений уравнения $a^n+b^n=c^n$ для целых $n>2$ и сформулированная в сер. XVII века, волновала величайшие умы вплоть до кон. XX века, когда была наконец доказана;
\item 10-я проблема Гильберта, вошедшая в 1900 г. в его список из 23 важнейших проблем математики и ставящая вопрос о существовании общего алгоритма решения диофантовых уравнений, была решена лишь 70 лет спустя с неутешительным вердиктом: нет, общего алгоритма существовать не может.
\end{itemize}

В этом разделе мы и не станем искать общий алгоритм решения диофантовых уравнений, а разберем лишь некоторые их типы и для них укажем способы решения. Первый шаг на этом пути --


\subsection{Линейные диофантовы уравнения с двумя неизвестными}
\setcounter{tasknum}{0}
\setcounter{exnum}{0}

\textbf{Линейное диофантово уравнение с двумя неизвестными} -- это уравнение вида $ax+by=c$ для целочисленных неизвестных $x,y$ и (тоже целых) чисел $a,b,c$. Давайте условимся, что $a^2+b^2\ne0$, иначе получим равенство вида $ax=c$ либо $0=c$ -- оба случая тривиальны, и мы предоставим читателю самостоятельно разобрать их. Итак, далее считаем $a\ne0$ и $b\ne0$.\\

\textbf{Теорема 1} (о существовании решений линейного диофантова уравнения с двумя неизвестными). Уравнение $ax+by=c$ с целыми $a,b,c$ имеет целочисленное решение $(x,y)$ тогда и только тогда, когда $c$ кратно НОД$(a,b)$.\\
\textsl{Доказательство.} Очевидно, $ax$ и $by$ для любых целых $x,y$ кратно НОД$(a,b)$, а значит, $ax+by=c$ тоже кратно НОД$(a,b)$. \\
Теперь докажем, что если $c$ кратно НОД$(a,b)$, то решение уравнения $ax+by=c$ существует. Будем считать, что числа $a,b$ взаимно простые, иначе поделим обе части равенства на НОД$(a,b)$. \\
Сначала рассмотрим уравнение $ax+by=1$ -- существование его решения вытекает из леммы Безу, доказанной в параграфе 1.5 <<Алгоритм Евклида>>. Итак, пусть $ax_0+by_0=1$. Тогда, очевидно, пара $(cx_0,cy_0)$ является решением исходного уравнения.\\ 
\\

Мы знаем, что решить уравнение -- значит найти все его корни. Так сколько же корней имеет линейное диофантово уравнение с двумя неизвестными?\\

\textbf{Теорема 2} (о корнях линейного диофантова уравнения с двумя неизвестными). Если $(x_0,y_0)$ -- корень уравнения $ax+by=c$, то для любого целого $t$ пара $(x_0+bt,y_0-at)$ -- тоже корень уравнения. Более того, любой корень этого уравнения может быть представлен в виде $(x_0+bt,y_0-at)$ для некоторого целого $t$.\\
\textsl{Доказательство.} Первая часть утверждения теоремы проверяется непосредственной подстановкой: $$a(x_0+bt)+b(y_0-at)=ax_0+by_0+abt-abt=ax_0+by_0=c$$
Докажем, что любой корень исходного уравнения может быть представлен в виде $(x_0+bt,y_0-at)$ для некоторого целого $t$. Итак, пусть пары $(x_0,y_0)$, $(x,y)$ -- корни этого уравнения. Тогда, очевидно, $ax+by=ax_0+by_0$, откуда $a(x-x_0)=b(y_0-y)$. Без ограничения общности можно считать, что НОД$(a,b)=1$ -- значит, $x-x_0$ кратно $b$; $y_0-y$ кратно $a$. Тогда эти разности можно представить в виде $x-x_0=tb$, $y_0-y=ta$ для некоторого целого $t$, а следовательно, $x=x_0+bt$, $y=y_0-at$, что и требовалось доказать.\\
\\Итак, для решения линейного диофантова уравнения с двумя переменными нужно 
\begin{itemize}
\item убедиться, что у него есть корни (см. теорему 1);
\item найти хотя бы один из них;
\item задать остальные корни с помощью целочисленного параметра (см. теорему 2).
\end{itemize}
\ex Решить в целых числах уравнение $$2x-7y=4$$
\sol Заметим, что $4$ кратно НОД$(2,-7)=1$ -- значит, корни есть. Можно угадать один из них -- $(9,2)$ -- тогда остальные выражаются через него следующим образом: $(9-7t,2-2t)$.\\
\\
\ex Остаток от деления некоторого натурального числа $n$ на $6$ равен $4$, остаток от деления $n$ на $15$ равен $7$. Чему равен остаток от деления $n$ на $30$?
\sol Число $n$ дает остаток $4$ при делении на $6$ -- значит, $n=6x+4$ для некотрого целого $x\ge0$; аналогично, $n=15y+7$ для целого $y\ge0$. Из двух последних равенств получим $6x+4=15y+7$, из чего следует $2x-5y=1$. Нетрудно указать частное решение этого уравнения: например, $(-2,-1)$. Согласно теореме 2, любое решение последнего уравнения представимо в виде $(-2+5k,-1+2k)$ для некоторого целого $k$. Заметим, что $x,y$ -- натуральные числа, а значит, $k>0$.\\
Вернемся к $n$: $n=6x+4=6(-2+5k)+4=30k-8=30(k-1)+22$ -- значит, остаток от деления $n$ на $30$ равен $22$.

\subsection*{Задачи}


\task Имеет ли уравнение $7x-371y=54$ решение в целых числах?

\task Решите в целых числах уравнение $5x-18y=21$.

\task Решите в целых числах уравнение $11x+39y=76$.

\task Сколько решений имеет уравнение $14x+9y=273$ в натуральных числах? 

\task У кассира есть только $72$-рублевые купюры, а у вас -- только $105$-рублевые (у обоих неограниченное количество купюр).\\
а) Сможете ли вы уплатить кассиру один рубль?\\
б) А 3 рубля?

\task Фирма продавала чай в центре города по $7$ рублей, а кофе по $10$ рублей стакан, на вокзале -- по $4$ рубля и $9$ рублей, соответственно. Всего было продано за час $20$ стаканов чая и $20$ стаканов кофе, при этом выручка в центре и на вокзале оказалась одинаковой. Сколько стаканов кофе было продано в центре?

\task Тёма сделал несколько мелких покупок в супермаркете, имея при себе сто рублей. Давая сдачу с этой суммы кассир ошиблась, перепутав местами цифры, и выплатила рублями то, что должна была вернуть копейками, и, наоборот, копейками то, что полагалось вернуть рублями. Купив в аптеке набор пипеток за $1$ руб. $40$ коп., Тёма обнаружил ошибку кассира и, пересчитав деньги, нашел, что оставшаяся у него сумма втрое превышает ту, которую ему должны были вернуть в супермаркете. Какова стоимость всех покупок Тёмы?

\task Длина дороги, соединяющей пункты $A$ и $B$, равна $2$ км. По этой дороге курсируют два автобуса. Достигнув пункта $A$ или пункта $B$, каждый из автобусов немедленно разворачивается и следует без остановок к другому пункту. Первый автобус движется со скоростью $51$ км/ч, а второй -- со скоростью $42$ км/ч. Сколько раз за $8$ часов движения автобусы\\
а) встретятся в пункте $B$;\\
б) окажутся в одном месте строго между пунктами $A$ и $B$? \\
Известно, что первый автобус стартует из пункта $A$, а второй -- из пункта $B$.

\task Дан набор попарно взаимно простых чисел $m_1,m_2,\ldots,m_n$. Докажите, что любую правильную дробь вида $\frac{c}{m_1m_2\cdots m_n}$ можно представить как сумму правильных дробей вида $\frac{c_i}{m_i}$ ($i=1,2,\ldots,n$).

\taskk \textbf{(Теорема Сильвестра).} Дано диофантово уравнение $ax+by=c$ при взаимно простых $a$ и $b$. Докажите, что наибольшее $c$, для которого это уравнение не имеет решений в целых неотрицательных числах, имеет вид $c=ab-a-b$.

\taskk Отметим на прямой красным цветом все точки вида $81x+100y$, где $x,y$ -- натуральные, и синим цветом -- остальные целые точки. Найдите на прямой такую точку, что любые симметричные относительно неё целые точки окрашены в разные цвета.


%\subsection*{Тест}

%\begin{enumerate}
%\item Решите в целых числах уравнение $4x-13y=31$ и укажите в ответе $y$, при котором $x$ принимает наибольшее отрицательное значение.

%\item Найдите количество решений уравнения $8x+9y=8899$ в натуральных числах.

%\item Остаток от деления натурального $p$ на $12$ равен $7$, остаток от деления $p$ на $21$ равен $13$. Найдите остаток от деления $p$ на $84$.

%\item Найдите все наборы $(x,y,z)$ натуральных чисел, удовлетворяющих системе
%\begin{equation*}
% \begin{cases}
   %11x-6y=z
   %\\
   %z-y=7
   %\\
   %x\le20
% \end{cases}
%\end{equation*}
%Запишите в ответ $z$, при котором величина $x+y$ принимает наименьшее значение.

%\end{enumerate}


\subsection{Нелинейные диофантовы уравнения}
\setcounter{tasknum}{0}
\setcounter{exnum}{0}

\subsection*{Задачи}

\task Докажите, что уравнение $3x^2+2=y^2$ неразрешимо в целых числах.

\task Решите уравнение $x^2+1=py$ для целых $x,y$ ($p$ -- простое число, дающее остаток 3 при делении на 4).

\task Докажите, что уравнение $x!y!=z!$ имеет бесконечно много решений для $1<x\le y<z$.

\task Докажите, что если $\frac{1}{a}+\frac{1}{b}+\frac{1}{c}+\frac{1}{d}+\frac{1}{e}+\frac{1}{f}=1$ для целых $a,b,c,d,e,f$, то по крайней мере одно из этих чисел чётно.

\task Решите в целых числах уравнение $2^x+7=y^2$.

\taskk Решите в целых числах уравнение $3^x-2^y=7$.

\taskk Натуральные $a,x,y>100$ таковы, что $y^2-1=a^2(x^2-1)$. Найдите наименьшее возможное значение дроби $\frac ax$.

\taskk Решите в положительных рациональных числах уравнение $x^y=y^x$ при $x>y$.

\taskk Докажите, что уравнение $x^2+7y^2=2^n$ разрешимо в нечетных целых числах при любом натуральном $n>2$. 

\taskkk \textsl{Всероссийская олимпиада по математике, 1996 г.} Найдите все такие натуральные $n$, что при некоторых взаимно простых $x$ и $y$ и натуральном $k > 1$,  выполняется равенство $3^n = x^k + y^k$.


\subsection{Пифагоровы тройки}
\setcounter{tasknum}{0}
\setcounter{exnum}{0}

Будем называть \textbf{пифагоровой} такую упорядоченную тройку $(a,b,c)$ натуральных чисел, что $a^2+b^2=c^2$. Если при этом НОД$(a,b)=1$, то соответствующая тройка называется \textbf{примитивной}. Очевидно, что каждая пифагорова тройка либо является примитивной, либо получена из примитивной домножением каждого своего числа на некоторое $k$.

Формулы для генерации примитивных пифагоровых троек называются \textbf{формулами Евклида}, и о них говорит следующая
 
\textbf{Теорема.} Если для некоторых натуральных взаимно простых $m,n$ разной четности выполнено $m>n$, то $(m^2-n^2, 2mn, m^2+n^2)$ -- примитивная пифагорова тройка. И наоборот: для каждой примитивной пифагоровой тройки $(a,b,c)$ найдутся такие взаимно простые натуральные числа $m,n$ различной четности, что $a=m^2-n^2,b=2mn,c=m^2+n^2$.

\textsl{Доказательство.} Первая часть утверждения теоремы доказывается простой подстановкой: $(m^2-n^2)^2+(2mn)^2=(m^2+n^2)^2$. Взаимная простота тоже очевидна.\\
Докажем вторую часть утверждения теоремы. Во-первых, несложным перебором четностей чисел в тройке убеждаемся, что $c$ должно быть нечетным, а среди чисел $a,b$ -- ровно одно четное (пусть это будет $b$). $b^2=c^2-a^2=(c-a)(c+a)$, откуда $\frac{c+a}{b}=\frac{b}{c-a}$. Поскольку $\frac{c+a}{b}$ рационально, представим его в виде несократимой дроби $\frac mn$, откуда $\frac{c-a}{b}=\frac nm$. Решая уравнения $\frac cb+\frac ab=\frac mn$ и $\frac cb-\frac ab=\frac nm$, получим $\frac cb=\frac{m^2+n^2}{2mn}$, $\frac ab=\frac{m^2-n^2}{2mn}$. Числители и знаменатели обеих частей равенств будут равны тогда и только тогда, когда правые части равенств несократимы -- значит, числа $m,n$ имеют противоположную четность.\\
Теперь, приравнивая числители и знаменатели дробей, получим утверждение теоремы.


\subsection{Задачи для практики}

\task В плоскости расположено $n$ зубчатых колёс таким образом, что первое колесо сцеплено своими зубцами со вторым, второе -- с третьим и т.д. Наконец, последнее колесо сцеплено с первым. Могут ли вращаться колёса такой системы?

\task Из квадратного листа бумаги в клетку, содержащего целое число клеток, вырезали квадрат, содержащий целое число клеток так, что осталось 124 клетки. Сколько клеток мог содержать первоначальный лист бумаги?

\task Пусть $x\equiv a\pmod m$, $y\equiv b\pmod m$. Верно ли, что если $x$ кратно $y$, $a$ кратно $b$, то $x:y\equiv a:b\pmod m$?

\task Найдите последнюю цифру числа $333^{456}$.

\task Сколько различных натуральных делителей имеет число $2^5\cdot7^9\cdot11^4\cdot101^{11}$?

\task Решите в целых числах уравнение $10x^2+11xy+3y^2+7=0$.

\task Докажите, что для различных простых чисел $p,q$ выполнено $p^q+q^p\equiv p+q\pmod{pq}$.

\taskk Число 1047 при делении на $n$ дает остаток 23, а при делении на $n+1$ -- остаток 7. Найдите $n$.

\taskk Найдите наименьшее $k$, при котором уравнение $7x+9y=k$ имеет шесть натуральных решений.

\taskk Докажите, что если сумма трех целых чисел кратна 6, то сумма их кубов тоже кратна 6. 

\taskk Найдите натуральное число вида $n = 2^x3^y5^z$, зная, что половина его имеет на 30 делителей меньше, треть -- на 35 и пятая часть -- на 42 делителя меньше, чем само число.

\taskk Сформулируйте и докажите признак делимости на 13.

\taskk Докажите, что ни одно из чисел вида $10^{3n+1}$ нельзя представить в виде суммы двух кубов натуральных чисел.

\taskk Пусть $p>5$ -- простое число. Докажите, что число 1...1, состоящее из $p-1$ <<единицы>>, кратно $p$. Докажите, что 1...1 ($p$ <<единиц>>) не делится на $p$.

\taskk Пусть числа $a$ и $b$ взаимно просты. Докажите, что для того, чтобы уравнение $ax + by = c$ имело ровно $n$ целых положительных решений, значение $c$ должно находиться в пределах $(n - 1)ab + a + b \le c \le (n + 1)ab$.

\taskkk \textsl{Московская математическая олимпиада, 1976 г.} \\Существует ли такое натуральное число, что если приписать его к самому себе справа, то полученное число окажется полным квадратом?

\taskkk \textbf{(Теорема Ламе)} Пусть число $m_1$ в десятичной системе счисления записывается при помощи $n$ цифр.
Докажите, что при любом $m_0$ число шагов $k$ в алгоритме Евклида для чисел $m_0$ и $m_1$ удовлетворяет неравенству $k \le 5n$.







\end{document} 